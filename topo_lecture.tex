\documentclass[language=english]{TemplateLecture}

\renewcommand{\ProfName}{Stefan Schwede}
\renewcommand{\LectureName}{Algebraic Topology I}
\renewcommand{\Semester}{WiSe 2025/26}
\renewcommand{\mName}{Jan Malmström}

\begin{document}

\subsection*{Organizatorial}

For this term we will be doing unstable homotopy theory. Next term we will be doing stable homotopy theory. Note that there were 2 previous courses. Note that all important information is shared on the website \url{https://www.math.uni-bonn.de/people/schwede/at1-ws2526}. You can sign up for the previous topology courses und see the lecture videos for these courses there.

There are no lecture notes for this lecture specifically, but some similar materials are linked on the webpage.

Exercise sheets will be uploaded fridays and handed in 11 days later via eCampus. Registration for eCampus opens at 4 today.

For exam admission you will have to score 50\% of the points on the exercise sheets and have presented 2 exercises in tutorial.

The first exam will be written in the last week of semester.

I fear I will not be able to copy pictures here.

\newpage

\setcounter{chapter}{1}

\section{Blakiers-Massy theorem/Homotopy excision}

We start with a reminder on relative homotopy groups.

\begin{defi}{Relative Homotopy Groups}{relative Homotopy groups}
    Let \((X,A)\) be a space pair i.e. \(A\) is a subspace of a topological space \(X\). We write
    \[I = [0,1] \quad I^{n} = [0,1]^n \text{ the } n\text{-cube}\]
    \[\partial(I^n) = \text{ boundary of } I^n\]
    \[I^{n-1} \subseteq I^n\]
    via Inclusion on the first \(n-1\) coordinates.
    \[J^{n-1} = I^{n-1}\times \set{1} \cup (\partial I^{n-1})\times [0,1]\]
    He draws a picture for \(n = 2\).

    For \(n \geq 1\) the \(n\)-th relative homotopy groups \(\pi_n(X,A,x)\) is the set of triple homotopy classes of trible maps \(x \in A\subseteq X\)
    \[(I^n, \partial I^n,J^{n-1}) \to (X,A, \set{x})\]
    where a triple map takes each subset on the left into the subset on the right. A triple-homotopy must also conserve these conditions.

    For \(n \geq 2\) or \(n = 1\) and \(A = \set{x}\) the set \(\pi_n(X,A,\set{x})\) has a group structure by concatenation in the first coordinate. He again draws a picture.

    The group structure is commutative if \(n \geq 3\) or \(n = 2\) and \(A = \set{x}\).
\end{defi}

\begin{defi}{n-Connectedness}{n-Connectedness}
    Let \(n \geq 0\). A space pair \((X,A)\) is \emph{\(n\)-connected}, if the following equivalent conditions hold:
    \begin{enumerate}
        \item For all \(0 \leq q\leq n\) every pair map \((I^q, \partial I^q) \to (X,A)\) is homotpic relative \(\partial(I^q)\) to a map with image in \(A\)
        \item For all \(a \in A\), \(\incl_*\colon \pi_q(A,a) \to \pi_q(X,a)\) is bijective for \(q \leq n\) and surjective for \(q = n\).
        \item \(\pi_0(A) \to \pi_0(X)\) is bijective and for all \(1 \leq q \leq n\) the relative homotopy group
        \[\pi_q(X,A,x) \cong 0\]
    \end{enumerate}
\end{defi}

\begin{proof}
    You proof the equivalence using the LES\footnote{Long exact sequence} of homotopy groups.
\end{proof}


Let \(Y\) be a space, \(Y_1, Y_2\) open subsets of \(Y = Y_1 \cup Y_2\), \(Y_0 \coloneq Y_1\cap Y_2\).

Excision in homology shows that for all abelian groups \(B\), \(i \geq 0\)
\[H_i(Y_2; Y_0, B) \to H_i(Y,Y_1; B)\]
is an isomorphism.

Excision does not generally hold for homotopy groups, i.e. for \(x \in Y_0\)
\[\incl_*\colon \pi_i(Y_2, Y_0; x) \to \pi_i(Y, Y_1; x)\]
is \textbf{not} generally an isomorphism.

\enquote {Blakiers mAssing theorem implies that excision holds for homotopy groups in a range.}

\begin{thm}{Blakiers Massey}{Blakiers Massey}
    Let \(Y\) be a space, \(Y_1, Y_2\) open subsets with \(Y= Y_1 \cup Y_2, Y_0 \coloneq Y_1 \cap Y_2\). Let \(p, q \geq 0\), such that for all \(y \in Y_0\)
    \[\pi_i(Y_1, Y_0, y) = 0 \text{ for all } 1 \leq i \leq p\]
    and
    \[\pi_i(Y_2, Y_0, y) = 0 \text{ for all } 1 \leq i \leq q\]
    Then for all \(y \in Y_0\), the map
    \[\incl_*\colon \pi_i(Y_2, Y_0, y) \to \pi_i(Y, Y_1, y)\]
    is an isomorphism for \(1 \leq i < p+q\) and surjective for \(i = p+q\). He notes how the referenced literature uses different indices. They have proofs in more detail and pictures, however Lücks script contains typos
\end{thm}

\begin{proof}
    Schwede explains he doesn't like the proof, it is to technical and not very enlightening.

    We define what cubes are

    Cubes in \(\ir^n\), \(n \geq 1\).
    \(a = (a_1, \dots, a_n) \in \ir^n\) the \enquote{lower left corner of the cube}

    \(\partial \in \IR_{\geq 0}\) \enquote{side length of the cube}

    \(L \subset \set{1, \dots n}\) \enquote{relevant dimensions}
    \[W = W(a, \delta, L) = \set{x = (x_1, \dots, x_n)\in \ir^n : a_i\leq x_1\leq a_i + \delta \text{ for all } i \in L, \; x_i = a_i \text{ for all } i \in \set{1, \dots n} \setminus L}\]\footnote{W weil Würfel}
    A face \(W'\) of \(W\) is a subset of the form
    \[W' = \set{x \in W : x_i = a_i \text{ for all } i \in L_0, x_i = a_i + \delta \text{ for all } i \in L_1}\]
    for some subsets \(L_0, L_1 \subseteq L\)

    Let \(1 \leq p \leq n\) we define two subsets of a cube \(W = ^(a, \delta, L)\).
    \[K_p(W) = \set{x \in W : x_i < a_i + \delta/2 \text{ for at least} p \text{values of} i \text{ in } L}\]
    We call these \enquote{\(p\) small coordinates}
    \[G_p(W) = \set{x \in W : x_i > a_i + \delta/2 \text{ for at least } p \text{ coordinates } i \text{ in } L}\]
    these are \enquote{\(p\)-big coordinates}.

    For \(p > \dim(W), K_p(W) = G_p(W)= \emptyset \) If \(p+ q \geq \dim(W)\), then \(K_p(W) \cap G_q(W) = \emptyset\).%Help

    He draws pictures.

    \begin{lem}{1.14}{1.14}
        It is Lemma 1.14 in Lücks Script.

        Let \((Y,A)\) be a space pair, \(W \subseteq \ir^n\) a cube, \(f \colon W \to Y\) continuous. Suppose that for some \(p\leq \dim(W)\), \(f^{-1}(A) \cap W' \subseteq K_p(W')\) for all proper\footnote{subcube of the boundary} faces \(W'\) of \(W\).
        
        Then there is a continuous map \(g \colon W \to Y\) homotopic to \(f\) relative \(\partial W\) such that all \(g^{-1}(A)\subseteq K_p(W)\)
    \end{lem}
    \begin{proof}
        Wlog: \(W = I^n = W(0, 1, \set{1,\dots ,n})\)

        Let \(I_2^n\) be the subcube \([0, 1/2]^n\). He draws a picture. \(x_4 = (1/4, \dots 1/4) \in I_2^n\). We define a continuous map \(h \colon I^n \to I^n\) by radical projection away from \(x_4\). Picture. Let \(r(y)\) be the ray from \(x\) to \(y\). We map all of \(r(y) \cap I^n\setminus I^n_2\) to the intersection point of \(r(y)\) and \(\partial I^n\) and the rest linearly extends as far as required.\footnote{I hope this description is clear, hard without the picture.}

        Obviously\footnote{Meaning he's too lazy to come up with formulas for the map.} \(h\) is homotopic relative boundary \(\partial I^n\) to the identity.

        We set \(g \colon f \circ h\colon I^n \to Y\), which is then homotopic relative \(\partial(I^n)\footnote{I am very inconsistent in remembering these parantheses with the boundary operator. Just imagine it always being as here.}\) to \(f\).

        It remains to show that \(g^{-1}(A) \subseteq K_p(W)\). Consider \(z \in I^n\) with \(g(z) \in A\).
        \begin{description}
            \item[Case 1] for all \(i = 1, \dots, n, z_1 < 1/2\), i.e. \(z \in I_2^n\), then \(z \in K_n(I^n) \subseteq K_p(I^n)\).
            \item[Case 2] There is an \(i \in \set{1, \dots,n}\), s.t. \(z \geq 1/2\). Then \(h(z) \in \partial(I^n)\). Let \(W'\) be some proper face of \(W\), with \(h(z) \in W'\). Since \(f(h(z)) = g(z)\in A\), by hypothesis, \(h(z) \in K_p(W')\), so \(h(z) < 1/2\) for at least \(p\) coordinates. By expansion\footnote{no idea if this is the word he wrote} property of \(h\), also \(p\) coordinates of \(z\) are small coordinates. 
        \end{description}
    \end{proof}

    \begin{proposition}
        Let \(Y_1, Y_2\) be open subsets of \(Y, Y_0\coloneq Y_1 \cap Y_2\). Suppose that \((Y_1, Y_0)\) is \(p\)-connected, \((Y_2, Y_0)\) is \(q\)-connected. Let \(f \colon I^n \to Y\) be continuous. Let \(\cW = \set{W}\) be a subdivision of \(I^n\) into subcubest of the same side length s.t. for all \(W \in \cW\) \(f(W) \subseteq Y_1\) or \(f(W) \subseteq Y_2\). Then there is a homotopy \(h \colon I^n \times I \to Y\) with \(h_0 = f\) such that for all \(W \in \cW\):
        \begin{enumerate}
            \item If \(f(W) \subseteq Y_j, j \in {0,1,2}\), then \(h_t(W) \subseteq Y_j\) for all \(t \in [0,1]\)
            \item If \(f(W) \subseteq Y_0\), then \(h_t\rvert_{W} = f\rvert_W\), i.e. \(h\) is constant on \(W\).
            \item If \(f(W) \subseteq Y_1\), then \(h_1^{-1}(Y_1 \setminus Y_0) K_{p+1}(W)\).
            \item If \(f(W) \subseteq Y_2\), then \(h_1^{-1}(Y_2\setminus Y_0) \subseteq G_{q+1}(W)\).
        \end{enumerate}
    \end{proposition}
    \begin{proof}
        We let \(C^k \subseteq I^n\) be the union of all cubes in \(\cW\) of dimension at most \(k\). We construct homotopies \(h[k]\colon C_k\times I \to Y\), such that for all \(W \in \cW, W \subseteq C_k\) conditions 1. to 4. hold, and \(h[k]\) is constant on \(C_{k-1}\times I\). Then the final \(h[n]\) does the job.

        \textbf{Note.} If \(W \in \cW\) and \(f(W)\subseteq Y_0\) and 2. holds, then also 3. and 4. hold.

        \[h^{-1}(Y_1\setminus Y_0) = h_1^{-1}(Y_2\setminus Y_0) = \emptyset\]
        If \(W \in \cW\), is such that \(f(W) \subseteq Y_1\) and \(f(W) \subseteq Y_2\), then \(f(W)\subseteq Y_1\cap Y_2 = Y_0\). So each \(W \in \cW\) is in excactly one of the following cases
        \begin{itemize}
            \item \(f(W) \subseteq Y_0\)
            \item \(f(W) \subseteq Y_1\) and \(f(W) \not\subseteq Y_1\)
            \item \(f(W) \subseteq Y_2\) and \(f(^)\not \subseteq Y_2\)
        \end{itemize}
        Inductive construction \(k = 0\), i.e. vertexes of the cubes \(w \in \cW\). If \(w\in Y_2\), take \(h[0]_t = \const_{w_0}\).

        Suppose \(f(W_0)\in Y_1\), but \(f(w_0) \not \in Y_2\). Since \(Y_1,Y_0\) is \(0\)-connected, there is a path \(\pi\colon I \to Y_1\) from \(w_0\) to a point in \(Y_0\). We take \(h[0]\) as the path on \(w_0\). Analoguos if \(f(W_0)\in Y_2\setminus Y_1\).

        \textbf{Inductive Step} Let \(W \in \cW\) be a cube of ecact dimenios \(k\). Then \(\partial W = W \cap C_{k-1}\). Since \((W, \partial W)\) has the HEP, we can extend the previous homotopy \(h[k-1]\rvert_{\partial W}\) to some homotopy on \(W\) relative to \(f\rvert_W\). Let this be \(h'[k]\colon C_k\times I \to Y\): this satisfies conditions 1. and 2. but not yet 3. and 4.

        We produce another homotopy \(h[k]''\) and set \(h[k] = h[k]' * h[k]''\).

        Consider a cube \(W \in \cW\) of dimension \(k\).

        If \(f(W) \subseteq Y_0\) set \(h[k]''\) as the constand homotopy on \(W\).

        If \(h[k]_1'(W) \subseteq Y_1\), but \(h[k]_1'(W) \not \subseteq Y_2\) there is a homotopy relative \(\partial W\) from \(h[k]_1'\) to a map \(f_1(W)\subseteq Y_0\).

        If \(k = \dim(W) > p\) the we use the lemma \ref{lem:1.14} for \(f = h[k]_1'\rvert_W\) and the resulting homotopy is \(h[k]''\rvert_W\).

        If \(h[k]_1'(W)\subseteq Y_2\) but \(h[k]_1'(W) \not \subseteq Y_1\), use the complement case of the lemma\footnote{rest of the proof next lecture. I am very sure some words won't make sense, as they were unreadable on the board.}
    \end{proof}
\end{proof}



\end{document}