\documentclass[language=english]{TemplateLecture}

\renewcommand{\ProfName}{Stefan Schwede}
\renewcommand{\LectureName}{Algebraic Topology I}
\renewcommand{\Semester}{WiSe 2025/26}
\renewcommand{\mName}{Jan Malmström}

\makeatletter
\@ifundefined{BKM@entry}{\def\BKM@entry#1#2{}}{}
\makeatother


\begin{document}

\newLecture{13.10.2025}

\subsection*{Organizatorial}

For this term we will be doing unstable homotopy theory. Next term we will be doing stable homotopy theory. Note that there were 2 previous courses. Note that all important information is shared on the website \url{https://www.math.uni-bonn.de/people/schwede/at1-ws2526}. You can sign up for the previous topology courses und see the lecture videos for these courses there.

There are no lecture notes for this lecture specifically, but some similar materials are linked on the webpage.

Exercise sheets will be uploaded fridays and handed in 11 days later via eCampus. Registration for eCampus opens at 4 today.

For exam admission you will have to score 50\% of the points on the exercise sheets and have presented 2 exercises in tutorial.

The first exam will be written in the last week of semester.

I fear I will not be able to copy pictures here.

\newpage

\setcounter{chapter}{1}

\section{Blakiers-Massy theorem/Homotopy excision}

We start with a reminder on relative homotopy groups.

\begin{defi}{Relative Homotopy Groups}{relative Homotopy groups}
    Let \((X,A)\) be a space pair i.e. \(A\) is a subspace of a topological space \(X\). We write
    \[I = [0,1] \quad I^{n} = [0,1]^n \text{ the } n\text{-cube}\]
    \[\partial(I^n) = \text{ boundary of } I^n\]
    \[I^{n-1} \subseteq I^n\]
    via Inclusion on the first \(n-1\) coordinates.
    \[J^{n-1} = I^{n-1}\times \set{1} \cup (\partial I^{n-1})\times [0,1]\]
    He draws a picture for \(n = 2\).

    For \(n \geq 1\) the \(n\)-th relative homotopy groups \(\pi_n(X,A,x)\) is the set of triple homotopy classes of trible maps \(x \in A\subseteq X\)
    \[(I^n, \partial I^n,J^{n-1}) \to (X,A, \set{x})\]
    where a triple map takes each subset on the left into the subset on the right. A triple-homotopy must also conserve these conditions.

    For \(n \geq 2\) or \(n = 1\) and \(A = \set{x}\) the set \(\pi_n(X,A,\set{x})\) has a group structure by concatenation in the first coordinate. He again draws a picture.

    The group structure is commutative if \(n \geq 3\) or \(n = 2\) and \(A = \set{x}\).
\end{defi}

\begin{defi}{n-Connectedness}{n-Connectedness}
    Let \(n \geq 0\). A space pair \((X,A)\) is \emph{\(n\)-connected}, if the following equivalent conditions hold:
    \begin{enumerate}
        \item For all \(0 \leq q\leq n\) every pair map \((I^q, \partial I^q) \to (X,A)\) is homotpic relative \(\partial(I^q)\) to a map with image in \(A\)
        \item For all \(a \in A\), \(\incl_*\colon \pi_q(A,a) \to \pi_q(X,a)\) is bijective for \(q < n\) and surjective for \(q = n\).
        \item \(\pi_0(A) \to \pi_0(X)\) is bijective and for all \(1 \leq q \leq n\) the relative homotopy group
        \[\pi_q(X,A,x) \cong 0\]
    \end{enumerate}
\end{defi}

\begin{proof}
    You proof the equivalence using the LES\footnote{Long exact sequence} of homotopy groups.
\end{proof}


Let \(Y\) be a space, \(Y_1, Y_2\) open subsets of \(Y = Y_1 \cup Y_2\), \(Y_0 \coloneq Y_1\cap Y_2\).

Excision in homology shows that for all abelian groups \(B\), \(i \geq 0\)
\[H_i(Y_2; Y_0, B) \to H_i(Y,Y_1; B)\]
is an isomorphism.

Excision does not generally hold for homotopy groups, i.e. for \(x \in Y_0\)
\[\incl_*\colon \pi_i(Y_2, Y_0; x) \to \pi_i(Y, Y_1; x)\]
is \textbf{not} generally an isomorphism.

\enquote {Blakiers Massey theorem implies that excision holds for homotopy groups in a range.}

\begin{thm}{Blakiers Massey}{Blakiers Massey}
    Let \(Y\) be a space, \(Y_1, Y_2\) open subsets with \(Y= Y_1 \cup Y_2, Y_0 \coloneq Y_1 \cap Y_2\). Let \(p, q \geq 0\), such that for all \(y \in Y_0\)
    \[\pi_i(Y_1, Y_0, y) = 0 \text{ for all } 1 \leq i \leq p\]
    and
    \[\pi_i(Y_2, Y_0, y) = 0 \text{ for all } 1 \leq i \leq q\]
    Then for all \(y \in Y_0\), the map
    \[\incl_*\colon \pi_i(Y_2, Y_0, y) \to \pi_i(Y, Y_1, y)\]
    is an isomorphism for \(1 \leq i < p+q\) and surjective for \(i = p+q\). He notes how the referenced literature uses different indices. They have proofs in more detail and pictures, however Lücks script contains typos
\end{thm}

\begin{proof}
    Schwede explains he doesn't like the proof, it is to technical and not very enlightening.

    We define what cubes are

    Cubes in \(\ir^n\), \(n \geq 1\).
    \(a = (a_1, \dots, a_n) \in \ir^n\) the \enquote{lower left corner of the cube}

    \(\partial \in \IR_{\geq 0}\) \enquote{side length of the cube}

    \(L \subset \set{1, \dots n}\) \enquote{relevant dimensions}
    \[W = W(a, \delta, L) = \set{x = (x_1, \dots, x_n)\in \ir^n : a_i\leq x_1\leq a_i + \delta \text{ for all } i \in L, \; x_i = a_i \text{ for all } i \in \set{1, \dots n} \setminus L}\]\footnote{W weil Würfel}
    A face \(W'\) of \(W\) is a subset of the form
    \[W' = \set{x \in W : x_i = a_i \text{ for all } i \in L_0, x_i = a_i + \delta \text{ for all } i \in L_1}\]
    for some subsets \(L_0, L_1 \subseteq L\)

    Let \(1 \leq p \leq n\) we define two subsets of a cube \(W = ^(a, \delta, L)\).
    \[K_p(W) = \set{x \in W : x_i < a_i + \delta/2 \text{ for at least} p \text{values of} i \text{ in } L}\]
    We call these \enquote{\(p\) small coordinates}
    \[G_p(W) = \set{x \in W : x_i > a_i + \delta/2 \text{ for at least } p \text{ coordinates } i \text{ in } L}\]
    these are \enquote{\(p\) big coordinates}.

    For \(p > \dim(W), K_p(W) = G_p(W)= \emptyset \) If \(p+ q \geq \dim(W)\), then \(K_p(W) \cap G_q(W) = \emptyset\).

    He draws pictures.

    \begin{lem}{1.14}{1.14}
        It is Lemma 1.14 in Lücks Script.

        Let \((Y,A)\) be a space pair, \(W \subseteq \ir^n\) a cube, \(f \colon W \to Y\) continuous. Suppose that for some \(p\leq \dim(W)\), \(f^{-1}(A) \cap W' \subseteq K_p(W')\) for all proper\footnote{subcube of the boundary} faces \(W'\) of \(W\).
        
        Then there is a continuous map \(g \colon W \to Y\) homotopic to \(f\) relative \(\partial W\) such that all \(g^{-1}(A)\subseteq K_p(W)\)
    \end{lem}
    \begin{proof}
        Wlog: \(W = I^n = W(0, 1, \set{1,\dots ,n})\)

        Let \(I_2^n\) be the subcube \([0, 1/2]^n\). He draws a picture. \(x_4 = (1/4, \dots 1/4) \in I_2^n\). We define a continuous map \(h \colon I^n \to I^n\) by radical projection away from \(x_4\). Picture. Let \(r(y)\) be the ray from \(x\) to \(y\). We map all of \(r(y) \cap I^n\setminus I^n_2\) to the intersection point of \(r(y)\) and \(\partial I^n\) and the rest linearly extends as far as required.\footnote{I hope this description is clear, hard without the picture.}

        Obviously\footnote{Meaning he's too lazy to come up with formulas for the map.} \(h\) is homotopic relative boundary \(\partial I^n\) to the identity.

        We set \(g \colon f \circ h\colon I^n \to Y\), which is then homotopic relative \(\partial(I^n)\footnote{I am very inconsistent in remembering these parantheses with the boundary operator. Just imagine it always being as here.}\) to \(f\).

        It remains to show that \(g^{-1}(A) \subseteq K_p(W)\). Consider \(z \in I^n\) with \(g(z) \in A\).
        \begin{description}
            \item[Case 1] for all \(i = 1, \dots, n, z_1 < 1/2\), i.e. \(z \in I_2^n\), then \(z \in K_n(I^n) \subseteq K_p(I^n)\).
            \item[Case 2] There is an \(i \in \set{1, \dots,n}\), s.t. \(z \geq 1/2\). Then \(h(z) \in \partial(I^n)\). Let \(W'\) be some proper face of \(W\), with \(h(z) \in W'\). Since \(f(h(z)) = g(z)\in A\), by hypothesis, \(h(z) \in K_p(W')\), so \(h(z) < 1/2\) for at least \(p\) coordinates. By expansion\footnote{no idea if this is the word he wrote} property of \(h\), also \(p\) coordinates of \(z\) are small coordinates. 
        \end{description}
    \end{proof}

    \begin{proposition}\footnote{11.5 in Lücks notes}
        Let \(Y_1, Y_2\) be open subsets of \(Y, Y_0\coloneq Y_1 \cap Y_2\). Suppose that \((Y_1, Y_0)\) is \(p\)-connected, \((Y_2, Y_0)\) is \(q\)-connected. Let \(f \colon I^n \to Y\) be continuous. Let \(\cW = \set{W}\) be a subdivision of \(I^n\) into subcubes of the same side length s.t. for all \(W \in \cW\) \(f(W) \subseteq Y_1\) or \(f(W) \subseteq Y_2\). Then there is a homotopy \(h \colon I^n \times I \to Y\) with \(h_0 = f\) such that for all \(W \in \cW\):
        \begin{enumerate}
            \item If \(f(W) \subseteq Y_j, j \in {0,1,2}\), then \(h_t(W) \subseteq Y_j\) for all \(t \in [0,1]\)
            \item If \(f(W) \subseteq Y_0\), then \(h_t\rvert_{W} = f\rvert_W\), i.e. \(h\) is constant on \(W\).
            \item If \(f(W) \subseteq Y_1\), then \(h_1^{-1}(Y_1 \setminus Y_0)\subseteq K_{p+1}(W)\).
            \item If \(f(W) \subseteq Y_2\), then \(h_1^{-1}(Y_2\setminus Y_0) \subseteq G_{q+1}(W)\).
        \end{enumerate}
    \end{proposition}
    \begin{proof}
        We let \(C^k \subseteq I^n\) be the union of all cubes in \(\cW\) of dimension at most \(k\). We construct homotopies \(h[k]\colon C_k\times I \to Y\), such that for all \(W \in \cW, W \subseteq C_k\) conditions 1. to 4. hold, and \(h[k]\) is constant on \(C_{k-1}\times I\). Then the final \(h[n]\) does the job.

        \textbf{Note.} If \(W \in \cW\) and \(f(W)\subseteq Y_0\) and 2. holds, then also 3. and 4. hold.

        \[h^{-1}(Y_1\setminus Y_0) = h_1^{-1}(Y_2\setminus Y_0) = \emptyset\]
        If \(W \in \cW\), is such that \(f(W) \subseteq Y_1\) and \(f(W) \subseteq Y_2\), then \(f(W)\subseteq Y_1\cap Y_2 = Y_0\). So each \(W \in \cW\) is in excactly one of the following cases
        \begin{itemize}
            \item \(f(W) \subseteq Y_0\)
            \item \(f(W) \subseteq Y_1\) and \(f(W) \not\subseteq Y_1\)
            \item \(f(W) \subseteq Y_2\) and \(f(W)\not \subseteq Y_2\)
        \end{itemize}
        Inductive construction \(k = 0\), i.e. vertexes of the cubes \(w \in \cW\). If \(w\in Y_2\), take \(h[0]_t = \const_{w_0}\).

        Suppose \(f(W_0)\in Y_1\), but \(f(w_0) \not \in Y_2\). Since \(Y_1,Y_0\) is \(0\)-connected, there is a path \(\pi\colon I \to Y_1\) from \(w_0\) to a point in \(Y_0\). We take \(h[0]\) as the path on \(w_0\). Analoguos if \(f(W_0)\in Y_2\setminus Y_1\).

        \textbf{Inductive Step} Let \(W \in \cW\) be a cube of exact dimension \(k\). Then \(\partial W = W \cap C_{k-1}\). Since \((W, \partial W)\) has the HEP, we can extend the previous homotopy \(h[k-1]\rvert_{\partial W}\) to some homotopy on \(W\) relative to \(f\rvert_W\). Let this be \(h'[k]\colon C_k\times I \to Y\): this satisfies conditions 1. and 2. but not yet 3. and 4.

        We produce another homotopy \(h[k]''\) and set \(h[k] = h[k]' * h[k]''\).

        Consider a cube \(W \in \cW\) of dimension \(k\).

        If \(f(W) \subseteq Y_0\) set \(h[k]''\) as the constant homotopy on \(W\).

        If \(h[k]_1'(W) \subseteq Y_1\), but \(h[k]_1'(W) \not \subseteq Y_2\) there is a homotopy relative \(\partial W\) from \(h[k]_1'\) to a map \(f_1(W)\subseteq Y_0\).

        If \(k = \dim(W) > p\) the we use the lemma \ref{lem:1.14} for \(f = h[k]_1'\rvert_W\) and the resulting homotopy is \(h[k]''\rvert_W\).

        If \(h[k]_1'(W)\subseteq Y_2\) but \(h[k]_1'(W) \not \subseteq Y_1\), use the complement case of the lemma\footnote{rest of the proof next lecture. I am very sure some words won't make sense, as they were unreadable on the board.}
    \end{proof}

\newLecture{15.10.2025}

Now for the actual proof of Blakiers Massey
Let \(F(Y_1,Y, Y_2) = Y_1 \times_Y Y^{[0,1]} \times_Y Y_2\).

Let \(F(Y_1, Y_1, Y_0)\) be the subspace of those \(w \in F(Y_1, Y, Y_2)\) such that \(w([0,1])\subseteq Y_1\).

\begin{proposition}
    Assume that \((Y_1, Y_0)\) is \(p\)-connected, \((Y_2, Y_0)\) is \(q\)-connected. Then the pair \((F(Y_1, Y,Y_2), F(Y_1, Y_1, Y_0))\) is \((p+q-1)\)-connected.
\end{proposition}

\begin{proof}
    We consider a map of pairs
    \[\phi\colon (I^n \partial I^n) \to (F(Y_1, Y, Y_2), F(Y_1, Y_1, Y_0))\]
    for \(1 \leq n \leq p+q+1\). We want to homotop \(\phi\) relative \(\partial I^n\) to a map with image in \(F(Y_1, Y_1, Y_0)\). We use the adjoint
    \[\map(X\times [0,1], Z) \cong \map(X, Z^{[0,1]})\]
    We let \(\Phi\colon I^n \times I \to Y\) be the adjoint of \(\phi\), this is \emph{admissable}, in the sense that
    \begin{enumerate}
        \item \(\Phi(x, 0) \in Y_1\) for all \(x \in I^n\)
        \item \(\Phi(x,1) \in Y_2\) for all \(x \in I^n\)
        \item \(\Phi(x,t) \in Y_1\) for all \(x \in \partial I^n\), \(t \in [0,1]\)
    \end{enumerate}
    We want a homotopy of \(\Phi\) through admissable maps to a map \(\Phi' \colon I^n \times I \to Y\) such that \(\img(\Phi')\subseteq Y_1\).

    Apply Proposition 11.5 to \(\Phi\) (with \(n+1\) instead of \(n\)). This gives a homotopy through admissable maps to \(\Psi = g\).
    Let \(h\colon I^n \times I \times I \to Y\) be a homotopy witnessing this. \(h_0 = \Phi, h_1 F \Psi\).

    Consider the projection \(\pr\colon I^n \times I\to I^n\) away from the last coordinate.

    \textbf{Claim.} The image under \(\pr\) of \(\Psi^{-1}(Y\setminus Y_1)\) and \(\Psi^{-1}(Y\setminus Y_2)\) are disjoint.

    Suppose there is \( y \in I^n\) in the intersection of the preimages, so \(z_1 \in \Psi^{-1}(Y\setminus Y_1), z_2 \in \Psi^{-1}(Y\setminus Y_2)\) s.t \(\pr(x_1) = y = \pr(z_2)\). Let \(W = I^{n+1}\) be a subcube of the subdivision of Proposition 11.5 such that \(z_1 \in W\). Since \(z_1 \in \Psi^{-1}(Y\setminus Y_1)\), \(z_1 \in K_{p+1}(W)\), so \(y = \pr(z_1) \in K_p(\img(W))\). Analogous \(y = \pr(z_2) \in G_q(\img(W))\). Since \(p+q > n\), this is a contradiction and the claim is proven.

    \textbf{Claim.} The intersection of \(\pr(\Psi^{-1}(Y\setminus Y_1))\) with \(\partial I^n\) is empty since \(\Psi\) is admissable, and thus \(\Psi(\partial I^n)\subseteq Y_1\). So \(\pr(\Psi^{-1}(Y\setminus Y_1))\) and \(\pr(\Psi^{-1}(Y\setminus Y_2))\cup (\partial I^n)\) are two disjoint closed subsets of the compact space \(I^n\).

    So there is a continuous separating function \(\tau\colon I^n \to [0,1]\), s,t. \(\tau\equiv 0\) is \(\pr(\Psi^{-1}(Y\setminus Y_1))\) and \(\tau\equiv 1\) is \(\partial I^n \cup \pr(Y\setminus Y_2)\)
    
    We define another homotopy starting with \(\Psi\) by
    \[h\colon (I^n\times I)\times I \to Y\]
    by
    \[((x,t),s) \mapsto \Psi(x,(1-s)t + s\cdot t \tau(x))\]
    This is
    \begin{itemize}
        \item Homotopy through admissable maps
        \item \(h(x,t,0) = \Psi(x,t)\)
    \end{itemize}
    \textbf{Claim.} \(h(\_,\_, 1)\) has image in \(Y_1\).

    \[h(x,t,1) = \Psi(x, t\cdot\tau(x))\]
    \begin{itemize}
        \item if \(x \in \Psi^{-1}(Y\setminus Y_1)\), then \(\tau(x) = 0\), \(h(x,t,1) = \Psi(x,0) \in Y_1\).
        \item if \(x \in \Psi^{-1}(Y_1)\), then by admissability \(\Psi(x,t\cdot\tau(x))\in Y\).
    \end{itemize}
\end{proof}

For the actual proof \((Y_1, Y_0)\) \(p\)-connected \((Y_2, Y_0)\) \(q\)-connected

some diagram about something being Serre fibration

We compare two Serre fibrations
\[\begin{tikzcd}
    F(\set{y_0}, Y_1, Y_0) & F(\set{y_0}, Y, Y_2) \\
    F(Y_1, Y_1, Y_0) \ar[r] & F(Y_1, Y, Y_2) \\
    Y_1  & Y_1 \\
\end{tikzcd}\]
partial 5-lemma shows that also the pair
\[(F(\set{y_0}, Y, Y_2), F(\set{y_0}, Y_1, Y_0))\]
is \((p+q-1)\)-connected.

The following square commutes for all \(n \geq 1\)
\[\begin{tikzcd}
    \pi_{n-1}(F(\set{y_0}, Y_1, Y_0), \const_{y_0})\ar[d, "\cong"] \ar[r] & \pi_{n-1}(F(\set{y_0}, Y, Y_1), x) \ar[d, "\cong"]\\
    \pi_n(Y_1, Y_0, y_0) \ar[r, "\id_*"] & \pi_n(Y, Y_1, y_0) \\
\end{tikzcd}\]
\end{proof}

\section{Feudenthal suspension theorem}
Section 6.4 in tom Deicks book

\begin{defi}{}{}
    Let \(X\) be a based space. The \emph{unreduced suspension} is \(X^\diamond = X\times [-1,1]/\sim\) where \(\sim\) identifies all points with second variable \(-1\) to \(S\) and all points with second variable \(1\) to \(N\). We use \(S\) as the base point of \(X^\diamond\).\footnote{south and north pole}
\end{defi}

\textbf{Note.} If \(X\) is well based, i.e. \(\set{x_0} \hookrightarrow X\) has the HEP, then the quotient map
\[X^\diamond \sigma X = \frac{X\times [0,1]}{X\times \set{0,1}\cup \set{x_0}\times [0,1]}\]
is a homotopy equivalence.

The suspension homomorphism \(S\colon \pi_k(X,x_0) \to \pi_{k+1}(X^\diamond, S)\) for \(k \geq 1\) is \(S[f\colon S^k \to X] \coloneq [f^\diamond \colon S^{k+1}= (S^k)^\diamond \to X^\diamond]\)

\begin{thm}{Freudenthal suspension theorem}{}
    Let \(X\) be an \((n-1)\)-connected space, \(n \geq 1\). Then the suspension homomorphism is \begin{itemize}
        \item bijective for \(i\leq 2n-2\)
        \item surjective for \(i = 2n-1\)
    \end{itemize}
\end{thm}

\begin{proof}
    We cover \(X^\diamond\) by \(Y_1 = X^\diamond \setminus \set{N}\), \(Y_2 = X^\diamond \setminus \set S\). We use repeatedly that \(Y_1\) and \(Y_2\) are contractible. Then \(X\to Y_1\cap Y_2 = Y_0\) is a homotopy equivalence. We claim without proof that the following diagram commutes:

    I couldn't keep up.

    So we may show that \(\pi_{i+1} (Y_2, Y_0, x_0)\to \pi_{i+1}(Y, Y_1, y_0)\) is bijective for \(i \leq 2n-2\) and surjective for \(i \leq 2n-1\).

    Because \(X\) is \((n-1)\)-connected
    \[\pi_{k+1}(Y_1, Y_0, x_0) \xrightarrow[\partial]{\cong} \pi_k(Y_0, x_0)\]
    So \(\pi_i(Y_1, Y_0, x_0) = 0\) for \(i \leq n\), i.e. \((Y_1, Y_0)\) is \(n\)-connected. Also \((Y_2, Y_0)\) is \(n\)-connected.

    BM gives \(\pi_k(Y_2, Y_0, y) \to \pi_k(X^\diamond, Y_1, y)\) is bjective for \(k < 2n-1\) and surjective for \(k = 2n\).

    Setting \(i = k-1\) ends the proof.
\end{proof}

Take \(X = S^n\)
\begin{corollary}
    The suspension homomorphism
    \[S\colon \pi_i(S^n, *) \to \pi_{i+1}(S^{n+1}, *)\]
    is bijective for \(i \leq 2n-2\) and surjective for \(i = 2n-1\).
\end{corollary}

\begin{corollary}
    For all \(n \geq 1\), \(\pi_n(S^n, *)\cong \IZ\), generated by the class of \(\id_{S^n}\). Moreover \(\deg\colon \pi_n(S^n, *)\to \IZ\) is an isomorphism.
\end{corollary}

\begin{proof}
    Induction on \(n\). For \(n= 1\) we have this by covering theory.

    \(n\geq 1\) By Freudenthal, the suspension homomorphism
    \[S\colon \pi_n(S^n, *), \to \pi_{n+1}(S^{n+1}, *)\]
    is surjective. The composite
    \[\pi_n(S^n, *) \xrightarrow{S} \pi_{n+1}(S^{n+1}, *) \xrightarrow{\deg} \IZ\]
    is bijective by induction. So \(S\) is also injective and \(\deg\) in one dimension higher is also an isomorphism.
\end{proof}

\textbf{Recall.} \(\pi_3(S^2, *) \cong \IZ\set{\eta}\), where \(\eta \colon S^3 \to S^2\) is the Hopf map.

\begin{proposition}
    For \(n \geq 3\), \(\pi_{n+1}(S^n, *) = \IZ/2\set{S^{n-2}(\eta)}\) cycles of order two.
\end{proposition}

\begin{proof}
    Only partial. For \(n \geq 2\), \(S\colon \pi_{n+1}(S^n, q) \mapsto \pi_{n+2}(S^{n+1}, *)\) is surjective.
    \[\iz\set{\eta} = \pi_3(S^2, *) \twoheadrightarrow \pi_4(S^3, *) \xrightarrow[\cong]{S} \pi_5(S^4, *)\dots\]
    \textbf{Claim.} \(S(2\eta) = 0\) in \(\pi_4(S^3, *)\). Which gives \(\pi_{n+1}(S^n, *)\) is either trivial or order 2.

    Consider the commutative square

    I did not manage to copy. Something complex conjugation

    \(\implies [\eta] = [\eta \circ \text{complex conjugation}] = [d \circ \eta]\) in \(\pi_3(S^2, * )\cong \iz\).

    If \(f\colon S^n \to S^n\) has degree \(k\), then precomposition of \(\pi_n(X,x)\to \pi_n(X,x)\) is multiplication by \(k\).

    Let \(c\colon S^1\to S^1\) be any map of degree \(-1\). Then \(c\wedge S^2\), \(S^1 \wedge d\): bothe have degree \(-1\). So \(c \wedge S^2 \sim S^1 \wedge d\)

    two diagrams I did not copy.

    We get \(S(\eta) = S(d \circ \eta)=S(\eta \circ (c\wedge S^1)) = S(-\eta) = - S(\eta)\) and hence \(2 \cdot S(\eta) = 0\).
\end{proof}

\newLecture{20.10.2025}

We make a bit of a preview\footnote{He just spammed random stuff, I don't think I copied enough for it to make sense.} for stable homotopy theory. Following Lücks notes today rather closely

\begin{defi}{}{Stable Homotopy Groups}
    Let \(X\) be a based space. The \(n\)-th stable homotopy group of \(X\) is the colimit \(\pi_n^S(X,*)\)
\[\pi_n(X,*)\xrightarrow{S} \pi_{n+1}(S^1 \wedge X, *) \xrightarrow{S} \pi_{n+2}(S^2 \wedge X, *)\dots\]
\end{defi}

Since \(S^k \wedge X\) is \((k-1)\)-connected, the suspension stabilises (i.e. \(S\) is isomorphism for \(\pi_{n+k}(S^k \wedge X, *)\) onwards).

\(\pi_n^S()\) is a functor from based spaces to abelian groups. and it is homotopy invariant. It comes with a natural transformation \(\pi_n(X,*) \to \pi_n^S(X,*)\).

\textbf{Preview:} \(\set{\pi_n^S}_{n \in \iz}\) form a generalized homology theory on based spaces.

If \((X,A)\) is a pair of based space, \(x \in A \subseteq X\), we define
\[\pi_n^S(X,A,*) \coloneq \pi_n^S(X\cup_A CA, *)\]
The map collapsing \(X\) is
\[X\cup_A CA \to \frac{X\cup_A CA}{X} \cong \frac{CA}{A} \cong A^\diamond\]
induces an connecting homomorphism
\[\partial\colon \pi_n^S(X,A,x) = \pi_n^S(X \cup_A CA, *) \xrightarrow{p_*} \pi_n^S(A^\diamond) \to \pi_n^S(A \wedge S^1) \xrightarrow{\cong} \pi_{n-1}^S(A)\]

The following sequence will then be exact:
\[\dots \pi_n^S(A,x)\xrightarrow{\incl_*} \pi_n^S(X,x,) \xrightarrow{\incl_*}\pi_n^S(X,A,*) \xrightarrow{\partial} \pi_{n-1}^S(A,*)\to \dots\]

Stable stems are the special case
\[\pi_n^S(S^0) = \colim(\pi_n(S^0, *) \to \pi_{n+1}(S^1, *)\to \dots)\]

\begin{center}
    \begin{tabular}{l|c|c|c|c|c|c|c|c|c}
        n & 0&1&2&3&4&5&6&7&8 \\\hline
        \(\pi_n^S = \pi_n^S(S^0)\) & \(\iz\) & \(\iz/2\) & \(\iz/2\) & \(\iz/24\) & \(0\)& \(0\) & \(\iz/2\) & \(\iz/240\) & \((\iz/2)^2\) \\\hline
        Generator & \(\id\) & \(\eta\) & \(\eta^2\) & \(\nu\), \(\eta^3 = 12 \nu\) & - & - & \(\nu^2\) & \(\sigma\) & \(\eta\sigma, \epsilon\)
    \end{tabular}
\end{center}

There is a graded-commutative ring structure on \(\pi_*^S = \set{\pi_n^S}_{n \in \iz}\). \(\pi_n^S \times \pi_m^S \to \pi_{n+m}^S\)

\[[f\colon S^{n+k}\to S^k\times S^k] \times [g \colon S^{m+l}\to S^l] = [SW{n+m+k+l} \to S^{n+k} \wedge S^{m+l} \xrightarrow{f\wedge g} S^{k+l}]\]

I missed a bit more

Nishidas theorem says: every positive dimensional element of \(\pi_*^3\) is nilpotent.

From Serre spectral sequence we will see: For \(m > n\geq 1\) \(\pi_m(S^n, *)\) is finite except \(\pi_{4k - 1}(S^{2k}, *) \cong \iz \oplus \) some finite group.

This was exercise 5.2 of last summer term: \emph{Hopf invariant} \(h\colon \pi_{2k-1}(S^k, *) \to \iz\) for \([f\colon S^{2k-1}\to S^k]\)
\[H^*(S^k\cup_f D^{2k}, \iz) = \begin{cases}
    \iz & i = 0,k,2k \\
    0 & \text{else}
\end{cases}\]
we have 
\(H^k(Cf, \iz) = \iz\set a\) \(H^{2k}(Cf, \iz) \cong \iz\set b\) and \(a \cup a = h(f) \cdot b\).

In general we can look at

\(t_k\colon S^{2k-1}\to S^k\vee S^k\) the attaching map of the \(2k\)-cell in the product minimal CW-structure of \(S^k \times S^k\). And we get \(h(t_k) = 2\).

The Hopf invariant 1 theorem tells us when we can realize Hopf invariant 1 and here we see that we always find Hopf invariant 2.

\section{Hurewicz-theorem}

This is a \enquote{trivial}\footnote{meaning 1.5 lectures of intermediate steps} Corollary of the Blakiers-Massey theorem.

\begin{defi}{}{Hurewicz-Homomorphism}
    Let \(X\) be a based space. Choose an orientation of \(S^n\), \(n \geq 1\), i.e. \([S^n] \in H_n(S^n, \iz)\). The Hurewicz homomorphism
    \[h\colon \pi_n(X,x) \to H_n(X;\iz)\]
    is defined by
    \[h[f\colon S^n\to X] = H_n(f, \iz)[S^n].\]
    This is a group homomorphism.
\end{defi}

\textbf{Group Homomorphism:} Let \(\Delta\colon S^n \to S^n \vee S^n\) be a pinch map.
Group structure on \(\pi_n(X,x)\) is given by \([f]+[g] = [S^n\xrightarrow{\Delta}S^n\vee S^n \xrightarrow{f+g} X]\). Then
\iffalse
\[
\begin{tikzcd}
    \pi_n(X, x) \arrow[d, "\cong"'] \arrow[rd, "h"] & \\
    \pi_n(X, y) \arrow[r, "h"] & H_n(X, \mathbb{Z})
\end{tikzcd}
\]
\fi

A diagram is missing here. 

\[([f]+ [g])_* = [f]_* + [g]_* \colon H_n(S^n) \to H_n(X)\]
So Hurewicz is a homomorphism.

\begin{lem}{}{}
    Let \(w\colon [0,1]\to X\)be a path from \(x = w(0)\) to \(y = w(1)\). Then the following commutes:
    \[\begin{tikzcd}
        \pi_n(X,x) \ar[d, "\cong"] \ar[rd, "h"] & \\
        \pi_n(X,y) \ar[r, "h"] & H_n(X, \iz) \\ 
    \end{tikzcd}\]
\end{lem}

\begin{proof}
    \(f\) and \(f *w\) are freely homotopic \(h_n(\_, \iz)\) is homotopy invariant
\end{proof}

For \(n = 1\) we have Poincaré: \(X\) based path conneted, \(h\colon \pi_1(X,x)\to H_1(X,\IZ)\) is surjective with kernel the commutator subgroup or equivalently
\[\pi_1(X,x)_{ab} \xrightarrow{\cong} H_1(X,\iz)\]

\begin{thm}{Hurewicz}{Hurewicz}
    Let \(X\) be an \((n-1)\)-connected space, \(n\geq 2\). Then the Hurewicz homomorphism
    \[h\colon\pi_n(X,x) \to H_n(X,\iz)\]
    is an isomorphism.
\end{thm}

We will need to work a bit for the proof.

\begin{proposition}[11.9 in Lücks notes]
    Let \(m,n \geq 0\), let
    \[\begin{tikzcd}
        A \ar[r, "f"] \ar[d, "i"] & B \ar[d, "\bar i"]\\
        X \ar[r, "\bar f"] & Y \\
    \end{tikzcd}\]
    be a pushout square of spaces. Suppose \(i\colon A \to X\) is a \emph{cofibration}, i.e. a closed embedding with the HEP.
    \begin{enumerate}
        \item If \(f\) is \(n\)-connected, then so is \(\bar f\)
        \item If \(f\) is \(n\)-connected, and \(i\) is \(m\)-connected, then for all \(a \in A\), \(\pi_i(f, \bar f) \colon \pi_i(X,A, a) \to \pi_i(Y, B, f(a))\) is bijective for \(1 \leq i < m+n\) and surjective for \(i = m+n\).
    \end{enumerate}
\end{proposition}

\begin{proof}
    \enquote{Basically reduction to Blakiers-Massey}

    We can replace \(X, B\) and \(Y\) up to homotopy equivalence by appropriate mpping cylinders:
    \[\mathrm{cyl}(f) = A\times [0,1]\cup_{A\times 1 , f} B\]
    We get a homotopy commutative diagram
    \[\begin{tikzcd}
        A \ar[r]\ar[d] & \mathrm{Cyl}(f) \ar[d]  \ar[rd, "\sim"]& \\
        \mathrm{Cyl}(i) \ar[r]\ar[rd, "\sim"] & \mathrm{cyl}(i) \cup_A \mathrm{cyl}(f)\ar[rd, "\sim"] & B\ar[d, "\bar i"] \\
        & X \ar[r, "\bar f"] & Y \\
    \end{tikzcd}\]
    This diagram does not commute. The homotopy equivalences shown are not trivial.

    We apply Blakiers-Massey then to the following open subspace of \(W = \mathrm{cyl}(i)\cup_A \cyl(f)\).

    \(W_2 = [0,1/2) \times A\cup_{A\times 0} \cyl(f)\), \(W_1 = \cyl(i) \cup_{A\times 0} [0, 1/2)\). \(W_0 = W_1\cap W_2 \sim A\)

    Now having \(f\) is \(n\)-connected gives \((W_2, W_0)\) is \(n-connected\). \(i\) being \(m\)-connected gives \((W_1, W_0)\) is \(m\)-connected.

    Now BM gives \(\pi(W_1, W_0, a)\to \pi_i(W, W_2, a)\) is bijective for \(1 \leq i \leq m+n-1\) and surjective for \(i = m+n\).

    Using the homotopy equivalences, we translate this back.
\end{proof}

\begin{proposition}[11.1 in Lücks]
    Let \(m,n \geq 0\), let \(c\colon A \to X \) be a cofibration. Suppose that \(i\) is \(m\)-connected and \(A\) is \(n\)-connected. Then 
    \[\pi_k(\pr)\colon \pi_k(X,A,a) \to \pi_k(X/A, *)\]
    for all \(a \in A\) is bijective for \(1 \leq k \leq m+n\) and surjective for \(k = m+n+1\)
\end{proposition}

\begin{proof}
    We consider the pushout
    \[\begin{tikzcd}
        A \ar[d, "i"]\ar[r, "j", "(n+1)\text{-con}"'] & C(A) = A\times [0,1]/A\times 1 \ar[d] \\
        X \ar[r] & X \cup_A CA \ar[r, "\sim"] & X/A\\
    \end{tikzcd}\]

    \begin{remark}
        \(X\) \(k\)-connected \(\Lra\) \(\set{*}\to X\) \(k\)-connected \(\Lra\) \(X\to \set{*}\) is \((n+1)\)-connected.
    \end{remark}
\end{proof}

\begin{proposition}[11.12 for Lück]
    Also an exercise (1.2) Let \(X,Y\) be well-pointed spaces. Let \(m,n \geq 1\). Let \(X\) be \(m\)-connected, \(Y\) \(n\)-connected. Then
    \begin{enumerate}
        \item The inclusion \(X\vee Y\to X\times Y\) induces isomorphisms
        \[\pi_k(X\vee Y, *) \to \pi_k(X\times Y, *)\]
        for all \(0 \leq k \leq m+n\).
        \item \(\pi_k(X\times Y, X\vee D, *)\) and \(\pi_k(X\wedge @, *)\) are trivial for all \(0 \leq k\leq m+n+1\).
        \item The map \((pr_*^X, pr_*^Y)\colon \pi_k(X\vee Y, x) \to \pi_k(X, x_0)\times \pi_k(Y, y_0)\) is an isomorphism for all \(1 \leq k \leq m+n\).
    \end{enumerate}
\end{proposition}

\newLecture{22.10.2025}

\begin{proposition}
    Let \(n \geq 2\). Let \(\set{X_i}_{i\in I}\) be a family of well-pointed, \((n-1)\)-connected spaces. Then the canonical map
    \[\bigoplus_{i\in I} \pi_n(X_i, *) \to \pi_n(\bigvee_{i\in I} X_i, *)\]
    is an isomorphism.
\end{proposition}

\begin{proof}
    \begin{description}
        \item[Step 1] If \(I\) is finite, we do Induction on \(\abs I\). Nothing to show if \(I = \emptyset\), \(\abs I = 1\). Let \(\abs I \geq 2\). Wlog \(I = \set{1,2, \dots k}\), \(k\geq 2\).
        
        By Proposition 1.19, \(\pi_n(\bigvee_{i = 1, \dots, k-1}X_i, *) \oplus \pi_n(X_k, *) \xrightarrow{\cong} \pi_n(\bigvee_{i = 1,\dots, k} X_i, *)\) and the first part is isomorphic to \(\bigoplus_{i = 1,\dots, k} \pi_n(X_i)\) by induction.

        \item[Case 2] \(I\) is infinite: We first show injectivity. The projection
        \[p_j\colon \bigvee_{i\in I} X_i \to X_j\]
        induces an homomorphism
        \[(p_j)_*\colon \pi_n(\bigvee_I X_i, *) \to \pi_n(X_j, *)\]
        For varying \(j\in I\), these assemble into a homomborphism
        \[\bigoplus_{i\in I}\pi_n(X_i,*)\xrightarrow{can} \pi_n(\bigvee_{i\in I} X_i, *) \xrightarrow{((p_j)*)} \prod_{j\in I} \pi_n(X_j, *)\]
        and the composition is inclusion of sum into product, hence injective. So also the first map is injective.

        For surjectivity let \(f\colon S^n \to \bigvee_{i\in I}X_i\) be a continuous map that represents a class in \(\pi_n(\bigvee_{i\in I} X_i)\). Because \(S^n\) is compact, there is a finite subset \(J\subseteq I\) s.t. \(\Img(f)\subseteq \bigvee_{i \in J} X_i\)\footnote{he explains, why this is not easy to see. But it is point-set topology, so we won't do it.} This implies \([f] \subseteq \pi_n(\bigvee_{j \in J} X_j,*) \to \pi_n(\bigvee_{i\in I} X_i, *)\).
        \[\begin{tikzcd}
            \pi_n(\bigvee_{j\in J}X_j, *) \ar[r] & \pi_n(\bigvee_{i \in I}X_i, *) \\
            \bigoplus_{j\in J} \pi_n(X_j,*) \ar[u, "\text{Case 1}"] \ar[r] & \bigoplus_{i\in I} \pi_n(X_i, *) \ar[u, "can"] \\
        \end{tikzcd}\]
        so \([f]\) is in the image of the canonical map.
    \end{description}
\end{proof}

\begin{thm}{Hurewicz}{}
    Let \(n\geq 2\). Let \(X\) be \((n-1)\)-connected base space. Then the Hurewicz homomorphism \(h\colon \pi_n(X,x)\to H_n(X,\iz)\), \(f\mapsto H_n(f, \iz)[S^n]\) is an isomorphism.
\end{thm}

\begin{proof}
    \begin{itemize}
        \item By CW-approximation (see later in this class) there is a CW-complex \(Y\) with one \(0\)-cell and no cells in dimensions \(1\leq i \leq n-1\) and a weak homotopy equivalence \(Y\xrightarrow{\simeq} X\).
        \item Every weak homotopy equivalence induces isomorphisms of \(H_n(\_, A)\) for all \(n \geq 0\), for all abelian groups \(A\). This will either be an exercise or proven later on.
    \end{itemize}
    We get a commutative diagram
    \[\begin{tikzcd}
        \pi_n(Y, y) \ar[r, "f_*", "\cong"'] \ar[d, "h"] & \pi_n(X,f(*)) \ar[d, "h"] \\
        H_n(Y, \iz) \ar[r, "f_*", "\cong"'] & H_n(X,\iz) \\
    \end{tikzcd}\]
    So wlog we can assume that \(X\) admits a CW-structure with a single \(0\)-cell and no cells of dimensions \(1, \dots, n-1\). The inclusion of the \((n+1)\)-skeleton \(X_{n+1} \to X\) induces isomorphisms
    \[\pi_n(X_{n+1}, *) \to \pi_n(X, *)\]
    by cellular approximation. Also
    \[H_n(X_{n+1}, \iz) \to H_n(X,\iz)\]
    is an isomorphism for example by cellular homology.

    It suffices to show the Hurewicz theorem for the \(n+1\)-skeleton, i.e. \(X\) is a CW-complex with a single \(0\)-cell, \(I\) many \(n\)-cells, \(J\) many \((n+1)\)-cells and no cellls in any other dimension.
    \[X= (\set x \cup_{I\times \partial D^n} D^n) \cup_{J\times \partial D^{n+1}} (J\times D^{n+1}) \cong (\bigvee_{i\in I} S^n) \cup_{J\times D^{n+1}} J\times D^{n+1}\]
    We can assume that all attaching maps \(\alpha\colon \partial D^{n+1}\to \bigvee_{i \in I} S^n\) are based.

    We see this since \(\bigvee_{i\in I} S^n\) is path connected, \(\alpha\) is homotopic by HEP to a based map \(\alpha'\). Since homotopic attaching maps yield homotopy equivalent glueings.

    So \(X\) can be written as a pushout
    \[\begin{tikzcd}
        \bigvee_{j\in J} S^n \ar[r, "f"]\ar[d, hook] & \bigvee_{i \in I} S^n  \ar[d]\ar[r, phantom, "="] & X_n \\
        \bigvee_{j \in J} D^{n+1} \ar[r] & X \ar[r, phantom, "="] & X_{n+1}\\
    \end{tikzcd}\]
    Since \(\bigvee_I S^n\) and \(\bigvee_J S^n\) are \((n-1)\)-connected, \(f\colon \bigvee_J S^n \to \bigvee_I S^n\) is \((n-1)\)-connected. Also \(\bigvee_J S^n \to \bigvee_I D^{n+1}\) is \(n\)-connected, as \(\bigvee_I D^{n+1}\) is contractible.

    By Proposition ?? \(\pi_k(\bigvee_J D^{n+1}, \bigvee_J S^n) \to \pi_k(X, X_n) = \pi_k(X, \bigvee_I S^n)\) is isomorphic for \(1 \leq k \leq 2n-2\) and surjective for \(k = 2n-1\). In particular \(\pi_{n+1}(\bigvee_J D^{n+1}, \bigvee_J S^n)\to \pi_{n+1}(X,X_n)\) is surjective. We compare the long exact homotopy sequences of the vertical pairs:
    \[\begin{tikzcd}
        0 \ar[r] & \pi_{n+1}(\bigvee_J D^{n+1}, \bigvee_J S^n)\ar[r, "\partial", "\cong"']\ar[d, "g_*", two heads] & \pi_n(\bigvee_J S^n) \ar[r] \ar[d, "f_*"] & 0 \\
        \phantom 0 \ar[r] & \pi_{n+1}(X, X_n) \ar[r, "\partial"] & \pi_n(X_n) \ar[r, "\text{surjective by cell. approx}"'] & \pi_n(X,*) \\
    \end{tikzcd}\]

    So the upper row is exact:
    \[\begin{tikzcd}
        \pi_n(\bigvee_J S^n) \ar[d, "h"] \ar[r, "f_*"] & \pi_n(\bigvee_I S^n) \ar[d, "h"] \ar[r, "\incl_*"] & \pi_n(X) \ar[r] \ar[d, "h"] & 0 \\
        H_n(\bigvee_J S^n, \iz) \ar[r, "f_*"] & H_n(\bigvee_I S^n, \iz) \ar[r] & H_n(X,\iz) \ar[r] & 0 \\
    \end{tikzcd}\]
    We claim the first 2 \(h\) are isomorphisms We have  a long exact sequence by excision. For all sets \(I\)
    \[\begin{tikzcd}
        \bigoplus_{i\in I} \pi_n(S^n, *) \ar[r, "\cong"] \ar[d, "\bigoplus h"] & \pi_n(\bigvee_{i\in I} S^n, *) \ar[d, "h"] \\
        \bigoplus_{i\in I} H_n(S^n, \iz) \ar[r, "\cong"] & H_n(\bigvee_I S^n, \iz) \\
    \end{tikzcd}\]
    so \(h\colon \pi_n(X,x) \to H_n(X, \iz)\) is an isomorphism by 5-lemma.
\end{proof}

Some applications

\textbf{Recall.} If \(X\) is simply connected, tehn \(H_1(X,\iz)\cong \pi_1(X,x)_{\mathrm{ab}} = 0\). But if \(X\) is path connceed and \(H_1(X,\IZ) = 0\), \(X\) need not be simply connected, because \(\pi_1(X,x)\) could be non-trivial and perfect (= abelianization is trivial) (E.g. \(A_5\)).

\begin{proposition}[12.7(i) for Lück]
    Let \(X\) be simply connected, \(n \geq 1\). then the following are equivalent
    \begin{enumerate}
        \item \(X\) is \(n\)-connected.
        \item \(H_i(X,\iz) = 0\) for all \(2 \leq i \leq n\).
    \end{enumerate}
\end{proposition}
\begin{proof}
    By induction on \(n\). Nothing to show for \(n=1\). The induction step is the Hurewicz theorem \(\pi_n(X,x) \cong H_n(X,\iz)\).
\end{proof}

\begin{proposition}[12.7 (ii) for Lück]
    Let \(X\) be simply connected. Then the following are equivalent:
    \begin{enumerate}
        \item \(X\) is weakly contractible\footnote{All homotopy groups vanish}
        \item \(H_i(X,\iz) = 0\) for all \(i\geq 2\).
    \end{enumerate}
\end{proposition}

\textbf{Warning.} There exist acyclic spaces, i.e. non-contractible CW-complexes, path connected with \(H_i(X,\iz) = 0\) for all \(i \geq 1\).

\begin{remark}
    There is a slightly better version of the Hurewicz theorem: If \(X\) is \(n-1\)-connected, \(n \geq 2\), then \(h\colon \pi_n(X,x) \to H_n(X, \iz)\) is isomorphism and \(h\colon \pi_{n+1}(X,x) \to H_{n+1}(X,\iz)\) is surjective.
\end{remark}

\subsection{Relative Hurewicz theorem}

\begin{defi}{Relative Hurewicz map}{}
    Let \((X,A)\) be a space pair, \(a\in A\). Choose a generator \([D^n, S^{n-1}]\in H_n(D^n, S^{n-1}, \iz)\). The relative Hurewicz homomorphism is
    \[h\colon \pi_n(X,A, a)\to H_n(X,A,\iz)\]
    is defined by
    \[[f\colon (D^n, S^{n-1}, {z})\to (X,A,\set a)]\mapsto h[f] \coloneq H_n(f, \iz)[D^n, S^{n-1}]\]
\end{defi}

The following diagram commutes:
\[\begin{tikzcd}
    \pi_n(X,a) \ar[r]\ar[d, "h"] & \pi_n(X,A, a) \ar[r, "\partial"] \ar[d, "h"] & \pi_{n-1}(A,a)\ar[d, "h"] \\
    H_n(X,\iz) \ar[r] & H_n(X,A,\iz) \ar[r, "\partial"] & H_{n-1}(A,\iz) \\
\end{tikzcd}\]

\begin{thm}{relative Hurewicz, (simply connected case)}{}
    Let \(n\geq 2\). Let \((X,A)\) be a space pair, such that \(X\) and \(A\) are simply connected and \((X,A)\) is \((n-1)\)-connected. Then
    \begin{enumerate}
        \item The Hurewicz homomorphism \(h\colon \pi_n(X,A,a) \to H_n(X,A, \iz)\) is an isomorphism, and
        \item The group \(H_i(X,A,\iz) = 0\) for all \(0 \leq i \leq n-1\).
    \end{enumerate}
\end{thm}

\begin{proof}
    We will deduce this from the absolute version and some other things we already did.

    By replacing \(X\) by the mapping cylinder \(A\times [0,1]\cup_{A\times 1} X\) and replacing \(A\) by \(A\times 0\), we can assume wlog that the inclusion \(i\colon A \to X\) is a cofibration. Since \(A\) is \(1\)-connected, and \((X,A)\) is \((n-1)\)-connected,
    \[pr\colon \pi_k(X,A,a) \to \pi_k(X/A, *)\]
    is bijective for \(1\leq k\leq n\) and surjective for \(k = n+1\).

    \(X/A\) is simply connected by the van Kampen theorem.

    Since \(\pi_k(X,A,*) = 0\) for \(k\leq n-1\) by hypothesis, we get that \(\pi_k(X/A, *) = 0\) for \(k\leq n-1\). So \(X/A\) is \((n-1)\)-connected. By the absolute Hurewicz theorem for \(X/A\), \(H_k(X/A, \iz) = 0\) for \(1 \leq k\leq n-1\) and
    \[\begin{tikzcd}
        \pi_n(X,A,a) \ar[r, "\cong", "pr_*"'] \ar[d] & \pi_n(X/A, *) \ar[d, "\cong", "h"']\\
        H_n(X,A,\iz) \ar[r, "\cong", "\text{excision}"'] & H_n(X/A, \iz) \\
    \end{tikzcd}\]
\end{proof}

\newLecture{27.10.2025}

\begin{proposition}
    Let \(f\colon X \to Y \) be a map of simply connected spaces. Let \(n \geq 1\). Then TFAE:
    \begin{enumerate}
        \item \(f\) is \(n\)-connected
        \item \(f_*\colon H_i(X,\IZ) \to H_i(Y, \iz)\) is bijective for \(2 \leq i \leq n-1\) and surjective for \(i = n\).
    \end{enumerate}
\end{proposition}

\begin{proof}
    By replacing \(Y\) by the mapping cylinder of \(f\colon X\to Y\), we may assume \(f\) is the inclusion \(i\colon A \hookrightarrow X\) of a closed subspace \((X,A)\) is \(n\)-connected is equivalent to \(\pi_k(X,A,a) = 0\) for all \(k \leq n\) which is then by Hurewicz equivalent to \(H_k(X,A,\iz) = 0\) for all \(k \leq n\). But this is equivalent to 2.
\end{proof}

\begin{thm}{Whitehead theorem}{}
    Let \(f\colon X\to Y\) be a continuous map between simply connected CW-complexes. Then the following are equivalent:
    \begin{enumerate}
        \item \(f\) is a homotopy equivalence.
        \item \(f\) is a weak homotopy equivalence.
        \item \(f_*\colon H_i(X,\IZ)\to H_i(Y, \iz)\) is an isomorphism for all \(i\geq 0\).
    \end{enumerate}
\end{thm}

\textbf{Note.} 1 \(\Lra\) 2 without simply connectedness is what we previously called the Whitehead theorem.

\begin{thm}{}{}
    Let \(f\colon X\to Y\) be a continuous map between path connected CW-complexes. Suppose that for some (hence any) \(x \in X\) \(f_*\colon \pi_1(X,x) \to \pi_1(Y, f(x))\) is an isomorphism. Let \(\tilde f\colon \tilde X\to \tilde Y\). be a lift to universal covers.
    \[\begin{tikzcd}
        \tilde X \ar[r, "f"]\ar[d, "q"] & \tilde Y \ar[d, "q"] \\
        X\ar[r, "f"] & Y \\
    \end{tikzcd}\]
    Then TFAE:
    \begin{enumerate}
        \item \(f\) is a homotopy equivalence
        \item \(\tilde f_*\colon H_i(\tilde X, \iz) \to H_i(\tilde Y, \iz)\) is an isomorphism for all \(i \geq 0\).
    \end{enumerate}
\end{thm}

\begin{proof}
    \(p_*\colon \pi_i(\tilde X, \tilde x)\to \pi_i(X,x)\) is an isomorphism for \(i\geq 2\). So  \(f\) is a weak homotopy equivalence iff \(\tilde f\colon \tilde X\to \tilde Y\) is a weak homotopy equivalence. and now this is equivalent too \(H_*(\tilde f, \iz) \) is an isomorphism for all \(*\geq 0\).
\end{proof}

\begin{thm}{12.16 for Lück}{}
    Let \(X\) be a path connected CW-complex, \(n\geq 2\). Then the following are equivalent:
    \begin{enumerate}
        \item \(X\) is homotopy equivalent to \(S^n\)
        \item \(X\) is simply connected and \(H_i(X,\iz)\cong \begin{cases}
            \iz & i = 0,n\\
            0 & \text{else}
        \end{cases}\)
    \end{enumerate}
\end{thm}

\begin{proof}
    \begin{description}
        \item[\(1. \implies 2.\)] 
        \item[\(2. \implies 1.\)] By the Hurewicz Theorem \(h\colon \pi_n(X,x) \to H_n(X,\iz)\cong \iz\) is an isomorphism. Let \(f\colon S^n\to X\) represent a generator of \(\pi_n(X,x)\). Then \(f\) induces an isomorphism of all integral homology groups. Since \(S^n\) and \(X\) are simply connected CW-complexes, \(f\) is a homotopy equivalence.
    \end{description}
\end{proof}

We will not proof a even more general relative Hurewicz theorem:

Let \((X,A)\) be a space pair, \(a \in A\). Recall that \(\pi_1(A,a)\) acts on \(\pi_n(X,A,a)\) for \(n\geq 1\):

With \(f\colon (I^n, \partial I^n, J^{n-1})\to (X,A,a)\) and \(w\colon ([0,1], \set{0,1}) \to (A,a)\) with \(w * f\), which he explains by a picture. Note \(w*f\) is pair homotopic to \(f\) (but \textbf{Not} triple homotopic).

If \([I^n, \partial I^n] \in H_n(I^n, \partial I^n, \iz) \cong \iz\) is a generator, then
\[f_*[I^n, \partial I^n] = (w*f)_*[I^n, \partial I^n]\]
so \(h\colon \pi_n(X,A,a) \to H_n(X,A, \iz)\) satisfies \(h[f] = h[w*f]\).

\begin{defi}{}{}
    Let \((X,A,a)\) be a space triple, \(n\geq 2\). Set
    \[\pi_n(X,A,a)^\dagger \coloneq \text{ quotient of } \pi_n(X,A,a) \text{ by the normal subgroup generated by } [w*f]\cdot [f]^{-1}\]
\end{defi}

\textbf{Note.} For \(n \geq 3\), the group \(\pi_n(X,A,a)\) is abelian, hence so is \(\pi_n(X,A,a)^\dagger\). For \(n = 2\) \(\pi_2(X,A,a)\) need not be abelian, but \(\pi_2(X,A,a)^\dagger\) is.

Let \(f,g\colon (I^2, \partial I^2, J^1)\to (X,A,a)\), set \(w = g\rvert_{[0,1]}\colon ([0,1], \set{0,1})\to (A,a)\). The rest of the proof is pictures, so good luck understanding without them. I'm sorry. We have \([g]^{-1}\cdot[f]\cdot[g]= [w]*[f]\) See this in Tom Dieck Prop. 6.2.6.

So in particular \([g]^{-1}\cdot[f]\cdot[g] \equiv [f]\) in \(\pi_2(X,A,a)^\dagger\), so it is abelian.

\begin{thm}{Relative Hurewicz with \(\pi_1\)}{}
    Let \((X,A)\) be a path connected space pair. Suppose for all \(a \in A\), \(\incl_*\colon \pi_1(A,a)\to \pi_1(X,a)\) is an isomorphism. Let \(n\geq 2\) be such that \(\pi_i(X,A,a) = 0\) for all \(1\leq i \leq n-1\). Then \(H_i(X,A,\iz) = 0\) for \(0\leq i \leq n-1\) and the modified Hurewicz map
    \[h^\dagger \colon \pi_n(X,A,a)^\dagger \to H_n(X,A,\iz)\]
    is an isomorphism.
\end{thm}

\textbf{Warning.} The hypothesis is on \(\pi_i(X,A,a)\), but the conclusion on \(\pi_n(X,A,a)^\dagger\)!


\section{CW-Approximation}

\enquote{Every topological space can be approximated by a CW-complex}\footnote{See 6 in Lücks notes}. More detailed, you can find a CW-complex with a weak homotopy equivalence to your space.

we state a relative refined version of CW-approximation and proof that.

\begin{defi}{}{}
    Let \((Y,A)\) be a space pair, \(n\geq 0\). A \emph{\(n\)-CW-model} for \((Y,A)\) is a relative CW-complex \((Z,A)\) and a certain map \(f\colon Z\to Y\) such that
    \begin{itemize}
        \item \(f\rvert_{A} = \text{ inclusion } A\hookrightarrow Y\)
        \item \((Z,A)\) is \(n\)-connected,
        \item The map \(f_*\colon \pi_i(Z,z)\to \pi_i(Y,f(z))\) is injective for \(i = n\) and bijective for \(i> n\) for all \(z \in Z\).
    \end{itemize}
\end{defi}

We have
\[A\xhookrightarrow[n\text{-connected}]{\text{relative CW}} Z \xrightarrow[n\text{-coconnected}]{} Y\]

\begin{thm}{}{}
    Let \((Y,A)\) be a space pair, \(A\neq \emptyset\), such that \(A\) is Hausdorff, \(n\geq 0\). Then there is a \(n\)-CW-model \((A,Z,f)\), s.t. \((Z,A)\) has no relative cells of dimension \(\leq n\).
\end{thm}

\textbf{Addendum.} If \(A\) comes with a CW-structure, then \(Z\) can be chosen as a CW-complex that contains A as a subcomplex.

\textbf{Special Case.} \(Y\) is \(n\)-connected, \(n\geq 0\), \(A = \set{y_0}\). Let \((Z,\set{y_0})\) be a \(n\)-CW-model with \(f\colon Z\to Y\), without relative \(i\)-cells for \(0\leq i\leq n\). Then both \(Y\) and \(Z\) are \(n\)-connected, \(f\colon \pi_i(Z,z)\to \pi_i(Y, z)\) is an isomorphism for \(i>n\), hence \(f\) is a weak homotopy equivalence

\begin{proof}
    We will inductively construct
    \[A = Z_n \subseteq Z_{n+1}\subseteq \dots\]
    and \(f_i\colon Z_i\to Y\), \(i\geq n\), such that
    \begin{itemize}
        \item for \(i >n\), \(Z_i\) is obtained from \(Z_{i-1}\) by attaching \(i\)-cells.
        \item \(f_i\rvert_{Z_{i-1}} = f_{i-1}\), \(f_n = \incl\colon A\hookrightarrow Y\)
        \item For all \(z\in Z_i\), \(\pi_j(f,z)\colon \pi_j(Z_i, z)\to \pi_j(Y, f_i(z))\) is
        \begin{itemize}
            \item injective for \(j = n\)
            \item bijective for \(n <j<i\)
            \item surjective for \(j = i\)
        \end{itemize}
    \end{itemize}
    Given this, we take \(Z = \bigcup_{i\geq n} Z_i\) then \((Z,A)\) is a relative CW-complex with cells of dimensions \(\geq n+1\), and \(f = \bigcup_{i\geq n} f_i\) has the desired property.

    \((Z,A)\) is \(n\)-connected by cellular approximation
    \[\begin{tikzcd}
        f_*\colon \pi_j(Z,z) \ar[r]& \pi_j(Y, f(z)) \\
        \pi_j(Z_{j+1},z) \ar[u]\ar[ru, "\cong", "(f_{j+1})_*"'] & \\
    \end{tikzcd}\]
    where the left up map is an isomorphism by cellular approximation.

    For the inductive Step A: make \(\pi_i\) injective

    Step B: make \(\pi_{i+1}\) surjective.

    Suppose \(i\geq n\) and \(Z_i\xrightarrow{f_i}Y\) have been constructed with the desired properties. For each path component \(C\) of \(A\) choose a basepoint \(x_c\in A\) in that component. For each element in the kernel of \((f_i)_*\colon \pi_i(Z_i, x_c)\to \pi_i(Y, f_i(x_c))\) choose a based continuous map  \(q_{c,u}\colon S^i \to Z_i\), s.t. \([g_{c,u}] = u\) Define \(Z_{i+1}'\) as the pushout
    \[\begin{tikzcd}
        \coprod_{C, u\in \Ker({f_i}_*)} S^i \ar[r, "\coprod_{q_c,u}"] \ar[d] & Z_i \ar[r, "f_i"]\ar[d] & Y \\
        \coprod_{C,u} D^{i+1} \ar[r] & Z_{i+1}' \ar[ru, dotted, "f_{i+1}"] Q \\
    \end{tikzcd}\]

    For \(j \leq i+1\) this is bijective for \(j <i\), surjective for \(j = i\) by cellular approximation. I missed a bit here.

    This was step A, now comes step B.

    For each \(C \in \pi_0(A)\) and each element \(v\) of \(\pi_{i+1}(Y, f(x_c))\) choose a representation \(q_{c,v}\colon S^{i+1} \to Y\) s.t. \([q_{c,v}] = v\). Define \(Z_{i+1} = Z_{i+1}'\vee_{\substack{C\in \pi_0 A\\ v \in \pi_{i+1}(Y,f(x_c))}} S^{i+1} \xrightarrow{f_{i+1} = f_{i+1}' \vee q_{c,v}} Y\).

    This now has all required properties, missed the diagramm with explaining everything.
\end{proof}

\newLecture{29.10.2025}

Was ill, might add in later.

This is from Tiens (thank him for it) lecture notes

\iffalse
Reminder:

Let \(f \colon X \to Y\) be a continuous map between simply connected CW-complexes.
Then the following are equivalent:
\begin{enumerate}
    \item \(f\)~is a weak homotopy equivalence.
    \item \(f\)~is a homotopy equivalence (Whitehead theorem).
    \item \(f_* \colon H_i(X; \iz) \to H_i(Y; \iz)\) is an isomorphism for \(i \ge 2\).
\end{enumerate}

Let \(f \colon X \to Y\) be a continuous map between path connected CW-complexes.
Let \(\tilde{f} \colon \tilde{X} \to \tilde{Y}\) be a lift to universal covers.
Then the following are equivalent:
\begin{enumerate}
    \item \(f\)~is a homotopy equivalence.
    \item \(\tilde{f}_* \colon H_i(\tilde{X}; \iz) \to H_i(\tilde{Y}; \iz)\) is an isomorphism for \(i \ge 2\).
\end{enumerate}

Relative Hurewicz theorem with~\(\pi_1\): Let \((X,A)\) be a space pair with \(X\)~and~\(A\) path connected.
Suppose that \(\incl_* \colon \pi_1(A,a) \to \pi_1(X,a)\) is an isomorphism and that \((X,A)\) is \((n+1)\)-connected.
Then \(H_i(X,A; \iz) = 0\) for all \(0 \le i \le n-1\), and the modified Hurewicz map
\begin{equation*}
    h^{\dag} \colon \pi_n(X,A,a)^{\dag} \to H_n(X,A; \iz)
\end{equation*}
is an isomorphism (\(\pi_n(X,A,a)^{\dag}\) is the quotient of \(\pi_n(X,A,a)\) by the group action of the homotopy group of the smaller space).

CW-approximation: Let \((Y,A)\) be a space pair with \(A\) Hausdorff.
Then there is a \(n\)-CW-model \((Z,f)\) for \((Y,A)\) (\(n \ge 0\)), i.\,e.:
\begin{itemize}
    \item \((Z,A)\) is a relative CW-complex with no relative cells in dimension \(0 \le i \le n\) (we can make it an absolute CW-complex by making the attaching maps cellular before attaching);
    \item \(f \colon Z \to Y\) is continuous with \(f|_A = \incl\);
    \item \(/Z,A\)   is \(n\)-connected;
    \item \(\pi_j(f) \colon \pi_j(Z,z) \to \pi_j(Y, f(z))\) is a monomorphism for \(j = n\) and bijective for all \(j > 1\) all \(z \in Z\).
        This means that \(A \to Z \to Y\) with \(A \inj Z\) an \enquote{equivalence} below~\(n\) and \(f \colon Z \to Y\) an \enquote{equivalence} above~\(n\).
\end{itemize}

Inductive construction \(A = Z_n \subseteq Z_{n+1} \subseteq Z_{n+2} \subseteq \dots \subseteq Z = \bigcup_{i \ge n} Z_n\), \(f_i \colon Z_i \to Y\), \(f_i|_{Z_{i-1}} = f_{i-1}\) giving \(f = \bigcup_{i \ge n} f_i \colon Z \to Y\).
If \(\underline{i > n}\), then \((f_i)_* \colon \pi_j(Z_i, z) \to \pi_j(Y, f_i(z))\) is monic for \(j = n\), bijective for \(n < j < 1\) and surjective for \(j = i\).
\fi

\begin{thm}{Lück Thm.~6.8}{}
    Let \(Y\) be any space.
    \begin{enumerate}
        \item There is a \emph{CW-approximation}, i.\,e., a pair \((X,f)\) consisting of a CW-complex~\(X\) and a weak homotopy equivalence \(f \colon X \iso Y\).
        \item Let \((X,f)\) and \((X',f')\) be two CW-approximations of~\(Y\).
            Then there is a continuous map \(g \colon X \to X'\) such that
            \begin{equation*}
                \begin{tikzcd}
                    X \ar[rr, "g"] \ar[dr, "f"', "\sim" sloped]
                    && X' \ar[dl, "f'", "\sim" sloped] \\
                    & Y
                \end{tikzcd}
            \end{equation*}
            commutes up to homotopy.
            Moreover, \(g\)~is unique up to homotopy and a homotopy equivalence.
    \end{enumerate}
\end{thm}

\begin{proof}\leavevmode
    \begin{enumerate}
        \item If \(Y = \varnothing\), take \(X = \varnothing\).
            If \(Y \ne \varnothing\), choose one point \(y_C\) in each path component \(C \in \pi_0(Y)\).
            Let \((Z_C, f_C)\) be a \(0\)-CW-approximation of~\(Y_C\), i.\,e., the path component of~\(C\), so \(f_C \colon Z_C \to Y_C\) is a weak homotopy equivalence.
            Then set \(X = \coprod_{C \in \pi_0(Y)} Z_C \to[\coprod f_C] Y\), which is a CW-approximation.
        \item Recall that if \(X\) is a CW-complex and if \(f \colon Y \to Z\) is a weak homotopy equivalence, then \([X,f] \colon [X,Y] \to [X,Z]\) is bijective.

            Since \(f' \colon X' \to Y\) is a weak homotopy equivalence and since \(X\) is a CW-complex, \([X,f'] \colon [X,X'] \iso [X,Y]\) is bijective.
            So, there is a \(g \colon X \to X'\), unique up to homotopy, such that \(f' \circ g \simeq f\).
            Then \(g\) is a weak equivalence because \(f\)~and~\(f'\) are; hence a homotopy equivalence. \qedhere
    \end{enumerate}
\end{proof}

\subsection{Killing of homotopy groups}

Let \(A\) be a path connected Hausdorff space.
Let \((Z,f)\) be an \(n\)-CW-model of \((CA,A)\) where \(CA = A \times [0,1] / A \times 1\) is the cone.
Then, for \(n \ge 1\),
\begin{equation*}
    \pi_i(f) \colon \pi_i(Z,z) \to \pi_i(CA, f(z)) = 0
\end{equation*}
is monic for \(i = n\) and bijective for \(i > n\).
So, \(\pi_i(Z,z) = 0\) for \(i \ge n\).
Also \((Z,A)\) is a relative CW-complex with relative cells of dimension \(n+1\) or larger.
So, \(\pi_i(A,a) \to \pi_i(Z,a)\) is an isomorphism for \(i < n\).
This means that \(A \to Z\) is an isomorphism for~\(\pi_{<n}\) and \(Z\) has trivial homotopy in~\(\pi_{\ge n}\).

\subsection{Eilenberg-MacLane spaces}

\begin{itemize}
    \item Let \(n \ge 1\), and let \(A\) be a group, abelian if \(n \ge 2\).
        Then there exists a path connected CW-complex \(X = K(X,n)\) such that
        \begin{equation*}
            \pi_i(X,x) \cong
            \begin{cases}
                A & \text{if } i = n, \\
                0 & \text{if } i \ne n.
            \end{cases}
        \end{equation*}
    \item \(X\)~is unique up to homotopy.
    \item Representability of cohomology: for \(Y\) a CW-complex and \(A\) abelian, we have \(H^n(Y,A) \cong [Y, K(A,n)]\).
\end{itemize}

We follow Lück's notes and Schwede's notes (which are in German!).

\begin{defi}{Eilenberg-MacLane space}{}
    Let \(n \ge 1\), and let \(A\) be a group, abelian if \(n \ge 2\).
    An \emph{Eilenberg-MacLane space} of type \((A,n)\) is a pair \((X, \varphi)\) consisting of a based path connected space \((X,x)\) and an isomorphism \(\varphi \colon \pi_n(X,x) \iso A\) such that \(\pi_i(X,x) = 0\) for all \(i \ge 1\) and \(i \ne n\).
    Shorthand notation: \enquote{\(X\)~is a~\(K(A,n)\)}.
\end{defi}

(In full generality, Eilenberg-MacLane spaces are very abstract.)

\begin{example}
    \(S^1\) is a \(K(\iz,1)\) since \(\pi_1(S^1,z) \cong \iz\) and \(\pi_i(S^1,z) = 0\) for all \(i \ge 2\) by covering space theory, because the universal cover \(\exp \colon \ir^ \to S^1\) of~\(S^1\) has a contractible total space.
    More generally, if \(p \colon \tilde{X} \to X\) is a universal cover of a path connected space with \(\tilde{X}\) contractible, then \(X\) is a \(K(G,1)\) with \(G \cong \operatorname{Deck}(p)\).
    Hence:
    \begin{itemize}
        \item \(\ir P^0 = K(\iz/2, 1)\) with universal cover \(S^{\infty} \simeq *\).
        \item \(S^1 \times \dots \times S^1 = K(\iz^m, 1)\) (\(m\)~times) with universal cover~\(\ir^^m\).
        \item Klein bottle (insert diagram of a square with arrows on the edges that indicate how the Klein bottle is glued) is a \(K(\iz \rtimes \iz, 1)\) with universal cover~\(\ir^2\).
            Here, \(\rtimes\)~is a semidirect product with \(\iz\)-action on~\(\iz\) by sign.
    \end{itemize}
\end{example}

\begin{example}
    \(\ic P^{\infty}\)~is a \(K(\iz,2)\) because of the locally trivial fibre bundle \(S(\ic^{\infty}) \to \ic P^{\infty}\), \(x \mapsto \ic x\) with fibre~\(S^1\). 
    This is a Serre fibration, so the long exact sequence of homotopy groups gives
    \begin{equation*}
        \set{0} = \pi_i(S(\ic^{\infty}), *) \to \pi_i(\ic P^{\infty}, *) \iso[\partial] \pi_{i-1}(S^1, x) \to \pi_{i-1}(S(\ic^{\infty}), *) = \set{0}.
    \end{equation*}
    Thus,
    \begin{equation*}
        \pi_i(\ic P^{\infty}, *) \cong
        \begin{cases}
            \iz & \text{if } i = 2, \\
            0 & \text{if } i \ne 0.
        \end{cases}
    \end{equation*}
    (Assume that Eilenberg-MacLane spaces are CW-complexes is not a huge loss of generality because of CW-approximation\@.)
\end{example}

\begin{example}
    If \(X\) is a \(K(A,n)\) and \(Y\) is a \(K(B,n)\), then \(X \times Y\) is a \(K(A \times B, n)\).
\end{example}

Construction of \(K(A,n)\)'s.
We distinguish two cases, whose constructions are very similar.

\(K(G,1)\)'s: let \(G\) be a group.
Start with the following \(2\)-dimensional CW-complex
\begin{equation*}
    CG \coloneq
    \left({\bigvee_{g \in G} S_g^1}\right) \cup \bigcup_{(h,k) \in G^2} D_k^2
\end{equation*}
The \(2\)-cells are attached as follows: let \(i_g \colon S^1 \to \bigvee_{g \in G} S_g^1\) denote the inclusion of the \(g\)th 
For \((h,k) \in G^2\), consider the following map \(\alpha_{h,k} \colon S^1 \to \bigvee_{g \in G} S_g^1\) which is \(i_h\) on the first third, \(i_k\) on the second third, and \(i_{hk}^{-1}\) on the last third of~\(S^1\) (insert image of circle with three equal segments, each segment being named after the respective map).
By covering space theory\slash Seifert-van~Kampen, we know that \(\pi_1(\bigvee_{g \in G} S_g^1, *)\) is a free group generated by the elements \([i_g]\).
By cellular approximation, \(\pi_1(\bigvee_{g \in G} S_g^1, *) \to \pi_1(CG, *)\) is surjective, so \(\pi_1(CG, *)\) is generated, as a group, by \([i_g \colon S^1 \to \bigvee_{g \in G} S_g^1 \inj CG]\).
The two cell indexed by \((h,k) \in G^2\) witnesses that \([i_h] [i_k] [i_{hk}]^{-1}\) map to~\(1\) in \(\pi_1(CG, *)\), so \([i_h] [i_k] = [i_{hk}]\) in \(\pi_1(CG, *)\).
Thus, \(G \to \pi_1(CG, *)\), \(g \mapsto [S^1 \inj[i_g] \bigvee_{} S_g^1 \inj CG]\) is a surjective group homomorphism.
By Seifert-van~Kampen, attaching a \(2\)-cell to a path connected space precisely kills the normal subgroup generated by the class of the attaching map.
Thus,
\begin{equation*}
    \pi_1(CG,*)
    = \frac{\text{free group on elements of } G}{{[h] [k] = [hk]}}
    \cong G.
\end{equation*}
So, \(CG\) is a connected space with correct~\(\pi_1\), so by killing the homotopy group~\(\pi_i\) for \(i \ge 2\), we obtain a \(K(G,1)\).

Construction for \(K(A,n)\) for \(n \ge 2\) and \(A\) abelian: choose a presentation of \(A\) as an abelian group:
\begin{equation*}
    \iz[I] \xrightarrow{[d]} \iz[J] \xrightarrow{[\eps]} A \to 0
\end{equation*}
is an exact sequence of abelian groups with \(I\)~and~\(J\) some sets.
We set \(X_n = \set{x} \bigcup_{J \times S^{n-1}} J \times D^n \cong \bigvee_J S^n\) as an \(n\)-dimensional CW-complex with one \(0\)-cell and no cells in dimension \(1 \le i \le n-1\).
So, \(X_n\)~is \((n-1)\)-connected by cellular approximation.
Then \(\iz[J] \iso \tilde{H}_n(X_n; \iz)\), \(j \mapsto (i_j)_* [D^n, S^{n-1}]\) where \([D^n, S^{n-1}] \in H_n(D^n, S^{n-1}; \iz)\) is a chosen generator and \(i_j .: D^n \to \set{x} \cup_{J \times S^{n-1}} J \times D^n\) is the characteristic map of the \(j\)th \(n\)-cell.
By the Hurewicz theorem, \(h \colon \pi_n(X_n, *) \to H_n(X_n; \iz) \cong \iz[J]\) is an isomorphism.
For each \(i \in I\), there is a unique \([\alpha_i] \in \pi_n(X_n, *)\) where Hurewicz image is~\(d(i)\).
Let \(\alpha_i \colon S^n \to X_n \cong \bigvee_{j \in J} S^n\) be a representative for this homotopy class.
We define \(X_{n+1}\) by attaching \((n+1)\)-cells along all the \(\alpha_i\)'s: \(X_{n+1} = X_n \cup_{I \times S^n, \alpha_i} I \times D^{n+1}\).
The cellular chain complex of~\(X_n\) gives
\begin{equation*}
    \begin{tikzcd}
        H_{n+1}(X_{n+1}, X_n; \iz) \ar[r]
        & H_n(X_n; \iz) \ar[r]
        & H_n^{\mathrm{cell}}(X_{n+1}; \iz) \ar[r]
        & 0 \\
        \iz[I] \ar[u, "\sim"' sloped, "{i \mapsto (\beta_i)_* [D^{n+1}, S^n]}"] \ar[r, "d"]
        & \iz[J] \ar[r] \ar[u, "\sim"' sloped]
        & A \ar[r] \ar[u, dashed, "\sim"' sloped, "\exists!"]
        & 0
    \end{tikzcd}
\end{equation*}
with exact rows where \(\beta_i \colon D^{n+1} \to X_{n+1}\) is the characteristic map for the \(i\)th \((n-1)\)-cell.
The left square commutes by construction, so we obtain the dashed map.
Hence we have an isomorphism \(A \cong H_n(X_{n+1}; \iz)\).
Since \(X_{n+1}\) is \((n-1)\)-connected, the Hurewicz map is an isomorphism \(h \colon \pi_n(X_{n+1}, *) \iso H_n(X_{n+1}; \iz) \cong A\).
By killing the homotopy groups~\(\pi_i\) for \(i \ge n+1\) from~\(X_{n+1}\) we obtain a \(K(A,n)\).

Next aim: for \(X\) an \((n-1)\)-connected based CW-complex,
\begin{equation*}
    \pi_n \colon [X, K(A,n)]_* \to \hom(\pi_n(X,*), A), \qquad
    [f \colon X \to K(A,n)] \mapsto \varphi \circ \pi_n(f),
\end{equation*}
where \(\varphi \colon \pi_n(K(A,n), *) \iso A\), is bijective.

\begin{lemma}[Schwede Lem.~49]
    Let \((X,Y)\) be a relative CW-complex and \(Z\) any space.
    Suppose that for all \(m \ge 1\) such that \((X,Y)\) has at least one relative \(m\)-cell, \(\pi_{m-1}(Z,z) = 0\) for all \(z \in Z\).

    Then any map \(Y \to Z\) can be extended to a map \(X\to Z\).
\end{lemma}

\begin{proof}
    By induction on the relative skeleta, we construct continuous maps \(f_m \colon X_m \to Z\) such that \(f_{-1} = f\), \(f_i|_{X_{i-1}} = f_{i-1}\).
    We define \(f_0 \colon X_0 = Y \amalg I \to Z\) by sending the new \(0\)-cells arbitrarily to~\(Z\).
    For \(n \ge 1\), if \(X_i \ne X_{i-1}\), then there is at least one relative \(i\)-cell.
    The attaching map~\(\alpha_j\) for every relative \(i\)-cell becomes null homotopic after composition with \(f_{i-1} \colon X_{i-1} \to Z\)  because \(\pi_{i-1}(Z,*) = 0\).
    We choose a null homotopy \(h_j \colon D^{i+1} \to Z\) of this composite and define \(f_i \colon X_i \to Z\) by \(h_j\) on the \(j\)th \(i\)-cell.
    \begin{equation*}
        \begin{tikzcd}
            \coprod_J S_j^i \ar[r, "\alpha_j"] \ar[d]
            & X_{i-1} \ar[r, "f_{i-1}"] \ar[d]
            & Z \\
            \coprod_J D_j^{i+1} \ar[r] \ar[urr, bend right=40, "h_j"']
            & X_i \ar[ur, dashed, "f_i"]
        \end{tikzcd}
    \end{equation*}
    (map exist by universal property of the pushout).
    This completes the induction step.
    Now, set \(f_{\infty} = \bigcup_{i \ge 0} f_i\).
\end{proof}


\newLecture{3.11.2025}

We want to show, that for CW-complexes, Eilenberg-Maclane-spaces are unique up to homotopy.

\begin{thm}{}{test}
    Let \(n \geq 1\). Let \(Y\) be an \((n-1)\)-connected CW-complex. Let \(Z\) be a based space s.t. \(\pi_m(Z,z) = 0\) for \(m > n\). Then for every group-homomorphism
    \[\Phi\colon \pi_n(Y,y) \to \pi_n(Z,z)\]
    there is a based continuous map \(f\colon Y \to Z\), s.t. \(\pi_n(f) = \Phi\). and such an \(f\) is unique up to based homotopy. Equivalently: the map
    \[\pi_n\colon[Y, Z]_* \to \Hom_{\mathrm{Grp}}(\pi_n(Y,y) ,\pi_n(Z,z))\]
    is bijective.
\end{thm}

\begin{proof}
    If \(Y\) is based homotopy equivalent to \(Y'\) and the claim holds for \(Y'\), then it holds for \(Y\). Then we can assume wlog, that \(Y\) has one \(0\)-cell and no cells in dimensions \(1\leq i\leq n-1\). So \(Y_n = \bigvee_I D^n/S^{n-1}\).
    \begin{description}
        \item[Step 1] we construct a continuous map \(f_n\colon Y_n\to Z\) such that the following commutes:
        \[\begin{tikzcd}
            \pi_n(Y_n, *)\ar[rr, "(f_n)_*", two heads]\ar[rd, "\incl_*"] && \pi_n(Z,z) \\
            & \pi_n(Y,y) \ar[ru, "\Phi"] 
        \end{tikzcd}\]
        We choose a homeomorphism \(S^n \cong D^n/\partial D^n\) and characteristic maps \(\chi_i\colon D^n \to Y\) for all \(i\in I\). then the \(S^n \cong D^n/\partial D^n\xrightarrow{\chi_i} Y_n \hookrightarrow Y\) represent a class \([\chi_i]\in \pi_n(Y, y)\), so \(\Phi[\chi_i] \in \pi_n(Z,z)\). Let \(w_i\colon S^n\to Z\) be a based map such that \([w_i] = \Phi[\chi_i]\). We define
        \[f_n\colon Y_n = \bigvee_{i\in I} D^n/\partial D^n \cong \bigvee_{i\in I} S^n \to Z\]
        as \(w_i\) on the wedge summand indexed by \(i\). Then the maps commute by construction on the classes \([\chi_i]\). Since the class \([\chi_i]\) for \(i\in I\) generates the group \(\pi_n(Y_n, *)\) (Hurewicz theorem). Since all the maps in the commutative diagram are group homomorphisms and commute on generators, the diagram commutes.
        \item[Step 2] We extend \(f_n\) continuously to \(f_{n+1}\colon Y_{n+1} \to Z\).
        
        Let \(J\) be an index set for the \((n+1)\)-cells of \(Y\). For \(j \in J\) let \(\chi_j\colon D^{n+1}\to Y\) be a characteristic map for the \(j\)-th \(n+1\)-cell. The associated attaching map is \(\chi_j\rvert_{S^n}\colon S^n \to Y_n \cong \bigvee_I D^n/ \partial D^n \cong \bigvee_I S^n\). Since \(Y_n\) is path-connected \(\chi_j\rvert_{S^n}\) is homotopic to a based map. By the HEP for \((D^{n+1}, S^n)\), we can extend the homotopy between \(\chi_j\rvert_{S^n}\) and the based map \(\alpha\colon S^n \to Y_n\) to the \((n+1)\)-cell. So
        \[[S^n \xrightarrow{\alpha} Y_n \hookrightarrow Y] \in \pi_n(Y,y)\]
        is the zero homotopy class, i.e. \([\alpha] \in \pi_n(Y_n, y)\) lies in the kernel of \(\incl_*\). By commutativity of the (way above) commutative diagram, we get \((f_n)_*[\alpha] = 0\) in \(\pi_n(Z, z)\). So the composite
        \[S^n \xrightarrow{\alpha} Y_n \xrightarrow{f_n} Z\]
        is nullhomotopic. Hence also \(S^n \xrightarrow{\chi_j\rvert_{S^n}} Y_n \xrightarrow{f_n} Z\) is nullhomotopic. We choose a continuous extension \(\beta_j\colon D^{n+1} \to Z\) of \(\chi_j\rvert_{S^n}\) and define \(f_{n+1}\colon Y_{n+1}= (\bigvee_I S^n) \cup_{J\times S^n} J\times D^{n+1}\to Z\) by \(f_n\) on \(Y_n\) and by \(\beta_j\) on the \(j\)-th \((n+1)\)-cell.
        \item[Step 3] We apply the previous lemmo to the relative CW-complex \((Y, Y_{n+1})\) of which all relative cells have dimension \(n\geq 2\), Since \(\pi_i(Z,z) = 0\) for all \(i \geq n+1\), the map \(f_{n+1}\colon Y_{n+1}\to Z\) extends to a map \(f\colon Y \to Z\).
        
        We have \(\Phi = f_*\) since \(\incl_*\) is surjective and we can cancel it on the left.
    \end{description}

    We still have to show uniqueness up to based homotopy. Let \(f, f'\colon Y\to Z\) be based continuous maps, such that \(\pi_n(f) = \pi_n(f')\colon \pi_n(Y)\to \pi_n(Z)\). Since \(Y_{n-1} = {*}\), they agree on \(Y_{n-1}\). Let \(I\) be an index set as before for the \(n\)-cells, \(\chi_i\colon D^n \to Y\) characteristic maps for the \(i\)-th \(n\)-cell. \([S^n\cong D^n/\partial D^n \xrightarrow{\chi_i} Y] \in \pi_n(Y, y)\) which implies \([f \circ \chi_i] = f_*[\chi_i] = f'_*[\chi_i] = [f'\circ \chi_i]\). So \(f\circ \chi_i\), \(f'\circ \chi_i\colon D^n/\partial D^n \cong S^n \to Z\) are based homotopic. Choose succh a homotopy for each \(i\in I\), and glue them into a homotopy \(H\)
    \[f\rvert_{Y^n}\sim f'\rvert_{Y_n} \colon \bigvee_I S^n \to Z\]
    We apply the lemma from last time to the relative CW-complex \((Y\times [0,1], Y\times \set{0} \cup Y_n\times [0,1]\cup Y\times \set 1)\) all of whose relative cells have dimension \(\geq n+2\). So by the lemma, the map
    \[f\cup H \cup f'\colon Y\times \set 0 \cup Y_n \times [0,1] \cup Y\times \set 1\]
    has a continuous extension \(K\colon Y\times [0,1]\to Z\). This map is a based homotopy from \(f\) to \(f'\).
\end{proof}


\begin{corollary}
    Let \(n \geq 1\), \(A\) a group, abelian if \(n \geq 2\). Let \((X, \phi), (Y, \psi)\) be Eilenberg-Maclane spaces of type \((A,n)\). If \(X\) is a CW-complex, then there is a based continuous map \(f\colon X\to Y\), unique up to based homotopy, such that
    \[\begin{tikzcd}
        \pi_n(X,x) \ar[rd, "\phi", "\cong"'] \ar[rr, "f_*", "\cong"'] && \pi_n(Y, y) \ar[dl, "\cong", "\psi"'] \\
        & A & \\
    \end{tikzcd}\]
    commutes.

    Moreover \(f\) is a weak homotopy equivalence, and a homotopy equivalence if \(Y\) is also CW.
\end{corollary}

\section{Representativity of cohomology}

Here \(A\) is abelian group and \(n \geq 0\). The Aim of this subsection is a natural transformation of functors in \(Y\)
\[[Y, K(A,n)]\to H^n(Y; A)\]
that is an isomorphism for CW-complexes. We want to construct a group structure on \([Y, K(A,n)]\), the \enquote{Homotopy group structure on \(K(A,n)\)}.

Our hypothesis are \(n\geq 1\), \(A\) abelian group, \(K(A,n)\) some EM-space of type \((A,n)\) that is a CW-complex. Because \(A\) is abelian, the addition \(A\times A\to A, \; (a,b)\mapsto a+b\) and inverse \(A\to A, \; a\mapsto a^{-1}\) are group homomorphisms. We apply the previous theorem to \(Y = K(A,n)\times K(A,n)\) (an EM-space of type \((A\times A, n)\)) and \(Z F K(A,n)\) and the momomorphism
\[\pi_n(K(A,n)\times K(A,n))\cong \pi_n(K(A,n))\times \pi_n(K(A,n)) \xrightarrow{\cong} A\times A \xrightarrow{+} A \xrightarrow{\cong} \pi_n(K(A,n))\]
So there is a continuous map \(\mu\colon K(A,n)\times K(A,n) \to K(A,n)\) that induces the addition on \(\pi_n\)\footnote{some mumbling about uncountable \(A\) and compactly generated topology, retopologizing \(K(A,n)\times K(A,n)\)}.

Similarly, the composite \(\pi_n(K(A,n))\cong A \xrightarrow{a\mapsto -a} A \cong \pi_n(K(A,n))\) is realized by a based continuous map \(i\colon K(A,n)\to K(A,n)\) unique up to based homotopy.

\(\mu\) is associative up to based homotopy:
\[\mu\circ (\mu \times \Id), \mu \circ (\id \times \mu)\colon K(A,n)^3 \to K(A,n) \]
Bothe induce the same map on \(\pi_n\), so by the theorem, \(\mu\circ(\mu \times \Id) \sim \mu\circ (\id \times \mu)\), i.e. \(\mu\) is homotopy associative.

Let \(\tau\colon K(A,n)\times K(A,n)\to K(A,n)\times K(A,n)\) be the flip map \(\tau(x,y) = (y,x)\). Then \(\mu\circ \tau, \mu\colon K(A,n)\times K(A,n) \to K(A,n)\) are by applying the theorem homotopic, i.e. \(\mu\) is homotopy commutative.

Since the constant map \(*\colon K(A,n)\to K(A,n)\) to the basepoint is \(0\) on \(\pi_n\), the theorem shows that \(\mu \circ (\id, i) = *\). So \(i\colon K(A,n)\to K(A,n)\) is a homotopy inverse for \(\mu\).

\textbf{Upshot.} \(K(A,n)\) is a \enquote{homotopy abelian group}.

\textbf{Remark.} It is possible to even take \(K(A,n)\) as a topological abelian group. E.g. \(\abs{\tilde A[\Delta^n/\partial \Delta^n]}\).

\begin{construction}
    For every (compactly generated) space \(Y\), the set of homotopy classes of maps \([Y, K(A,n)]\) becomes an abelian group via
    \[+ \colon [Y, K(A,n)]\times [Y, K(A,n)] \to [Y, K(A,n)]\]
    \[([f\colon Y\to K(A,n)], [g\colon Y\to K(A,n)]) \mapsto [Y\xrightarrow{(f,g)} K(A,n)\times K(A,n)]\xrightarrow{\mu_*} [Y, K(A,n)]\]
    The homotopy associativity, homotoyp commutativity and homotopy inverse properties of \((\mu, i)\) imply that \(+\) is associative, commutative, and has an inverse \(-[f] = [i\circ f]\).
\end{construction}

\newLecture{05.11.2025}

\begin{example}
    We have in this class not talked about Vector- and Line-bundles. Schwede encourages us to do so. \(\IR P^\infty\) and \(\ic P^\infty\) are EM-spaces of type \((\iz, 1)\) and \((\iz, 2)\) respectively, so they have \enquote{homotopy group strtuctures}.

    So we get abelian group structures on \([Y, \ir P^\infty]\), and \([Y, \ic P^\infty]\).

    Also \(\ir P^\infty\) and \(\ic P^\infty\) classify real/complex line bundles for compact spaces \(Y\),
    \[[Y, \ir P^\infty] \xrightarrow{\cong} \mathrm{Pic}_\ir(Y) = \text{groups under \(\otimes\) of isoclasses of line bundles over } Y\]
    \[[f\colon Y\to \ir P^\infty \mapsto f^*(\gamma)]\]
    That is an iso of abelian groups \(\mu^*(\gamma) \cong p_1^*(\gamma) \otimes p_2^*(\gamma)\) as line bundles over \(\IR P^\infty\times \ir P^\infty\).

    Similarly \([Y, \ic P^\infty] \cong \mathrm{Pic}_\ic (Y)\).
\end{example}

\begin{example}
    A strictly associative \& commutative model for \(K(\iz/2, 1)\) and \(K(\iz/2, 2)\).

    The ?? as polynomial algebra on \(\ic[x]\) has no zero-divisors. So multiplication restricts to a commutative and associative operation
    \begin{itemize}
        \item \((\ic[x]\setminus \set 0)\times (\ic[x]\setminus \set 0) \to \ic[x]\setminus \set 0\)
        For \(z\in \ic \setminus 0\) \(z(fg) = (zf)g = f(zg)\) for \(frg \in \ic[x]\) So this descnds to a well defined map
        \[(\ic[x]\setminus \set 0)/\ic^* \times \ic[x]\setminus \set 0/\ic^* \to (\ic[x]\setminus \set 0)/\ic^*\]
        which then is a multiplication on \(\ic P^\infty\). This structure is associative and commutative, but not invertible.
    \end{itemize}
\end{example}

\begin{lem}{Fundamental class (non-standard-notation)}{}
    Let \((X, \phi)\) be an EM-space of type \((A,n)\). Then there is a unique class \(\iota = \iota_{A,n}\in H^n(X,A)\) such that the composite
    \[\pi_n(X, *)\xrightarrow{\text{Hurewicz}} H_n(X,\iz)\xrightarrow{\Phi(\iota)} A\]
    equals \(\phi\colon \pi_n(X,x)\xrightarrow{\cong} A\).
    Where we have
    \[\Phi\colon H^n(X,A)\twoheadrightarrow \Hom(H_n(X,\iz),A)\]
    is the map from UCT.

    
    \(\iota\) is called the fundamental class.
\end{lem}

\begin{proof}
Since \(X\) is \((n-1)\)-connected and \(A\) abelian, the Hurewicz homomorphism \(\pi_n(X,*)\to H_n(X,\iz)\) is an isomorphism. So
\[\phi\circ h^{-1} \colon H_n(X,\iz)\xrightarrow{\cong} A\in \Hom (H_n(X,\iz), A)\]
Since \(X\) is \((n-1)\)-connected, the \(\Phi\colon H^n(X,A)\xrightarrow{\cong} \Hom(H_n(X,\iz), A)\) is an isomorphism and \(\iota\mapsto \phi\circ h^{-1}\).
\end{proof}

\begin{construction}
    An EM-space of type \((A, 0)\) is the group \(A\) with the discrete topology and \(\iota_{A, 0} \in H^0(A,A)\) is the class represented by \(\id\colon A\to A\).
\end{construction}

\begin{thm}{Group homomorphism}{}
    Let \(n\geq 0\), \(A\) any abelian group. Then for all (compactly generated) \(Y\), the map \([Y, K(A,n)]\to H^n(Y,A)\), \([f\colon Y\to K(A,n)]\mapsto f^*(\iota_{A,n})\) is an homomorphism of groups.
\end{thm}

\begin{proof}
    In 2 steps
    \begin{description}
        \item[Step 1] The universal example: \(Y = K(A,n)\times K(A,n)\) and \(f = \mu\colon K(A,n)^2\to K(A,n)\). Let \(p_1, p_2\colon K(A,n)^2\to K(A,n)\) denote the two projections. Then
        \[[p_1] + [p_2] = [K(A,n)^2 \xrightarrow[\id]{(p_1, p_2)} K(A,n)^2\xrightarrow{\mu} K(A,n)]\]
        So we neet to show that \(\mu^*(\iota)= p_1^*(\iota) + p_2^*(\iota)\) in \(H^n(K(A,n)\times K(A,n), A)\). By UCT and Hurewicz
        \[\begin{split}
            &H^n(K(A,n)\times K(A,n), A)\overset{\Phi}{\cong} \Hom(H_n(K(A,n)\times K(A,n),\iz), A) \\
            &\overset{\text{Hurewicz}}{\cong } \Hom(\pi_n(K(A,n), *)^2, A) \cong \Hom(A\times A, A)
        \end{split}\]
        The composit isomoprhism sends \(\mu^*(\iota)\mapsto \text{ Addition } A\times A\to A\) and \(p_1^*(\iota) \mapsto p_1\colon A\times A\to A\) and \(p_2^*(\iota)\mapsto p_2\colon A\times A\to A\). In \(\Hom(A\times A, A)\) the relation \(\text{Addition } = p_1 + p_2\) holds.
        \item[Step 2] General case: Let \(Y\) be a compactly generated space, \(f, g \colon Y\to K(A,n)\). Then
        \[\begin{split}
            ([f]+ [g])^*(\iota) &= [\mu\circ (f,g)]^*(\iota) = (f,g)^*(\mu^*(\iota)) \\
            &\overset{\text{step }1}{=} (f,g)^*(p_1^*(\iota) + p_2^*(\iota)) \\
            &= (f,g)^*(p_1^*(\iota)) + (f,g)^*(p_2^*(\iota)) \\
            &= (p_1\circ (f,g))^*(\iota) + (p_2\circ (f,g))^*(\iota) = f^*(\iota) + g^*(\iota) \\ 
        \end{split}\]
    \end{description}
\end{proof}

\begin{lem}{}{}
    Let \(n\geq 1\), \(A\) an abelian group. \(Y\) an based CW-complex. Then the forgetful map
    \[[Y, K(A,n)_*\to [Y, K(A,n)]]\]
    is bijective.
\end{lem}

\begin{proof}
    Earlier we have done an exercise: \(Y\) is a non-degenerately\footnote{inclusion of base point has HEP} based space, \(Z\) any based space. Then the forgetful map
    \[[Y,Z]_* \to [Y, Z]\]
    is surjective if \(Z\) is path-connected and bijective if \(Z\) is simply connected.

    This implies the lemma for \(n\geq 2\) and surjectivity for \(n = 1\). Injectivity for \(n=1\): Let \(f,g \colon Y\to K(A,1)\) be freely homotopic. Then \(f_* = g_*\colon H_1(Y,\iz)\to H_1(K(A,1),\iz)\). Since \(A\) is abelian, \(\pi_1(K(A,1), *)\xrightarrow{\cong} H_1(K(A,1),\iz)\). So \(\pi_1(f) = \pi_1(g)\). So by the earlier theorem \ref{thm:test}, \(f\) and \(g\) are based homotopic.
\end{proof}

\begin{thm}{}{}
    Let \(n\geq 0\), \(A\) an abelian group. Then for all CW-complexes \(Y\), the homomorphism
    \[[Y, K(A,n)]\mapsto H^n(Y,A), \qquad [f]\mapsto f^*(\iota_{A,n})\]
    is an isomorphism.
\end{thm}

\begin{proof}
    Both \([\_, K(A,n)]\) and \(H^n(\_, A)\) takes disjoint unions in \(Y\) to products of abelian groups. Every CW-complex \(Y\) is the disjoint union of its path-components. So we can assume wlog, that \(Y\) is path-connected.

    Induction on \(n\). For \(n = 0\) \([Y, K(A,0)] = [Y, A^{\text{disc}}] \cong A\) and also \(H^0(Y,A) = \Hom(\pi_0(Y), A)\cong A\). We do not check, that the map is indeed the identity, which we get between these two.

    Now for \(n\geq 1\). By the lemma we can replace free homotopy classes by based homotopy classes. We first treat a special case:

    \begin{description}
        \item[Special Case] \(Y\) is \(n-1\)-connected. Then \(H_{n-1}(Y, \iz)\) is trivial for \(n \geq 2\) or free \(n= 1\), so \(\Ext(H_{n-1}(Y,\iz), A) =0\). So the UCT provides an isomorphism
        \[\Phi\colon H^n(Y, A)\to \Hom(H_n(Y, \iz), A)\]
        also the Hurewicz map \(\pi_n(Y, *)\to H_n(Y, \iz)\) is an isomorphism for \(n\geq 2\) or the universal homomorphism into an abelian group for \(n =1\). In any case
        \[h^*\colon \Hom(H_n(Y,\iz), A) \xrightarrow{\cong} \Hom(\pi_n(Y,*), A)\]
        is bijective. The composite
        \[[Y, K(A,n)]_* \xrightarrow{[f]\mapsto f^*(\iota)} H^n(Y,A) \xrightarrow[\cong]{\Phi} \Hom(H_n(Y, \iz), A) \xrightarrow[h^*]{\cong} \Hom(\pi_n(Y,*), A)\]
        sends \(f\) to \(\phi\circ \pi_n(f)\colon \pi_n(Y, *)\to A\). So again by theorem \ref{thm:test}, the composite is bijective. Also \(\Phi\) and \(h^*\) are bijective. So the first map is bijective
        \item[General case] We consider \(Y \cup_{Y_n-1} C(Y_{n-1}) = Y \cup_{Y_{n-1} < 0} Y_{n-1} \times [0,1] \cup_{Y_{n-1}\times 1} \set{*}\). This is \(n-1\)-connected.
        We consider the following commutative diagram:
        \[
        \begin{tikzcd}
            {[\Sigma Y_{n-1}, K(A,n)]_*} \ar[r, "p^*"] \ar[d] 
            & {[Y \cup_{Y_{n-1}} C Y_{n-1}, K(A,n)]_*}
                \ar[d, "\cong", "\text{(special case)}"'] 
                \ar[r, "\iota^*"] 
            & {[Y, K(A,n)]_*} \ar[d] \ar[r] & 0 \\
            H^n(\Sigma Y_{n-1}, A) \ar[r, "p^*"] 
            & H^n(Y \cup_{Y_{n-1}} C Y_{n-1}, A) \ar[r] 
            & H^n(Y, A) \ar[r] & 0
        \end{tikzcd}
        \]

        Here \(i\) and \(p\) are inclusion and projection \(p\colon Y\cup_{Y_{n-1}} C Y_{n-1}\to \Sigma Y_{n-1}\).

        \textbf{Claim.} The upper row is exact:
        
        Surjectivity of \(i^*\): all relative cells of \((Y\cup_{Y_{n-1}} C Y_{n-1}, Y)\) have relative cells of dimension at most \(n\). But \(\pi_i(K(A,n), *) = 0\) for all \(i < n\). By a previos \enquote{extension lemma}, every continuous map \(Y\to K(A,n)\) admits a continuous extension to \(Y\cup_{Y_{n-1}} CY_{n-1}\). Since \(p\circ i\) is nullhomotopic, \(i^* \circ p^* = 0\). 
        Suppose \(f\colon Y\cup_{Y_{n-1}} C Y_{n-1}\to K(A,n)\) is based s.t. \(i^*[f] = 0\). So \(f\rvert_{Y}\colon Y\to K(A,n)\) is based nullhomotopic. The HEP of the pair \((Y\cup_{Y_{n-1}} CY_{n-1}, Y)\) lets us replace \(f\) by a based homotopic map \(f'\), s.t. \(f'\rvert_Y = \const_*\). So \(f'\) ??

        The lower sequence is also exact by the l.e.s. for a CW pair \((Y\cup_{Y_{n-1}} C Y_{n-1}, Y)\).

        The left map in the diagram is surjective. We let \(\rho\colon K(A,n-1)\to \Omega K(A,n)\) be the unique up to based homotopy map that induces
        \[\pi_{n-1}(K(A,n-1)) \xrightarrow{\rho_*} \pi_{n-1}(\Omega K(A,n))\cong \pi_n(K(A,n)) \cong A \]
        Such that the composition is \(\phi_{n-1}\).

        Let \(\kappa_n\colon \Sigma K(A, n-1)\to K(A,n)\) be the adjoint of \(\rho\).

        Then the following commutes: I did not manage to copy the diagram.

        So the right vertical map is an isomorphism by the 5-lemma.
    \end{description}

\end{proof}

\newLecture{10.11.2025}

Schwede remarks that when he was a student someone cleaned the board for the professor.

\begin{example}
    If you have already heard about vector bundles somewhere:

    \(K(\IZ/2, 1) = \ir P^{\infty}\). For \(X\) a CW-complex, Also \(\mathrm{Pic}(X) \xleftarrow{\cong} [X, \ir P^\infty] H^1(X,\iz /2)\). \([\gamma\colon E\to X] \mapsto w_1(\gamma)\) is called the first Stiefel-Whitney class. So real line bundles are completely classified by their first Stiefel-Whitney class.
\end{example}

\begin{example}
    \(K(\iz, 2) = \ic P^\infty\). Then
    \[\mathrm{Pic}^\ic(X)\xleftarrow{\cong} [X, \ic P^\infty] \xrightarrow{\cong} H^2(X,\iz)\]
    so complex line bundles \([\gamma\colon E\to X] \mapsto c_1(\gamma)\) are classified by their first Chern class.
\end{example}

\section{Cohomology operations and the Steenrod algebra}

\textbf{Aim.} Define \enquote{Steenrod squares}, natural homomorphisms
\[\mathrm{Sq}^i\colon H^n(X, \IF_2) \to H^{n+i}(X, \IF_2)\]
together with properties and applications.

Schwede will follow some notes of his own which are not yet finalized and he will eventually publish on his website.

\begin{defi}{Cohomology operation}{}
    Let \(A,B\) be abelian groups. A \emph{A cohomology operation} of type \((A,n,B,m)\) is a natural transformation
    \[\tau = \set{\tau_X\colon H^n(X,A)\to H^m(X,B)}\]
    of set valued functors on topological spaces. Specifically these operations need not be group homomorphisms.

    We write
    \[\Oper(A,n,B,m) = \text{abelian group under pointwise addition of all such operations}.\]
    This is not standard notation.
\end{defi}

\begin{lem}{}{}
    The map \(\Oper(A,n,B,m)\to H^m(K(A,n), B)\) given by \(\tau\mapsto \tau_{K(A,n)}(\iota_{A,n})\) is an isomorphism of groups.
\end{lem}

\begin{proof}
    On the homotopy category of CW-complexes, the pair \((K(A,n), \iota_{A,n})\) represents the functor \(H^n(\_; A)\). For any other functor \(F\colon Ho(\text{CW-complexes})\to \mathbf{Sets}\) we have
    \[\mathrm{Nat}(H^n(\_, A), F) \xrightarrow{\cong} F(K(A,n))\]
    by the Yoneda lemma. Apply this to \(F = H^n(\_, B)\) yields a bijection
    \[\mathrm{Nat}_{Ho\text{CW}\to \mathbf{Sets}}(H^n(\_, A), H^m(\_, B))\cong H^m(K(A,n), B)\]

    CW-approximation \(X^{CW}\xrightarrow{\sim} X\) of spaces \(X\) are natural and unique up to homotopy, and we are considering functors on \(\mathbf{Top}\) which are homotopy invariant and take weak equivalences to isomorphisms.
    \[\begin{tikzcd}
        H^n(X^{CW}, A) \ar[r, "\tau_{X, CW}"]\ar[d, "\cong"] & H^m(X^{CW}, B) \ar[d, "\cong"] \\
        H^n(X,A) \ar[r, "\tau_X"] & H^m(X,B) \\
    \end{tikzcd}\]
    So
    \[\begin{split}
        \mathrm{Nat}_{\mathbf{Top}}(H^n(\_,A), H^m(\_, B)) &\xleftarrow{\cong} \mathrm{Nat}_{Ho(\mathbf{Top})}(H_n(\_,A), H^m(\_,B)) \\
        &\xrightarrow{\cong} \mathrm{Nat}_{Ho(\mathbf{CW})}(H^n(\_,A), H^m(\_, B))
    \end{split}\]
    using weak homotopy equivalences and CW-approximation
\end{proof}

\begin{example}
    \begin{enumerate}
        \item \(K(A,n)\) is simply connected, so \(H^0(K(A,n), B)\cong B\) for \(n\geq 1\) and
        \[H^i(K(A,n), B) = 0 \text{ for } 1\leq i \leq n-1.\]
        so the only operations of type \((A,n, B,0)\) are the constant functions with image in \(B\).

        For \(1\leq i\leq n-1\), \(\Oper(A,n,B,i)= \set b\)
        \item Let \(f\colon A\to B\) be a group homomorphism. This induces a natural group homomorphism
        \item \[f_*\colon H^n(X,A)\to H^n(X,B)\]
        This is a additive cohomology operation for all \(n\geq 0\). So we have
        \[\begin{split}
            \Hom_{grp}(A,B)&\to \Oper(H^n(\_, A), H^n(\_, B))\cong H^n(K(A,n),B) \\
            &\xrightarrow[\text{UCT}]{\cong} \Hom(H_n(K(A,n),\iz), B)\cong_{\text{Hurewicz}} \Hom(\pi_n(K(A,n), *), B) \cong_\phi \Hom(A,B)
        \end{split}\]
        \item Bockstein Homomorphisms associated with a short exact sequence
        \[0\to B \to E\to A\to 0\]
        of abelian groups are cohomology operations \(\beta\colon H^n(X,A)\to H^{n+1}(X,B)\) for all \(n\geq 0\).

        \(\beta\) only depends on the equivalence class of the s.e.s. in \(\Ext(A,B)\). So we get
        \[\Ext(A,B)\to \Oper(A,n, B, n+1) \xrightarrow{\cong} H^{n+1}(K(A,n),B)\]
        where for \(n\geq 2\) the homomorphism from UCT is a isomorphism.

        These are all cohomology operations for \(n\geq 2\).
        \item The group \(H_2(K(A,1), \iz)\) is not generally trivial. For a group \(G\) (not necessarily abelian)
        \[H^2(K(G,1), B) = \text{isomorphism classes of centric extensions, i.e.}\]
        short exact sequences of groups
        \[0\to B E\to G\to 0\]
        such that \(B\) is central in \(E\). If \(G\) is abelian, then
        \[\text{abelian extensions} = \Ext{A,B} \to H^2(K(A,1), B) = \text{Central extensions}\]
        As an exercise we will see a sort of Bockstein that gives a cohomology operation from degree 1 to degree 2.

        So far we have
        \[\Oper(A,n,B,i)\cong \begin{cases}
            B & i = 0 \\
            0 & 1\leq i\leq n-1\\
            \Hom(A,B) & i = n \\
            \Ext(A,b) & i = n+1 \geq 3 \\
        \end{cases}\]
        \item Let \(R\) be a ring, \(k\geq 0\). Then
        \[H^n(X,R)\to H^{kn}(X,R)\quad x\mapsto x^k = x\cup \dots \cup x\]
        is a cohomology operation of type \((R,n,R,kn)\). This is typically not additive.

        For example
        \[H^n(K(\IF_2, 1, \IF_2))\cong\Oper(\IF_2, 1, \IF_2, k)\in \set{x \mapsto x^k}\]
        where this is equal to
        \[H^n(\ir P^\infty, \IF_2)\]
        and
        \[H^*(\ir P^\infty, \IF_2) = \IF_2 [\iota_{\IF_2, 1}]\]
        So \(\Oper(\IF_2, 1\IF_2, k) = \set{0, x\mapsto x^k}\).

        We also calculated \(H^*(\ic P^\infty, \iz)= \iz[\iota_{\iz, 2}]\). So this gives
        \[\Oper(\iz, 2, \iz, k) = \begin{cases}
            0 & k \text{ odd} \\
            \iz \set{x\mapsto x^m} & k = 2m \\
        \end{cases}\]

        \item \(\Oper(\iz, 1, B,n) = ?\) We need to compute the cohomology of \(K(\iz, 1) = S^1\).
        \[H^n(S^1, B) = \begin{cases}
            B & n = 0,1 \\
            0 & \text{else} \\
        \end{cases}\]
        This is now the operations.
    \end{enumerate}

    This completes the operations we can really calculate.
\end{example}

\textbf{Reminder} Let \(R\) be a commutative ring. We discussed
\[\cup_1\colon C^n(X,R)\otimes C^m(X,R) \to C^{n+m-1}(X,R)\]
with coboundary formula \(\delta(f\cup_1 g) = (\delta f)\cup_1 g + (-1)^n f\cup_1 (\delta g) - (-1)^{n+m} f\cup g - (-1)^{(n+1)(m+1)} (g\cup f)\). This was how we proofed commutativity of the cup product. We can use this to produce new cohomology operations.

Suppose \(n\) is even and \(f \in C^n(X,R)\) is a cocycle, \(\delta f = 0\). Then \(\delta(f\cup_1 f) = 0\), and the cohomology class \([f\cup_1 f]\in H^{2n-1}(X,R)\) only depends on the cohomology class of \(f\). Same for \(n\) odd and \(2 = 0\) in \(R\).

So we get cohomology operations \(\Sq_1\colon H^n(X,R)\to H^{2n-1}(X,R)\) for \(n\) even and
\[\Sq_1\colon H^n(X,R)\to H^{2n-1}(X,R/2)\]
for \(n\) odd. They are defined by \(\Sq_1[f]\coloneq [f\cup_1 f]\).

\textbf{Preview.} We will define Stable cohomology operations of type \((A,B,n) = \set{\tau_i\colon H^i(\_, A)\to H^{n+i}(\_, B)}\) and some compatibility with a suspensions isomorphism.

\begin{defi}{Reduced cohomology operation}{}
    A \emph{reduced cohomology operation} of type \((A,n,B,m)\) is a natural transformation \(\tilde H^n(\_,A)\to \tilde H^m(\_, B)\) of set valued functors of based spaces.
\end{defi}

Then \(\mathrm{red}\Oper(A,n,B,m)\cong \tilde H^m(K(A,n), B) = H^m(K(A,n), B) = \Oper(A,n,B,m)\) for \(m \geq 1\). And for \(m = 0\) the contsant operations induced by \(B\setminus\set 0\) vanish.

\begin{construction}
    Let \((X,\phi)\) and \((Y, \psi)\) be EM-spaces of type \((A,n)\) and \((A, n+1)\), respectively for \(n\geq 1\). By an earlier theorem, there is a unique-up-to-based homotopy map \(\rho\colon X\to \Omega Y\), such that \(\rho_*\colon \pi_n(X,*)\to \pi_n(\Omega Y, *)\) agrees with the composite
    \[\pi_n(X,*)\xrightarrow{\phi}A \xrightarrow{\psi^{-1}} \pi_{n+1}( Y, *)\cong \pi_n(\Omega Y,*)\]
\end{construction}

we have \(S^n \wedge S^1 \cong S^{n+1}\) and by understanding \(S^n = \ir^n\cup \set \infty\), this isomorphism is by coordinates.

Further \(\Sigma X = X\wedge S^1\) and \(\Omega X = \map_*(S^1, X)\)

We let \(\varepsilon\colon \Sigma X\to Y\) be the adjoint of \(\rho\).

\begin{lem}{}{}
    Let \((X,\phi)\), \((Y, \psi)\) be EM-spaces of type \((A,n)\) and \((A,n+1)\) respecitvely.
    \begin{enumerate}
        \item The following commutes:
        \[\begin{tikzcd}
            \pi_n(X,*)\ar[r, "\Sigma"]\ar[rrd, "\phi", "\cong"'] & \pi_{n+1}(\Sigma X, *) \ar[r, "\varepsilon"] & \pi_{n+1}(Y, *) \ar[d, "\cong", "\psi"'] \\
            && A
        \end{tikzcd}\]
        \item The fundamental class \(\iota_{A,n}\in H^n(X,A), \iota_{A,n+1}\in H^{n+1}(Y,A)\) fullfil
        \[\varepsilon^*(\iota_{A,n+1}) = \Sigma(\iota_{A,n}) \in H^{n+1}(\Sigma X, A)\]
    \end{enumerate}
\end{lem}

\begin{proof}
    \begin{enumerate}
        \item \[\begin{tikzcd}
            \pi_n(X,*) \ar[dd, bend right=80, "\cong", "\phi"'] \ar[d, "\rho_*"]\ar[r, "\Sigma"] & \pi_{n+1}(\Sigma X, *) \ar[d, "\varepsilon_*"] \\
            \pi_n(\Omega Y, *) \ar[r] & \pi_{n+1}(Y,*)\ar[dl, "\cong", "\psi"'] \\
            A & \\
        \end{tikzcd}\]
        Commutes because \(\varepsilon\) is adjoint to \(\rho\).
        \item Equivalently \(\Sigma^{-1}(\varepsilon^*(\iota_{A,n+1})) = \iota_{A,n}\). We show that \(\Sigma^{-1}(\varepsilon^*(\iota_{Ar n+1}))\)has the property that defines \(\iota_{A,n}\):
        \[\pi_n(X,*) \xrightarrow[{[f]\mapsto f_*[(S^n)]}]{\text{Hurewicz}} H_n(X,\iz)\xrightarrow{\Sigma^{-1}(\varepsilon^*(\iota_{A,n+1}))\cap \_} A\]
        equals \(\phi\). We make our choices \([S^n]\in H_n(S^n, \iz)\) consistent with suspension.

        Then
        \[\begin{split}
            \Sigma^-1(\varepsilon^*(\iota_{A,n+1}))\cap f_*[S^n] &= f^*(\Sigma^{-1}(\varepsilon^*(\iota_{A,n+1})))\cap [S^n] \\
            &= \Sigma^{-1}((\Sigma f)^*(\varepsilon^*(\iota_{A,n+1}))) \cap [S^n] \\
            &= (\Sigma f)^*(\varepsilon^*(\iota_{A,n+1}))\cap \Sigma[S^n] \\
            &= (\varepsilon\circ (\Sigma f))^*(\iota_{A,n+1})\cap[S^{n+1}]\\
            &= \iota_{A,n+1} \cap (\varepsilon\circ (\Sigma f))_*[S^{n+1}] \\
            &= \iota_{A, n+1} \cap \mathrm{Hur}(\varepsilon\circ (\Sigma f)) \\
            &= \psi[e\circ (\Sigma f)] = \phi(f) \\
        \end{split}\]
    \end{enumerate}
\end{proof}

\newLecture{12.11.2025}

Was not there, unfortunately

\newLecture{17.11.2025}

We give some examples of stable cohomology operations

\begin{example}\leavevmode
    \begin{itemize}
        \item There are no stable operations of negative degrees.

        \item \(\mathrm{Stab}(A,B,0) = \lim_{\la} H^n(K(A,n),B) \cong \Hom(A,B)\).
        \item \(\mathrm{Stab}(A,B,1) = \lim_{\la} \Oper(A,i,B,i+1) = \lim_{\la} \Ext(A,B)\)
        \item If \(R\) is a ring, \(x\mapsto x^n = x\cup \dots \cup x\) is usually not additive, so whenever it is not additive, it does \underline{not} extend to a stable operation.
        \item If \(R\) is a \(\IF_p\)-algebra, \(p\) prime, then \(x\mapsto x^p\) is an additive operation. We will see that \(p = 2\), \(x\mapsto x^2\in \Oper(\IF_2, n\IF_2, 2n)\) does extend to a stable operation \(\Sq^n\in \mathrm{Stab}(\IF_2, \IF_2, n)\)
        
        If \(p\) is an odd prime \((x\mapsto x^p)\in \Oper(\IF_p, 2k, \IF_p, 2k\cdot p)\) does extend to a stable operation \(P^k \in \mathrm{Stab}(\IF_p, \IF_p, 2k(p-1))\). We will do this only for \(p=2\) and Schwede will mumble how it is more complicated for other \(p\) but works.
    \end{itemize}
\end{example}


\begin{defi}{Steenrod-Algebra}{}
    The \emph{Steenrod albera} for \(A\) is the graded ring \(\cA(A)\) with
    \[\cA^n(A) = \mathrm{Stab}(A,A,n)\]
    The graded multiplication
    \[\circ\colon \cA^n(A)\times \cA^m(A) \to \cA^{n+m}(A)\]
    is composition of operations.
\end{defi}

\textbf{Remark.} In practice we will mostly have \(A = \IF_p\).

\(\cA^*(A)\) acts tautologically on the \(A\)-cohomology of any space, by \(\tau\in \mathrm{Stab}(A,A,n)\), \(x\in H^m(X,A)\).
\[\tau\cdot x\coloneq \tau_m(x)\in H^{m+n}(X,A)\]

This makes \(H^*(X,A) = \set{H^m(X,A)}_{m\geq 0}\) into a graded left \(\cA^*(A)\)-module.

The suspension isomorphisms \(\Sigma\colon H^*(X,A)\to H^{*+1}(\Sigma X, A) \eqcolon H^*(\Sigma X, A)[1]\) is an isomorphism of left \(\cA^*(A)\)-modules.

\begin{thm}{}{}
    Let \(X\) be an \(n\)-connected based space, \(n\geq 1\), \(\varepsilon\colon \Sigma(\Omega X)\to X\) counit of the adjunction \((\Sigma, \Omega)\). Then for all abelian groups \(B\), the map
    \[\varepsilon^*\colon H^i(X,B)\to H^i(\Sigma(\Omega X), B)\]
    is an isomorphism for \(0\leq i\leq 2n\) and injective for \(i = 2n+1\).
\end{thm}

This gives us that the stable operations are in fact quite the nice inverse limit.

\begin{proof}
    Since \(X\) is \(n\)-connected, \(\Omega X\) is \((n-1)\)-connected. Freudenthal suspension theorem says
    \[\Sigma\colon \pi_i(\Omega X, *) \to \pi_{i+1}(\Sigma (\Omega X), *)\]
    is an isomorphism for \(1\leq i \leq 2n-2\) and surjective for \(i = 2n-1\).
    
    The composite
    \[\pi_i(\Omega X, *) \xrightarrow{\Sigma} \pi_{i+1}(\Sigma \Omega X, *) \xrightarrow{\varepsilon_*} \pi_{i+1}(X,*)\]
    is an isomorphism, so the first map is injective. So \(\Sigma\colon \pi_{i}(\Omega X, *) \to \pi_{i+1}(\Sigma \Omega X, *)\) is bijective for all \(1\leq i \leq 2n-1\). So \(\varepsilon_*\colon \pi_{i+1}(\Sigma \Omega X, *) \to \pi_{i+1}(X,*)\) is bijective for all \(1\leq i\leq 2n-1\) and surjective for \(i = 2n\).

    Set \(j = i+1\).
    \[\varepsilon_*\colon \pi_j(\Sigma\Omega X, *) \to \pi_j(X,*)\]
    is bijective for \(1\leq j \leq 2n\) and surjective for \(j = 2n+1\).

    Relative CW-approximation provides a relative CW-complex \((Z, \Sigma\Omega X)\) and a weak equivalence \(f\colon Z\xrightarrow{\sim} X\) extends \(\varepsilon\) and all relatve celss have dimension \(\geq 2n+2\).

    So \(H^i(Z, \Sigma\Omega X; A) = 0\) for \(i\leq 2n+1\). The long exact cohomology sequence of the pair shows that
    \[H^i(Z,A)\to H^i(\Sigma\Omega X, A)\]
    is an isomorphism for \(i\leq 2n\). And the following is exact
    \[0 \to H^{2n+1}(Z,A) \to H^{2n+1} (\Sigma\Omega X, A) \xrightarrow{\partial} H^{2n+2}(Z, \Sigma\Omega X, A)\]
    so the last map is injective.
\end{proof}

\begin{corollary}
    (\(X = K(A,n+1)\)) Let \(n\geq 1\), \(A,B\) abelian groups, \(\varepsilon\colon \Sigma K(A,n)\to K(A,n+1)\). Then
    \[\varepsilon^*\colon H^i(K(A,n+1),B)\to H^i(\Sigma K(A,n),B)\]
    is bijective for \(i\leq 2n\) and injective for \(i = 2n+1\).
\end{corollary}

\begin{corollary}
    \[\mathrm{Stab}(A,B,n) \cong \lim_{\la} \]
    \[H^{n+i+1}(K(A,i+1), B)\xrightarrow{\cong} H^{n+i}(K(A,n),B) \xrightarrow{\cong}\dots\xrightarrow{\cong} H^{2n+1}(K(A,n+1),B)\hookrightarrow H^{2n+1}(K(A,n+1), B)\to \dots \]
    So
    \[\mathrm{Stab}(A,B,n)\xrightarrow[\text{isomorphism}]{\cong} H^{2n+1}(K(A,n+1), B), \quad \tau\mapsto \tau_{n+1}(\iota_{A,n+1})\]
    and
    \[\mathrm{Stab}(A,B,n)\xhookrightarrow{\text{injective}} H^{2n}(K(A,n), B), \tau\mapsto \tau_n(\iota_{A,n})\]
\end{corollary}

Since \(\tau_n\) is additive operation, the class \(u = \tau_n(\iota_{A,n}) \in H^{2n}(K(A,n), B)\) satisfies \(\mu^*(u) = p_1^*(u)+ p_2^*(u)\).

So \(\mathrm{Stab}(A,B,n)\hookrightarrow \set{u \in H^2n(K(A,n),B) | \mu^*(u) = p_1^*(u) + p_2^*(u)}\).

\begin{thm}{}{}
    The map \(\mathrm{Stab}(A,B,n)\to \set{u \in H^{2n}(K(A,n), B) | \mu^*(u) = p_1^*(u)+ p_2^*(u)}\) given by \(\tau =\set{\tau_i}_{i\geq 0} \tau_n(\iota_{A,n})\) is an isomorphism
\end{thm}

Equivalently we could say the map
\[\mathrm{Stab}(A,B,n)\to \Oper^{\text{add}}(A,n,B,2n), \quad \tau\mapsto \tau_n\]
is an isomorphism

To proof this we would need a loooot of simplicial homotopy theory which we did not do. We will use this fact to construct Steenrod squares anyways. Later on, Schwede will show a construction not relying on this fact.

\textbf{Remark.} If \(A = B= R\) is a ring, then
\[\times \colon H^k(X,R)\times H^m(Y, R) \mapsto H^{k+m}(X\times Y, R), \quad x \times y = p_1^*(x)\cup p_2^*(y)\]
Then \(\mu^*(u) = p_1^*(u) + p_2^*(u)\) is equivalent to \(\mu^*(u) = u\times 1+ 1\times u\), i.e. it is \enquote{primitive}.

\begin{example}
    Take \(A = B = \IF_2\), \(x\mapsto x^2 \in \Oper(\IF_2, n, \IF_2, 2n)\) and we have
    \[(x+y) \cup (x+ y) = x\cup x + x\cup y + y \cup x + y\cup y = x^2 + y^2\]
    so it is additive. Hence we have a unique stable operation \(\Sq^n = \set{\Sq^n_i}_{i\geq 0}\in \mathrm{Stab}(\IF_2, \IF_2, n) = \cA^n(\IF_2) = \cA^n_2\) the mod 2 Steenrod algebra.

    This operation is uniqely characterised yb the property
    \[\Sq^n(x) = x^2\]
    for all \(x \in H^n(X, \IF_2)\). We want to see how this operation looks in other degrees. We can only do this for \(\IF_2\) coefficients, constructing the Steenrod squares in this way. For general \(\IF_p\) we will need to use another strategy.
\end{example}

\begin{example}
    For \(n = 0\) we have \(\iota_0^2 = \iota_0\) where \(\iota_0 \in H^0(K(\IF_2, 0), \IF_2) = H^0(\IF_2, \IF_2)\) and \(\iota_0\) corresponds to \(\id_{\IF_2}\). So we get \(\Sq^0 = \id\).

    For \(n = 1\), we have seen earlier in an exercise, for all \(x \in H^1(X,\IF_2), v^2 = \beta(x)\) where \(\beta\) is the Bockstein of the short exact sequence
    \[0 \to \IF_2 \to \iz_4 \to \IF_2 \to 0\]
    where we have \(\beta\colon H^n(X,\IF_2)\to H^{n+1}(X, \IF_2)\)a stable operation. We get \(\Sq^1= \beta\).
\end{example}

\begin{thm}{Properties of Steenrod squares}{}
    For each \(i\geq 0\), there is a unique stable \(\mod -2\) cohomology operation \(\Sq^i\), such that \(\Sq^i(x) = x^2\) for all \(x\in H^i(X,\IF_2)\). Moreover, the following properties hold
    \begin{enumerate}
        \item \(\Sq^0 = \id, \Sq^1 = \beta\)
        \item (Unstability) For all \(x\in H^n(X, \IF_2)\) and \(i> n\), \(\Sq^i(x) = 0\)
        \item (Cartan Formula) \(x,y \in H^*(X, \IF_2)\) homogenous
        \[\Sq^i(x\cup y) = \sum_{a+b = i}\Sq^a(x)\cup \Sq^b(y)\]
    \end{enumerate}
\end{thm}

\begin{proof}
    \begin{enumerate}
        \item Was the above example
        \item Consider the iterated suspension isomorphism \(i> n\)
        \[\Sigma^{i-n}\colon H^n(X, \IF_2) \to H^i(X, \IF_2)\]
        \[\Sigma^{i-n}(\Sq^i(x)) = \Sq^i(\underbrace{\Sigma^{i-n}(x)}_{i\text{-dim}}) = (\Sigma^{i-n}(x))\cup (\Sigma^{i-n}(x)) = 0\]
        as cup products vanish in suspensions. Since \(\Sigma^{i-n}\) is an isomorphism, \(\Sq_i(x)= 0\)
        \item Will be deduced from
        \begin{proposition}[External Catan Formula]
            For all spaces \(X,Y\) and all \(x uin H^n(X, \IF_2)\), \(y\in H^m(Y, \IF_2)\), \(i\geq 0\):
            \[\Sq^i(x\times y) = \sum_{a+b = i} \Sq^a(x)\times \Sq^b(y)\]
        \end{proposition}
        The external catan implies the internal catan by
        \[x\cup y = \Delta^*(x/times y), \quad \Delta\colon X\to X\times X\text{ diagonal}\]
        for \(x,y \in H^*(X)\). Use external form with \(\Delta^*\colon H^*(X\times X, \IF_2)\to H^*(X, \IF_2)\).

        \begin{proof}
            If \(i> m+n\) both sides of the desired equation are 0 by instability. For \(i = n+m\)
            \[\Sq^{n+m}(x\times y) = (x\times y)\cup (x\times y) = (x\times x)\cup (y\times y) = x^2 \times y^2 = \Sq^m(x)\times \Sq^n(y) = \sum_{a+b=n+m} \Sq^a(x) \times \Sq^b(y)\]
            again using instability on most of the sum-terms. For \(v>n+m\) Induction on \(n+m\). For \(n+m = 0\), \(n= m=0\), so \(\Sq^0(x\times y) = x\times y = \Sq^0(x)\times \Sq^0(y)\).

            Now let \(n+m \geq 1\), \(i<n+m\). \(X= K(n) = K(\IF_2, n)\), \(K(m) = K(\IF_2, m) = Y\), \(x = \iota_n\), \(y = \iota_m\). For \(p \leq 2n-1\):
            \[\varepsilon^*\colon H^p(K(n), \IF_2) \to H^p(\Sigma H(n-1), \IF^2)\]
            is injective for \(q \leq 2m -1\)
            \[\varepsilon^*\colon H^q(K(m), \IF_2)\to H^q(\Sigma K(m-1), \IF_2)\]
            is injective and
            \[H^k(K(n)\times K(m), \IF_2)\xrightarrow[\text{Künneth theorem}]{\cong}\bigoplus_{p+q = k} H^p(K(n), \IF_2)\otimes H^q(K(n), \IF_2)\]
            so
            \[(\varepsilon^*\otimes 1, 1\otimes \varepsilon^*)\hookrightarrow \bigoplus_{p+q = k} H^p(\Sigma K(n-1), \IF_2)\otimes H^q(K(m), \IF_2)\oplus H^p(K(n), \IF^2)\otimes H^q(\Sigma K(n-1), \IF_2)\]
            is injective for all \(k\leq 2n+2m-1\).

            \(\Sq^i(x\times y)\in H^{n+m+i}(K(n)\times K(m), \IF_2)\). To prove the desired relation, it suffices to prove it after applying \(\varepsilon^*\otimes 1\) and \(1\otimes \varepsilon^*\).
            \[\begin{split}
                (\varepsilon\times 1)^*(\Sq^i(\iota_n\times \iota_m)) &= \Sq^i((\varepsilon\times 1)^*(\iota_n\times \iota_m)) \\
                &= \Sq^i(\varepsilon^*(\iota_n)\times \iota_m) \\
                &= \Sq^i(\Sigma (\iota_{n-1})\times \iota_m) \\
                &= \Sigma(\Sq^i(\iota_{n-1}\times \iota_m)) \\
                &\underset{\text{induction}}{=} \Sigma \sum_{a+b = i}\Sq^a(\iota_{n-1})\times \Sq^b(\iota_m) \\
                &= \sum_{a+b = i}\Sq^a(\Sigma (\iota_{n-1})\times \Sq^b(\iota_m)) \\
                &= \sum_{a+b = i} \Sq_a(\varepsilon^*(\iota_n))\times \Sq^b(\iota_m) \\
                &= (\varepsilon\times 1)^* (\sum_{a+b= i} \Sq^a(\iota_n)\times \Sq^b(\iota_m)). 
            \end{split}\]
            This proofs external catan in the universal example \(X = K(n), Y = K(m)\), \(x = \iota_n, y = \iota_m\). For CW-complexes it follows by representability. In general by CW-approximation.
        \end{proof}
    \end{enumerate}
\end{proof}

\newLecture{19.11.2025}

Some standard applications of Steenrod squares:

\textbf{Reminder.} Let \(f\colon X \to Y\) be a continuous map. We want to understand when \(f\) is essential, i.e. not homotopic to a constant map.
\begin{itemize}
    \item if \(f\) induces a non-zero map on \(\pi_n, H_n(\_, A), H^m(\_, B)\), then \(f\) is essential.
    \item if \(f\) is nullhomotopic, the s.e.s. for
    \[C(f)\coloneq X\times [0,1] \cup_f Y, \; i\colon Y \to C(f)\]
    yields
    \[H^{n+1}(X,A) \xrightarrow{ } H^n(Cf, A)\xrightarrow{i^*} H^n(Y,A)\xrightarrow{f^*} H^n(X,A)\]
    If \(f^* = 0\), \(H^*(Y,A) \to H^*(X,A)\) also gives a s.e.s. If \(f\) is nullhomotopic, then by a choice of nullhomotopy \(H\colon X\times [0,1]\to Y\) from a constant map to \(f\) and we get
    \[\sigma = H\cup \id_Y \colon C(Y)\to Y\]
    with \(\sigma\circ i =\id_Y\) So \(i\colon Y\to C(f)\) has a retraction, which induces a map in cohomology.
    \[0\to H^{k+1}(X,a)\xrightarrow{\partial} H^k(Cf, A)\xrightarrow{i^*} H^k(Y,A) \to 0\]
    where \(\sigma^*\) gives a section to \(i^*\).

    Our strategy here is to endow cohomology with more natural structure to show there is no algebraic map that respects the structure and is a section to \(i^*\).

    The Cup-product: \(f = \eta\colon S^3\to S^2\) the Hopf fibration has \(C(f) \cong \IC P^2\).
    \[H^*(C(f),\iz) = H^*(\ic P^2, \iz)\cong \iz[x]/(x^3)\]
    and
    \[\iz[x]/(x^3) = H^*(\ic P^2, \iz) \xrightarrow{i^*} H^*(\ic P^1, \iz) = \iz[x]/(x^2)\]
    has additive sections, but \emph{no} multiplicative sections. Hence there is no continuous retration \(\iota\colon \ic P^1 \hookrightarrow \ic P^2\).

    And hence we get \(\eta\) is essential. \(\pi_3(S^2, *) \cong \iz\set \eta\).

    We can do the same for \(S(\ih^2) = \nu\colon S^7 \to S^4 = \ih P^1\). Hence \(\nu\) is essential and we had \(\pi_7(S^4) \cong \iz\set{\nu} \oplus \) a finite group Schwede forgot.

    And once more the same for \(\sigma\colon S^{15} = S(\io ^2)\to \io \cup \set\infty \cong S^8\). We define \(C(\sigma) \eqcolon \io P^2\) and \(H^*(\io P^2, \iz)\cong \iz[z]/(\iz^3)\). We see \(\sigma\) is essential, \(\pi_15(S^8)\cong \iz\set\sigma \oplus\) some finite group.\footnote{some remark about how we can't proof that \(\sigma\) has infinite order.}

    \item We want to see that \(\eta\) is stably essential, i.e. also suspensions of \(\eta\) are essential. Same for \(\nu, \sigma\). We saw \(2\cdot (\Sigma \eta) = 0\) in \(\pi_4(S^3)\). And similarly also
    \[24 \cdot (\Sigma\nu) = 0 \in \pi_8(S^5) \quad 240 \cdot (\Sigma\sigma) \in \pi_{16}(S^9)\]
\end{itemize}

\begin{defi}{stably essential maps}{}
The map \(f\colon X\to Y\) of based spaces is \emph{stably essential} if \(\Sigma^nf\colon \Sigma^n X\to \Sigma^n Y\) is not homotopic to a constant map for all \(n\geq 0\).
\end{defi}

\textbf{Note.} Cohomology and cup product cannot show that \(\Sigma^nf\) is essential for \(n\geq 1\) because the cup product on the cohomology of \(\Sigma Y\) is trivial!

\(\eta\colon S^3\to S^2\) is stably essential, i.e. \(\Sigma^n \eta \not\simeq *\) for all \(n\geq 0\). We want to show this using cohomology operations.

\begin{corollary}
    \(\pi_1^{\mathrm{st}} = \iz_2\set\eta\).
    \[\Sigma^n\eta \colon S^{n+3}\to S^{n+2}\to \Sigma^n C(\eta) \to S^{n+4}\]
    where we write \(\mathrm{Cone}(\Sigma \eta) \cong \Sigma^n(\eta)\).

    \[0\to \tilde H^*(S^{n+4}, \IF_2)\to \tilde H^*(\Sigma^n(C(\eta)), \IF_2) \to \tilde H^*(S^{n+2}, \IF_2)\to 0\]
\end{corollary}

Schwede explains how to draw cohomology to see there is no section to \((\Sigma^{n_i})^*\) compatible with Steenrod operation \(\Sq^2\).

\(H^*(X,\IF_2)\) is a graded module over \(\cA^*(\IF_2)\). Teh s.e.s. descends split as modules over \(\cA^*(\IF_2)\). We have \(C(\eta)\cong \ic P^2\). \(H^*(C(\eta), \IF^2) = H^*(\IC P^2, \IF^2) = \IF[x]/(x^3)\) and we have \(\Sq^2(x) = x^2 \neq 0\).

So as the steenrod squares are stable cohomology operations, they are still non-zero after suspending, while the cup-product is gone.
\[\tilde H^k(\Sigma^n C(\eta), \IF_2) \cong \begin{cases}
    \IF_2\set{\Sigma^n x} & k = n+2 \\
    \IF_2\set{\Sigma^n(x^2)} & k = n+4 \\
    0 & \text{else}
\end{cases}\]
and
\[\Sq^2(\Sigma^n(x)) = \Sigma^n(X^2)\]

Schwede rambles about how there are secondary cohomology operations and he will not talk about that.

For \(\nu\) we have \(C(\nu)\cong \ih P^2\) and \(H^*(\ih P^2, \IF_2)\cong \IF_2[y]/(y^3)\) for \(y \in H^4(\ih P^2, \IF_2)\). He again draws the cohomology and concludes \(\Sigma ^n(\nu) \not \simeq 0\) for all \(n\geq 0\) using \(\Sq^4\).

For \(C\sigma \cong \io P^2\) we use \(\Sq^8\) to obtain the same result.

Schwede explains that \(\Sq^n\) is decomposable for \(n \neq 2^i\), i.e.
\[\Sq^n = \sum_{a+b = n} \Sq^a \circ \Sq^b\]
where \(\circ\) is composition of operations. This is just motivation we will do this later.

We give one more application of the Steenrod squares. \footnote{Schwede does something for the people in his seminar and tells the rest of us to ignore.}

\subsection{Moore Spaces}

\begin{defi}{Moore Space}{}
The mod \(n\) moore space is
\[M(n) = \mathrm{Cone}(S^1\xrightarrow{\cdot n} S^1)\]
\end{defi}

For example \(M(2)\cong \IR P^2\). And we get
\[\tilde H_k(M(n), \iz) = \begin{cases}
    \iz/n & k = 1 \\
    0 & k \neq 1 \\
\end{cases}\]

\textbf{Fact.} \(\times_n\colon S^1 \wedge M(n)\xrightarrow{\deg(n)\wedge \id_{M(n)}} S^1 \wedge M(n)\).

If \(p\) is odd then \(\times_p\) is stably nullhomotopy on \(M(p)\). But
\[\times_2\colon S^1\wedge M(2) \to S^1\wedge M(2)\]
is stably essential. schwede draws us a picture to proof this. You could read it in his script. I did not copy this part.

\newLecture{24.11.2025}


We want to now construct the Steenrod-squares. We want to construct the total power operation:

\(p\) prime. \(\cP_p\colon H^n(X, \IF_p) uto H^{np}(X\times L(p), \IF_2)\) where \(L(p) = \) infinite dimensional lense space \(= S^\infty/{C_p}\) as generalizations of \(L(2) = \IR P^\infty\). By Künneth theorem we will have
\[H^*(X\times L(p), \IF_p) \cong H^*(X, \IF_p)\otimes_{\IF_p} H^*(L(p), \IF_p)\]
where the second part is 1-dimensional in every degree.

For \(p= 2\), \(H^*(L(2),\IF_2) = H^*(\IR P^\infty, \IF_2) = \IF_2[u]\) for \(u \in H^1(\IR P^\infty, \IF_2)\). \(\cP_2(X) = \sum_{i\geq 0}\Sq^i(x)\times u^{n-i}\). We check that all degrees match up.

And now in more detail:

\begin{defi}{}{}
    We write \(S^\infty = \bigcup_{n\geq 0} S(\ic^n)\) for the infinite dimensional sphere,
    where \(S(\ic^n) = \set{z\in \ic^n | \abs z = 1}\). We write
    \[C_p = \set{z\in \ic | z^n = 1}\]
    for the \(p\)-th roots of unity. \(C_p\) acts freely on \(S^\infty\) by scalar multiplication. We set
    \[L(p) = S^\infty/C_p = S^\infty/(v\sim \zeta_p v)\]
    where \(\zeta_p  = e^{2\pi i/p}\) product with \(p\)-th root of 1.
\end{defi}

For \(p = 2\), we have \(C_2 = \set{\pm 1}\), this is the antipodal action on \(S^\infty\). 

Since the \(C_p\)-action on \(S^\infty\) is free and \(S^\infty\) is contractible, the quotient map
\[S^\infty \to S^\infty/C_p = L(p)\]
is the universal covering. So \(L(p)\) is an Eilenberg-Maclane space of type \(K(C_p, 1)\).

\(S^\infty\) has a CW-structure with \(S^\infty_{2k-1} = S(\ic^k)\) and \(S^\infty_{2k} = \) join in \(S(\IC^{k+1})\) of \(S^\infty_{2k-1} = S(\ic^k \oplus 0)\) and \(\set{(0, \dots, 0, z)| z\in C_p}\). He draws a picture on how this works.

This is a CW-structure on \(S^\infty\) with \(p\) cells in each dimension; \(C_p\) acts cellularly and free permuting the \(k\)-cells for all \(k\).

As \(L(p)\) is a K(p,1) we get
\[C_p\cong \pi_1(L(p), *) \cong H_1(L(p), \iz)\]
By UCT we get
\[H^1(L(p), \IF_p)\cong \Hom(H_1(L(p),\iz), \IF_p) \cong \hom(C_p, \IF_p)\]
mapping some \(x\) to \(\zeta_p \mapsto 1\). We set \(y \coloneq \beta(x)\in H^2(L(p), \IF_p)\). Where \(\beta\) = Bockstein (for \(0\to \IF_p\to \iz/_{p^2}\to \IF_p\to 0\)).

If \(p = 2\), then \(y = \beta(x) = x^2\). If \(p\geq 3\), then \(x^2 = 0\).

\begin{thm}{}{}
    Let \(p\) be an odd prime. Then
    \[H^*(L(p), \IF_p) = \IF_p[x,y]/(x^2) = \IF_p[y]\otimes \Lambda(x)\]
\end{thm}

\begin{proof}
    \[H^k(L(p), \IF_p) \cong \begin{cases}
        \IF_p\set{y^{k/2}} & k \text{ even} \\
        \IF_p\set{x\cdot y^{(k-1)/2}} & k \text{ odd} \\
    \end{cases}\]

    The \(C_p\)-action on \(S^\infty\) is cellular for the CW-structure, so the induced action on \(C_*^{cell}(S^\infty)\) makes theis into a complex of modules over \(\iz[C_p]\). This makes \(C^{cell}_k(S^\infty)\) a free \(\iz[C_p]\)-module of rank 1.

    choose characteristic maps for all of the \(k\)-cells; chooste the other characeristic maps by following with the action of \(C_p\). We get \(e_k^0, e_k^1 = \zeta_p\cdot e^0_k, \dots, e_k^i, \dots, e_k^{p-1}\).

    \textbf{Claim.} 
    \[\partial^{cell}(e_k^0) = \begin{cases}
        e_{k-1}^0 - e_{k-1}^1 & k \text{ odd} \\
        e_k^0 + e_k^1 + \dots + e_k^{p-1} & k \text{ even} \\
    \end{cases}\]
    We again draw pictures to proof this claim. And some notes so scribbled I didn't follow. Now
    \[C^{cell}_*(L(p)) = C^{cell}_*(S^\infty/C_p) = C^{cell}_*(S^\infty)\otimes_{\iz[C_p]}\iz\]
    He \enquote{hits the picture with that tensor.} We get that the connecting maps alternatingly are \(0\) and \(p\).
    So
    \[H_*(L(p), \IF_p), H^*(L(p), \IF_p)\]
    is \(1\)-dimensional in every degree.
    
    For the multiplicative structure We show by induction on \(k\), that
    \[H^*(L(p)_{2k-1}, \IF_p) = \IF_p[x,y]/(x^2, y^k)\]

    We have \(L(p)_{2k-1}= S(\ic^k)/{C_p}\) with \(C_p\)-action free and orientation preserving. So \(L(p)_{2k-1}\) (the Lense space) is an orientable, connected, compact (smooth) manifold. So
    \[H^*(L(p)_{2k-1}, \IF_p)\]
    satisfies Poincare duality. So
    \[\cup\colon H^i(L(p)_{2k-1}, \IF_p)\times H^{2k-1-i}(L(p), \IF_p)\to H^{2k-1}(L(p), \IF_p)\]
    is a perfect pairing. So
    \[\cup y\colon H^{2k-3}(L(p)_{2k-1}, \IF_p)\to H^{2k-1}(L(p)_{2k-1}, \IF_p)\]
    is an isomorphism. However the first term is also \(H^{2k-3}(L(p)_{2k-3}, \IF_p) = \IF_p\set{x\cdot y^{k-2}}\). So \(y^{k-1} \neq 0\). So \(y^{k-1}\) generates \(H^{2k-2}(L(p)_{2k-1}, \IF_p)\)
\end{proof}

\begin{defi}{Extended power Construction}{}
    the \(p\)-th extended power of a space \(X\) is the space
    \[D_p(X) = X^p\times_{C_p} S^\infty= (X^p\times S^\infty)/(x_1,\dots, x_p, v)\sim (x_2, \dots, x_p, x_1, \zeta_p\cdot v)\]
    If \(X\) is based, the reduced \(p\)-th extended power is
    \[\tilde D_p(X) = X^{\wedge p}\wedge_{C_p} S^\infty_+\]
    \(D_p(X)\) has different notation: \(X^p\times _{C_p} EC_p\) for some \(EC_p\) a free contractible \(C_p\) space.

    \((X^p)_{hC_p}\) the homotopy orbit space.
\end{defi}

We have \(\tilde D_p(X) = \frac{D_p(X)}{(\mathrm{FatWedge}_{C_p})\times S^\infty}\). For \(Y\) unbased \(\tilde D_p(Y_+) \cong D_p(Y)_+\).

\begin{proposition}
    Let \(Y\) be a based \((n-1)\)-connected CW-complex equipped with a continuous based \(C_p\)-action. Then the space \(Y\wedge_{C_p}S^\infty_+\) is \((n-1)\)-connected. Moreover, the map
    \[j\colon Y\to Y\wedge_{C_p}S^\infty_+ \quad y\mapsto [y\wedge (1,0\dots)]\]
    induces an isomorphism \(j_*\colon H_n(Y, A)/C_p\to H_n(Y\wedge_{C_p}S^\infty_+, A)\)
    where the left hand side is a abelian group quotient. And
    \[j^*\colon H^n(Y\wedge_{C_p} S^\infty_+, A)\to H^n(Y, A)^{C_p}\]
    is also an isomorphism, where \(H^n(Y,A)^{C_p}\) denotes fixpoints under the action of \(C_p\).
\end{proposition}

\begin{proof}
    The subquotients of the equivariant skeleton filtration on \(S^\infty\) are \(S^\infty_k/S^\infty_{k-1} \cong (C_p)_+ \wedge S^k\) where the isomorphism is \(C_p\)-equivariant.

    This induces a filtration on \(Y\wedge_{C_p}S^\infty_+\) by \(\set{Y\cap_{C_p} S^\infty_k}_{k\geq 0}\). with subquotients
    \[\frac{Y\wedge_{C_p}(K^\infty_k)+}{Y\wedge_{C_p}(S^\infty_{k-1})_+} \cong Y\wedge_{C_p} ((C_p)_+\wedge S^k) \cong Y\wedge S^k\]
    with the end result being \((n+k-1)\)-connected. So
    \[Y\wedge_{C_p}(S^\infty_1)_+ \to Y\wedge_{C_p}S^\infty_+\]
    induces isomorphisms on \(H_n(\_, A)\) and \(H^n\_, A\). Hence
    \[H_n(Y\wedge_{C_p} S(\IC)_+, A)\to H_n(Y\wedge_{C_p}S^0_+, A)\]
    is an isomorphism.

    We get a cofibre sequence of \(C_p\)-spaces.
    \[C_p = \set{z\in \ic | z^p = 1} \to S(\ic) \to (C_p)_+ \wedge S^1\]
    and we can apply \(Y\wedge_{C_p}\_\) to it. Missed what that looked like. We get
    \[H_{n+1}(Y\wedge S^1, A) \to H_n(Y,A)\to H_n(Y\wedge_{C_p}S(\ic)_+, A) \to 0\]
    where the first term is isomorphic to \(H_n(Y,A)\) and we want to undertsand the first map acting on homology. i missed a bit.

    Since \(Y\) is \(n-1\)-connected, UCT gives
    \[H^n(Y, A) \xrightarrow{\cong} \hom(H_n(Y, \iz), A)\]
    since the map in the UCT is natural, this isomorphism is \(C_p\)-equivariant. So it retracts to an isomorphis of \(C_p\)-fixed points.
    \[H^n(Y,A )^{C_p}\xrightarrow{\cong} \Hom(H_n(Y,\iz), A)^{C_p} \cong \Hom(H_n(Y,\iz)/C_p, A)\]
    Since \(Y\wedge_{C_p}S^\infty_+\) is \((n-1)\)-connected, so UCT gives
    \[H^n(Y\wedge_{C_p}S^\infty_+, A)\xrightarrow{\cong} \hom(H_n(Y \wedge_{C_p}S^\infty_+, \iz), A)\]
    These data participate in a commutative diagramm I didn't copy.
\end{proof}

\newLecture{26.11.2025}

We applly this proposition to \(Y = X^{\wedge p}\), for \(X\) a based space.

\begin{proposition}
    Let \(p\) be a prime, \(n\geq 1\).
    \begin{enumerate}
        \item For every based space \(X\) and every \(x \in \tilde H^n(X,\IF_p)\) the class
        \[x\wedge \dots \wedge x \in \tilde H^{np}(X^{\wedge p}, \IF_p)\]
        is invariant under the \(C_p\)-action.
        \item There is a unique class \(\tilde\iota_{n,p} \in \tilde H^{np}(\tilde D_p(K(\IF_p,n), \IF_p))\),\footnote{Schwede remarks how this might look random to us right now. Can confirm.} s.t.
        \[j^*(\tilde\iota_n, p) = \iota\wedge \dots \wedge \iota \text{ p-times } \in H^{np}(K(\IF_p,n)^{\wedge p}, \IF_p)\]
    \end{enumerate}
\end{proposition}

\begin{proof}
    \begin{enumerate}
        \item Recall that for based spaces \(X,Y\) and \(x \in \tilde H^k(X,\IF_p), y \in \tilde H^l(Y, \IF_p)\) we have
        \[x\wedge y = (-1)^{kl}\tau_{x,y}^* y\wedge x\]
        where \(\tau_{X,Y}\colon X\wedge Y \xrightarrow{\cong} Y\wedge X\) is switching around.

        Let \(m\geq 2\), \(X_1, \dots, X_n\) based spaces
        \[c_m\colon X_1\wedge \dots \wedge X_m \xrightarrow{\cong} X_2\wedge \dots \wedge X_m\wedge X_1\]
        the cyclic permutation of factors, \(x_i\in H^{k_i}(X_i, \IF_p)\).
        \[\begin{split}
            x_1\wedge \dots \wedge X_m = (-1)^{k_1 \cdot (k_2+k_3+\dots + k_m)} c_m^*(x_2\wedge \dots \wedge x_m \wedge x_1)
        \end{split}\]
        For \(m=2\) this is true. For \(m\geq 3\)
        \[\begin{tikzcd}
            X_1\wedge\dots\wedge X_m \ar[d, "c_{m-1}\wedge X_m"]\ar[r] & X_2\wedge\dots\wedge X_m \wedge X_1 \\
            X_2\wedge\dots\wedge X_{m-1}\wedge X_1\wedge X_m\ar[ru, "X_2\wedge\dots \wedge X_{m-1}\wedge \tau_{X_1, X_m}"] & \\
        \end{tikzcd}\]
        \[c_m^*(x_2\wedge \dots\wedge x_m\wedge x_1) = (c_{m-1}\wedge X_m)^*(X_2\wedge \dots\wedge X_{m-1}\wedge \tau_{X_1, X_m})^*(x_2\wedge \dots\wedge x_m\wedge x_1) =\]
        and this is where he continued with the next part. A lot of this calculation was missing.

        Now specializing \(m = p\) prime, \(X_1= \dots = X_m = X, x_1 = x_2 = \dots = x\), \(k_1 = \dots = k_p = m\) then
        \[c_p^*(x\wedge \dots \wedge x) = (-1)^{n\cdot (n+n\dots +n)} x\wedge\dots \wedge x = (-1)^{n^2 \cdot (p-1)} x\wedge \dots \wedge = x\wedge \dots \wedge x\]

        \item We use \(K(\IF_p,n)\) is \(n\)-connected. Hence \(K(\IF_p, n)^{\wedge p}\) is \(np -1\)-connected, with \(C_p\)-action by cyclic permutation. So
        \[j^*\colon H^{np}(\tilde D_p(K(\IF_p, n)), \IF_p) =H^{np}(K(\IF_p,n)^{\wedge p}\wedge_{C_p}S^\infty_+, \IF_p) \to H^{np}(K(\IF_p, n)^{\wedge p}, \IF_p)^{C_p}\]
        by
        \[\tilde\iota_{n,p}\mapsto \iota\wedge \dots \wedge \iota\]
        which exists uniquely.
    \end{enumerate}
\end{proof}

We let
\[\pi\colon D_p(K(\IF_p, n)) \to \tilde D_p(K(\IF_p, n))\]
be the projection. Set
\[\iota_{n,p} \coloneq \pi^*(\tilde \iota_{n,p}) \in H^{np}(D_p(K(\IF_p, n)), \IF_p)\]
The following commutes
\[\begin{tikzcd}
    K(\IF_p, n)^{\times p} \ar[d, "j"]\ar[r, "\pi"] & K(\IF_p, n)^{\wedge p} \ar[d, "j"] \\
    D_p(K(\IF_p, n)) \ar[r, "\pi"] &\tilde D_p(K(\IF_p, n)) \\
\end{tikzcd}\]
Then
\[j^*(\iota_{n,p}) = j^*(\pi^*(\tilde \iota_{n,p})) = \pi^*(j^*(\tilde\iota _{n,p})) F \pi^*(\iota\wedge \dots \wedge \iota) = \iota\times \dots \times \iota\]


Let \(X\) be a CW-complex. The diagonal map \(\Delta\colon X\to X^p\) is \(\Delta(x) = (x,\dots, x)\). This is \(C_p\)-equivariant for the trivial action on the source and the cyclic permutation action on target.
\[\_ \times_{C_p}S^\infty \colon (X\times_{C_p}S^\infty) \xrightarrow{\Delta\times_{C_p} S^\infty} X^p\times_{C_p} D_p(X)\]
where the first term is also \(X\times L(p)\) and we call this \(\Delta_x\) sending \(x, [v]\mapsto [x,x,\dots, x, v]\)

\begin{defi}{Power operation}{}
    The \(p\)-th power operation \(P_p\colon H^n(X,\IF_p)\to H^{np}(X\times L(p), \IF_p)\) is defined by
    \[x = f^*(\iota) = \Delta_X^*(D_p(f)^*(\iota_{n,p}))\]
\end{defi}

In more detail \(f\colon X\to K(\IF_p, n)\) is continuous, such that \(f^*(\iota) = x\), where \(\iota = \iota_{\IF_p,n}\in H^n(K(\IF_p, n), \IF_p)\).

\[\begin{tikzcd}
    X\times L(p) \ar[r] & D_p(X)\ar[r, "D_p(f)"]& D_p(K(\IF_p, n)) \\
    P_p(x) & D_p(f)^*(\iota_{n,p}) \ar[l, mapsto] & \iota_{n,p} \ar[l, mapsto] \\
\end{tikzcd}\]
so \(P_p(\iota) = \Delta_{K(\IF_p, n)}(\iota_{n,p})\)

Let \(j\colon X \to X\times L(p)\) be the map \(j(x) = (x, [1, 0\dots, 0])\). This new \(j\) fullfills \(\Delta_X \circ j = j\) for respective \(j\)

\begin{proposition}
    \begin{enumerate}
        \item The composite
        \[H^n(X, \IF_p) \xrightarrow{P_p} H^{np}(X\times L(p); \IF_p) \xrightarrow{j^*} H^{np}(X; \IF_p)\]
        raises a class to its \(p\)-th power.
        \item The exterior product interacts with the total power operation in the following way: \(x \in H^n(X, \IF_p), y\in H^m(Y, \IF_p)\)
        \[P_p(x\times y) =  \Delta^*(P_p(x)\times P_p(y))\]
        in \(H^{(n+m)\cdot p}(X\times Y\times L(p), \IF_p)\), \(\Delta\colon X\times Y \times L(p) \to X\times L(p)\times Y \times L(p) \), \((x,y, [v]) \mapsto (x, [v], y, [v])\).
    \end{enumerate}
\end{proposition}

This is what gives rise to the cartan-formula.

\begin{proof}
    \begin{enumerate}
        \item By naturality it suffices to show this relation for the universal example \(X f K(\IF_p, n), x = \iota = \iota_{\IF_p, n} \in H^n(K(\IF_p, n), \IF_p)\). We get a commutative square
        \[\begin{tikzcd}
            K(\IF_p,n) \ar[d, "j"] \ar[r, "\Delta"] & K(\IF_p, n)^p \ar[d, "j"] \\
            K(\IF_p, n)\times L(p) \ar[r, "\Delta_{K(\IF_p,n)}"] & K(\IF_p,n)^p\times_{C_p}S^\infty = D_p(K(\IF_p, n))
        \end{tikzcd}\]
        we observe
        \[j^*(P_p(i)) = j^*(\Delta_{K(\IF_p, n)}^*(\iota_{n,p})) = \Delta^*(j^*(\iota_{n,p})) = \Delta^*(\iota\times \dots \times \iota) = \iota\cup\dots \cup \iota = \iota^p\]
        \item By naturality it suffices to show  the universal case \(X = K(\IF_p, n) = K(n)\) and \(Y = K(\IF_p,m) = K(m)\), \(x = \iota_n = \iota_{\IF_p, n} \in H^n(K(n), \IF_p)\), \(y = \iota_m \in H^m(K(m), \IF_p)\).
        Wen need the reduced diagonal
        \[\tilde \Delta\colon \tilde D_p(X\wedge Y) \to \tilde D_p(X)\wedge \tilde D_p(Y)\]
        given by \([\underline x \wedge \underline y\wedge v] \mapsto[\underline x\wedge v]\wedge [\underline y \wedge v] \). The following commutes
        \[\begin{tikzcd}
            (X\wedge Y)^p \ar[d, "j"] \ar[r, "\text{shuffle}", "\cong"'] & (X^{\wedge p})\wedge (Y^{\wedge p}) \ar[d, "j\wedge j"] \\
            \tilde D_p(X\wedge Y) \ar[r, "\tilde\Delta"] & \tilde D_p(X)\wedge \tilde D_p(Y) \\
        \end{tikzcd}\]
        We let \(\tilde c\colon K(n)\wedge K(m)\to K(n+m)\) be the unique up to homotopy based map s.t.
        \[\tilde c^*(\iota_{n+m}) = \iota_n\wedge \iota_m\]
        It induces a based map
        \[\tilde D_p(K(n)\wedge K(m))\to \tilde D_p(K(n+m))\]

        \textbf{Claim.} \(\tilde\Delta(\tilde \iota_{n,p}\wedge \tilde \iota_{m,p}) = \tilde D_p(\tilde c)^*(\tilde\iota_{n+m,p})\). This is in \(H^{(n+m)p}(\tilde D_p(K(n)\wedge K(m), \IF_p))\).

        Since \(K(n)\) is \((n-1)\)-connected, \(K(m)\) is \((m-1)\)-connected, \(K(n)\wedge K(m)\) is \((n+m-1)\)-connected. So \((K(n)\wedge K(m))^{\wedge p}\) is \((p(n+m)-1)\)-connected. In a previous proposition we had \(j\colon (K(n)\wedge K(m))^{\wedge p} \to \tilde D_p(K(n)\wedge K(m))\) and \(j^*\) is injective.
        \[\begin{split}
            j^*(\tilde \Delta^*(\tilde \iota_{n,p}\wedge \tilde \iota_{m,p})) &= \text{shuffle}^*((j\wedge j)^*(\tilde \iota_{n,p}\wedge \tilde \iota_{m,p})) \\
            &= \text{shuffle}^*(j^*(\tilde \iota_{n,p})\wedge j^*(\tilde\iota_{m,p})) \\
            &= \text{shuffle}^*((\iota_n\wedge \dots \wedge \iota_n) \wedge (\iota_m\wedge \dots \wedge \iota_m)) \\
            &= ((\iota_n\wedge \iota_m)\wedge \dots \wedge (\iota_n\wedge \iota_m)) \\
            &= \tilde c^*(\iota_{n+m})\wedge \dots \wedge \tilde c^*(\iota_{n+m}) \\
            &= (\tilde c\wedge \dots \wedge \tilde c)^*(\iota_{n+m}\wedge\dots \wedge \iota_{n+m}) \\
            &= (\tilde c\wedge \dots \wedge\tilde c)^*(j^*(\tilde\iota_{n+m, p})) \\
            &= j^*(\tilde D_p(\tilde c)(\tilde\iota_{n+m, p}))
        \end{split}\]
        Since \(j^*\) is injective, this proofs the claim.

        We turn the relation into an unreduced form.
        \[\Delta\colon D_p(X\times Y) \to D_p(X)\times D_p(Y)\]
        the diagonal. We define \(c = \tilde c \circ \pi\colon K(n)\times K(m)\xrightarrow{\pi} K(n)\wedge K(m)\xrightarrow{\tilde c} K(n+m)\). We have
        \[c^*(\iota_{n+m}) = \pi^*(\tilde c^*(\iota_{n+m})) = \pi^*(\iota_n \wedge \iota_m) = \iota_n \times \iota_m\]
        If \(X\) and \(Y\) are based, the following commutes:
        \[\begin{tikzcd}
            X\times Y\times L(p) \ar[d, "\Delta_{X\times Y}"]\ar[r, "\Delta"] & X\times L(p)\times Y\times L(p) \ar[d, "\Delta_X \times \Delta_Y"] \\
            D_p(X\times Y) \ar[r, "\Delta"] \ar[d, "\pi"] & D_p(X)\times D_p(Y) \ar[d, "\pi_0(\pi\times \pi)"] \\
            \tilde D_p(X\wedge Y) \ar[r, "\tilde\Delta"] & \tilde D_p(X)\wedge \tilde D_p(Y)
        \end{tikzcd}\]
        \[\begin{split}
            \Delta^*(\iota_{n,p}\times \iota_{m,p}) &= \Delta^*(\pi^*(\iota_{n,p}\wedge \iota_{m,p})) \\
            &= \pi^*(\tilde \Delta^*(\iota_{n,p}\wedge \iota_{m,p})) \\
            &= \pi^*(\tilde D_p(c)^*(\tilde \iota_{n+m}, p)) \\
            &= D_p(c)^*(\pi^*(\tilde\iota_{n+m}, p)) \\
            &= D_p(c)^*(\iota_{n+m}, p) \\
        \end{split}\]

        Thus
        \[\begin{split}
            P_p(\iota_n\times \iota_m) &= P_p((c\circ \pi)^*(\iota_{n+m})) = \Delta^*_{K(n)\times K(m)}(:_p(c\circ \pi)^*(\iota_{n+m,p}))\\
            &= \Delta_{K(n)\times K(m)}(\Delta^*(\iota_{n,p}\times \iota_{m,p})) \\
            &= \Delta^*((\Delta_{K(n)}\times \Delta_{K(m)})(\iota_{n,p}\times \iota_{m,p}))= \Delta^*(P_p(\iota_n)\times P_p(\iota_m))
        \end{split}\]
    \end{enumerate}
\end{proof}

\newLecture{01.12.2025}

We want to see how \(\tilde D_2(S^1)\) looks like.

\begin{proposition}
    There is a homeomorphism \(h\colon \tilde D_2(S^1)\xrightarrow{\cong} S^1 \wedge \ir P^\infty\), such that the composite
    \[S^1\wedge \ir P^\infty_+ \xrightarrow{\tilde \Delta_{S^1}} \tilde D_2(S^1)\xrightarrow[h]{\cong} S^1 \wedge \ir P^\infty\]
    is homotopic to the projection. The first map was \(x\wedge [v] \mapsto [x\wedge x\wedge v]\).
\end{proposition}

\begin{proof}
    The proof will be a bit more geometric than what we did of last.
    \begin{description}
        \item[Step 1] \(S^1_{\text{sgn}} = \ir\cup \set \infty\) with sign action \(x\mapsto -x\). We construct a homeomorphism
        \[S^1_{\mathrm{sgn}} \wedge_{C_2} S^\infty_+ \xrightarrow{\cong} \ir P^\infty\]
        Fix \(m\geq 0\). Consider the continuous map
        \[\ir \times S \subset \IR^m \to \ir P^m, \quad (x, v_1, \dots, v_m) \mapsto [x:v_1: \dots : v_m]\]
        For \(x\neq 0\), \([x:v_1: \dots : v_m] = [1:v_1/x : \dots : v_m: x]\), so this extends continuously to \((\ir \cup\set\infty)=S^1\times S(\ir ^m) \to \ir P^m\) by \((\infty, v_1, \dots, v_m) \mapsto [1: 0 : \dots : 0]\).

        This factors over the quotient
        \[S^1 \wedge S(\ir ^m)_+ = \frac{S^1\times S(\ir ^m)}{\set\infty \times S(\ir^m)}\]
        and also
        \[[x:v_1: \dots : v_m] = [-x : -v_1 : \dots : -v_m]\]
        so this factors over a continuous surjection
        \[S^1_{\mathrm{sgn}}\wedge_{C_2} S(\ir ^m)_+ \to \ir P^m\]
        which is also injective. This is a continuous bijection from a quasicompact space to a Hausdorff space, hence a homeomorphism.
        
        For \(m\to \infty\).
        \[\begin{tikzcd}
            S^1_{\mathrm{sgn}} \wedge_{C_2} S(\ir^m)_+ \ar[r, "{[x, v_1, \dots, v_m]\mapsto [x, v_1,\dots, v_m, 0]}"] \ar[d, "\cong", "k_m"'] & S^1_{\mathrm{sgn}}\wedge_{C_2}S(\ir^{m+1})_+ \ar[d, "\cong", "k_{m+1}"'] \\
            \ir P^m \ar[r, "{[y_0: \dots : y_m]\mapsto [y_0: \dots : y_m : 0]}"] & \ir P^{m+1} \\
        \end{tikzcd}\]
        Commutes, so we get a homeomorphism \(k\colon K^1_{\mathrm{sgn}} \wedge_{C_2} S^\infty_+ \xrightarrow{\cong} \ir P^\infty\).

        The composite \(\ir P^\infty \to S^1_{\mathrm{sgn}}\wedge_{C_2} S^\infty_+ \xrightarrow[\cong]{h} \ir P^\infty\) where the first map is given by \([v]\mapsto [0\wedge v]\) is given by \([y_0 : y_1 : \dots] \mapsto [0:y_0 : y_1:\dots]\). This is homotopic to the identitiy.
        \[[0, \pi/2]\times \ir P^\infty \to \ir P\infty, \quad (t, [y_0: y_1:\dots]) \mapsto [\sin(t)y_0 : \cos(t)y_0 + \sin(t)y_1 : \cos(t)y_1 + \sin(t)y_2 : \dots]\]
        is a homotopy between the two maps.
        \item[Step 2] We consider the invertible matrix \(A = \frac{1}{2} \begin{pmatrix}
            1 & 1 \\
            -1 & 1 \\
        \end{pmatrix}\). This is such that \(A \cdot(x,x) = (x,0)\). We have \(\det (A) = 1/2 > 0\) So the induced map
        \[A\cdot\_\colon S^2 = \ir ^2\cup\infty \to \ir ^2\cup \infty = S^2\]
        is homotopic to \(\id\). \(A\) is equivariant for two different involutions on \(S^2\): On the source we flip \(S^2\to S^2 x\wedge y \mapsto y\wedge x\). And on the target \(S^1 \wedge S^1_{\mathrm{sgn}}, x\wedge y \mapsto x\wedge -y\). We get an induced homeomorphism
        \[A\wedge_{C_2}S^\infty_+ \colon S^2_{\text{flip}} \wedge_{C_2}S^\infty_+ \xrightarrow{\cong} (S^1 \wedge S^1_{sgn})\wedge_{C_2}S\infty_+ = S^1 \wedge (S^1_{\mathrm{sgn}} \wedge _{C_2} S^\infty _+)\xrightarrow[\cong]{\cong S^1\wedge k}S^1 \wedge \ir P^\infty\]
        So \(h = (S^1 \wedge k)\circ (A\wedge _{C_2} S^\infty_+)\).
        \[\begin{tikzcd}
            & S^1 \wedge \ir P^\infty_+ \ar[d, "{S^1 \wedge [0, \_]}"]\ar[dl, "\tilde \Delta_{S^1}"]\ar[dr, "S^1 \wedge \mathrm{proj}"] & \\
            \tilde D_2(S^1)\ar[rr, "h", bend right=30] \ar[r, "A\wedge ??"] & S^1 \wedge (S_{\mathrm{sgn}}^1\wedge_{C_2} S^\infty_+) \ar[r, "S^1 \wedge k"] & S^1 \wedge \ir^\infty \\
        \end{tikzcd}\]
        He claims the top right triangle commutes on the nose
    \end{description}
\end{proof}

Let \(\iota\in H^1(S^1, \iz)\) be a generator. Let \(u \in H^1(\ir P\infty, \IF_2)\) be the unique generator. \(P_2(\iota) \in H^2(S^1\times \ir P^\infty, \IF_2) = \IF_2{1\times u^2, \iota\times u}\).

\begin{proposition}
    \(P_2(\iota) = \iota\times u \) in \(H^2(S^1\times \ir P^\infty, \IF_2)\).
\end{proposition}

\begin{proof}
    Let \(g\colon S^1 \to \ir P^\infty\) be a based map representing the non bero element of \(\pi_1(\ir P\infty, *)\). Then \(g^*(u) = \iota\) in \(H^1(S^1, \IF_2)\). We have \(H_2(S^1 \wedge \ir P\infty, \iz) \cong H_1(\ir P\infty, \iz) = \iz/2\). Since \(S^1\wedge \ir P^\infty\) is simply connected, Hurewicz theorem says that \(\pi_2(S^1 \wedge \ir P\infty, ?) \cong H_2(S^1 \wedge \ir P^\infty, \iz) = \iz/2\).

    The map \(S^1 \wedge g\colon S^1\wedge S^1 \to S^1 \wedge \ir P^\infty\) is nonzero on \(H^2(\_, \IF_2)\). \(S^1 \wedge S^1 \xrightarrow{j} (S^1 \wedge S^1)\wedge _{C_2}S^\infty_+ = \tilde D_2(S^1) \xrightarrow[h]{\cong} S^1 \wedge \ir P^\infty\) given by \(x\wedge y \mapsto [x\wedge y\wedge (1, 0,\dots)]\) is also nonbero as \(H^2(\_, \IF_2)\) by proposition 5.5 in Schwedes own notes.

    So \(S^1\wedge g\) and \(h\circ j\colon S^1 \wedge S^1\to S^1 \wedge \ir P^\infty\) are both nontrivial in \(\pi_2(S^1 \wedge \ir P^\infty,*) \cong \iz /2\). So they are homotopic.

    \(\iota\wedge u \in H^2(S^1 \wedge \ir P^\infty, \IF_2)\). Note
    \[\begin{split}
        j^*(h^*(\iota\wedge u)) &= (h\wedge j)^*(\iota\wedge u) = (S^1 \wedge g)^*(\iota\wedge u) \\
        &= \iota\wedge g^*(u) = \iota\wedge u \\
    \end{split}\]
    and, as \(\ir P^\infty \simeq K(\IF_2, 1)\) and \(u = \iota_1\),\(\iota_{1,2}\in H^2(\tilde D_2(\ir P^\infty), \IF_2)\),
    \[\begin{split}
        j^*(\tilde \iota_{1,2}) = \iota_1 \wedge \iota_1 = u\wedge u \in H^2(\ir P^\infty\wedge \ir P^\infty, \IF_2)\\
    \end{split}\]
    we get a diagram
    \[\begin{tikzcd}
        S^1 \wedge S^1 \ar[d, "j"] \ar[r, "g\wedge g"] & \ir P^\infty \wedge \ir P^\infty \ar[d, "j"] \\
        \tilde D_2(S^1) \ar[r, "\tilde D_2(g)"] & \tilde D_2(\ir P^\infty) \\
    \end{tikzcd}\]
    \[j^*(\tilde D_2(g)^*(\tilde \iota_{1,2})) = (g\wedge g)^*(j^*(\tilde \iota_{1,2})) = (g\wedge g)^*(u\wedge u) = g^*(u)\wedge g^*(u) = \iota\wedge \iota\]
    and also
    \[h^*(\iota\wedge u) = \tilde D_2(g)^*(\tilde \iota_{1,2})\]

    \[\begin{tikzcd}
        S^1\times \ir P^\infty  \ar[d, "\Delta_{S^1}"] \ar[r, "\pi"] & S^1 \wedge \ir P^\infty _+ \ar[d, "\tilde\Delta_{S^1}"]\ar[dr, "S^1 \wedge \mathrm{proj}"] &   \\
        D_2(S^1) \ar[d, "D_2(g)"]   \ar[r, "\pi"] & \tilde D_2(S^1) \ar[d, "\tilde D_2(g)"]  \ar[r, "\cong", "h"'] & S^1 \wedge \ir P^\infty \\
        D_2(\ir P^\infty) \ar[r, "\pi"]  & \tilde D_2(\ir P^\infty) &\\
    \end{tikzcd}\]
    and now we get
    \[\begin{split}
        P_2(\iota) &= \Delta^*_{S^1}(D_2(g)^*(\iota_{1,2})) \\
        &= \Delta^*_{S^1}(D_2(g)^*(\pi^*(\iota_{1,2}))) \\
        &= \pi^*(\tilde \Delta_{S^1}^*(\tilde D_2(g)^*(\tilde \iota_{1,2})))\\
        &= \pi^*(\tilde \Delta_{S^1}^*(h^*(\iota\wedge u))) \\
        &= \pi^*(S^1 \wedge ??)^*(\iota\wedge u) \\
        &= \pi^*(\iota\wedge u)0= \iota \times u \\
    \end{split}\]
\end{proof}

\begin{thm}{Steenrod squares}{}
    The total squaring operation \(P_2\) and the Steenrod square are related by
    \[P_2(x) = \sum_{i = 0}^n \Sq^i(x)\times u^{n-i}\quad \text{ in } H^{2n}(X\times \ir P^\infty, \IF_2)\]
    for \(x \in H^n(X, \IF_2)\), \(u\in H^1(\ir P^\infty, \IF_2)\).
\end{thm}

\begin{proof}
    We define \(T_n\colon H^n(X, \IF_2) \to H^{n+i}(X, \IF_2)\) as the cohomology operation characterized by \(P_2(x) = \sum_{i = 0}^n T_n^i(x)\times u^{n-i}\). We need to show \(T_n^i = \Sq^i\).

    \begin{description}
        \item[Step 1] \(T_n^n(x) = x^2\) for \(x\) of degree \(n\). We have seen earlier \(j^*(P_2(x)) = x^2\),  where \(j \colon X\to X\times \ir P^\infty\) and
        \[j^*(y\times u^i) = \begin{cases}
            y & \text{ if } i = 0 \\
            0 & i > 0 \\
        \end{cases}\]
        We see
        \[j^*(P_2(x)) = j^*(\sum_{i = 0}^n T_n^i(x)\times u^{n-i}) = T^n_n(x)\]
        \item[Step 2] Cartan formula: \(T^i_{k+l}(x\times y) = \sum_{a+b = i} T_k^a(x)\times T_l^b(y)\), \(x \in H^k(X, \IF_2), y \in H^l(X, \IF_2)\).
        We also know \(P_2(x\times y) = \Delta^*(P_2(x)\times P_2(y))\) for \(\Delta\colon X\times Y \times \ir P^\infty \to X\times \ir P^\infty \times Y\times \ir P^\infty\) and in total
        \[\begin{split}
            P_2(x\times y) &= \Delta^*(P_2(x)\times P_2(y)) \\
            &= \Delta^*(\sum_{a,b, \geq 0} T_k^a(x)\times u^{k-a}\times (T_l^b(y)\times u^{l-b})) \\
            &= \sum_{a, b \geq 0} T_k^a(x)\times T_l^b(y)\times u^{k+l-(a+b)} \\
            &= \sum_{i\geq 0} (\sum_{a+b = i} T^a_k(x)\times T_l^b(y))\times u^{k+l-i}
        \end{split}\]
        And now this is the cartan formula. We have seen for \(\iota\in H^1 (S^1, \IF_2)\)
        \[\sum_{i\geq 0} T_1^i(\iota)\times u^{1-i} P_2(\iota) = \iota\times u\]
        and hence
        \[T_1^i(\iota) = \begin{cases}
            \iota & i = 0 \\
            0 & i\geq 0
        \end{cases}\]
        \item[Step 3] Stability: \(T_{n+i}^i(x\wedge \iota) = \sum_{j= 0}^i T_n^j(x)\wedge T_1^{i-j}(\iota) = T_n^i(x)\wedge \iota\). This is because \(\wedge \iota\) is exactly the suspension isomorphism. Hence
        \[T^i = \set{T^i_n}_{n\in \IN}\]
        form a stable operation. Uniqueness of \(\Sq^i\)'s proove the Theorem.
    \end{description}
\end{proof}

\newLecture{08.12.2025}

We sketch how total power operations work for odd primes. We denote them by \(P^i\).

For the entire lecture \(p\geq 3\)a odd prime

\textbf{Recall.} \(H^*(L(p), \IF_p) = \IF_p[x,y]/(x^2)\) for some \(x \in H^1(L(p), \IF_p), y = \beta(x)\in H^2(L(p), \IF_p)\).

We set \(u = x\cdot y^{p-2}\in H^{2p-3}(L(p), \IF_p)\) and \(v = y^{p-1}\in H^{2p-2}(L(p), \IF_p)\). And then
\[\beta(u) = \beta(x\cdot y^{p-2}) = \beta(x)\cdot y^{p-2} = y^{p-1} = v\]
by an exercise from ages ago.

The secret reason why we choose these particular \(u,v\) in \(H^*(L(p),\IF_p)\) to show up.
\[\IF_p^\times = \IF_p\setminus\set0 \]
is a multiplicative\footnote{and cyclic!} group of \(\IF_p\) and acts on \((\IF_p, +)\) by multiplication. This \(\IF_p^\times\)-action induces an action-up-to-homotopy on \(L(p)= K(\IF_p,1)\). Earlier: \([K(G,1), K(H,1)]_*\mapsto \hom_{\mathrm{Grp}}(G,H)\) is an isomorphism. So for all \(\lambda\in \IF_p^*\) there is a unique based homotopy class of map \(t_\lambda\colon L(p)\to L(p)\) that induces \(\lambda\cdot\_ \colon \IF_p \to \IF_p\) on \(\pi_1\).

We also get \(t_\mu \circ t_\lambda = t_{\mu\cdot\lambda}\).

We have \(\pi_1(L(p), *) \xrightarrow{\cong} H_1(L(p), \iz) \cong \IF_p\), so \(t_\lambda\) induces multiplication by \(\lambda\) also on \(H^1(\_, \iz)\), hence by UCT also on \(H^1(\_, \IF_p)\). So in particular for \(x \in H^1(L(p), \IF_p)\).
\[t_\lambda^*(x) = \lambda x \neq x \quad \text{unless } \lambda = 1\]
and also
\[t_\lambda^*(y) = t_\lambda^*(\beta(x)) = \beta(t_\lambda^*(x)) = \beta(\lambda \cdot x) = \lambda \cdot\beta(x) = \lambda \cdot y \neq y\]
so these elements \(x,y\) are \textbf{not} invariant under these automorphisms except for when \(p = 2\).

But we have
\[t^*_\lambda(u) = t^*_\lambda(x\cdot y^{p-2}) = t_\lambda^*(x)\cdot t_\lambda^*(y)^{p-2}= \lambda \cdot x \cdot (\lambda y)^{p-2} = \underbrace{\lambda^{p-1}}_{= 1} \cdot x\cdot y^{p-1} = u\]
and
\[t_\lambda^*(v) = t_\lambda^*(y)^{p-1} = (\lambda y)^{p-1} = \lambda^{p-1} \cdot y^{p-1} = v\]
So these two elements \(u,v\) are invariant under the action of the automorphism group on \(H^*(L(p), \IF_p)\) induced by the homotopy-action on \(L(p)\). A \enquote{small} Algebra exercise gives
\[(H^*(L(p), \IF_p))^{\IF_p^\times} = \text{ subring of homogenous elements invariant under the \(\IF_p^\times\)-action} = \IF_p[u,v]/(v^2)\]

\begin{proposition}
    For every prime \(p\), every space \(X\), all \textbf{even} \(n\geq 0\), the image of
    \[\cP_p\colon H^n(X,\IF_p) \to H^{np}(X\times L(p), \IF_p)\]
    is invariant under the \(\IF_p^\times\)-action induced by the homotopy action on \(L(p)\).
\end{proposition}

\begin{proof}
    We don't want to do this in full generality. The general argument uses \(X^p\times_{\Sigma_p} E\Sigma_p\). We would need to talk about universal constructions of group actions or something.

    For \(p=2\) the statement is empty.

    For \(p = 3\) we consider the involution \(\Psi\colon (v_1, v_2,\dots) = (\bar V_1, \bar v_2,\dots)\) on \(S^\infty = S(\IC^\infty)\), where \(\bar x\) is the complex conjugate of \(x\in \IC\). Then
    \[\zeta_3 = e^{2\pi i/3} \text{ has the property that } \bar\zeta_3 = \zeta_3^2\]
    so \(\Psi(\zeta_3 \cdot v) = \bar \zeta_3 \cdot \Psi(v) = \zeta_3^2 \Psi(v)\).

    This means \(\Psi\colon S^\infty \to S^\infty\) descends to a well defined continuous on the quotient space \(\bar\Psi\colon L(3)\to L(3)\). On fundamental group we get \(\pi_1(\bar\Psi)\colon \pi_1(L(3),*)\to \pi_1(L(3), *)\) and this is the inverse map by covering theory and a picture (I didn't copy). So \(\bar\Psi = t_{-1}\).

    We define an involution \(\bar\Psi_X\colon D_3(X)\to D_3(X)\)
    \[D_3(X) = X^3\times_{C_3} S^\infty\to X^3\times_{C_3}S^\infty = D_3(X)\]
    by sending \([x,y,z;v]\mapsto [y,x,z;\Psi(v)]\).
    We rambled a bit about why \(x,y\) are swapped in the second coordinate. This is well defined:
    \[(x,y,z;v) \mapsto (y,x,z; \Psi(v)), \quad (y,z,x; \zeta_3 v) \mapsto (z,y,x;\Psi(\zeta_3 v))\]

    The following square commutes:
    \[\begin{tikzcd}
        K(\IF_3, n)^{\wedge 3} \ar[r, "j"] \ar[d, "\tau\wedge \id"] & \tilde D_3(K(\IF_3, n)) \ar[d, "\bar\Psi_{K(\IF_3,n)}"] \\
        K(\IF_3, n)^{\wedge 3} \ar[r, "j"] & \tilde D_3(K(\IF_3,n)) \\
    \end{tikzcd}\]


    by observing for \(\tilde\iota_{n,3}\in H^{3n}(\tilde D_3(K(\IF_3,n)), \IF_3)\)
    \[\begin{split}
        j^*(\bar\Psi_{K(\IF_3,n)}^*(\tilde\iota_{n,3})) &= (\tau\wedge \id)^*(j^*(\tilde \iota_{n,3})) \\
        &= (\tau\wedge \id)^*(\iota_n\wedge \iota_n\wedge \iota_n) \\
        &= (-1)^{n\cdot n} \iota_n\cdot\iota_n \cdot\iota_n \overset{n\text{ even}} \iota_n\cdot\iota_n\cdot\iota_n = j^*(\tilde\iota_{n,3}) \\
    \end{split}\]
    \(j^*\) is injective in \(H^{3n}(\_, \IF_3)\) and this implies \(\bar\Psi_{K(\IF_3,n)} ^*(\tilde \iota_{n,3}) = \bar\iota_{n,3}\).

    The following also commutes
    \[\begin{tikzcd}
        D_3(K(\IF_3,n)) \ar[r, "\prod\nolimits"]\ar[d, "\bar\Psi_{K(\IF_3,n)}"] & \tilde D_3(K(\IF_3, n)) \ar[d, "\tilde \Psi_{K(\IF_3, n)}"] \\
        D_3 (K(\IF_3,n)) \ar[r, "\prod\nolimits"] & \tilde D_3(K(\IF_3,n)) \\
    \end{tikzcd}\]
    So
    \[\begin{split}
        \bar\Psi_{K(\IF_3,n)}^*(\iota_{n,3}) &= \bar\Psi_{K(\IF_3,n)}(\prod\nolimits^*(\tilde\iota_{n,3})) \\
        &= \prod\nolimits^*(\tilde \Psi_{K(\IF_3,3)}^*(\tilde\iota_{n,3})) = \prod\nolimits^*(\tilde\iota_{n,3}) = \iota_{n,3}
    \end{split}\]
    The following commutes too
    \[\begin{tikzcd}
        X\times L(3) \ar[r, "\Delta_X"] \ar[d, "X\times \bar\Psi"] & D_3(X)\ar[d, "\bar\Psi_X"] \\
        X\times L(3) \ar[r, "\Delta_X"] & D_3(X)
    \end{tikzcd}\]
    So we also get setting \(X = K(\IF_3,n)\)
    \[(K(\IF_3,n) \times \bar\Psi)^*(P_3(\iota_n)) = \Delta^*_{K(\IF_3,n)}(\bar\Psi_X^*(\iota_{n,3})) = \Delta_{K(\IF_3,n)}(\iota_{n,3}) = P_3(\iota_n)\]

    By naturality now all \(P_3(X)\) are invariant under \(X\times \bar\Psi\).
\end{proof}

We now have
\[\begin{tikzcd}
    H^n(X,\IF_p)\ar[rrd, dotted] \ar[r, "\cP_p"] & H^{np}(X\times L(p); \IF_p) \ar[phantom, r, "="] & H^*(X,\IF_p)[x,y]/(x^2) \\
    & & H^*(X,\IF_p)[u,v]/(u^2) \ar[u, phantom, sloped, "\subseteq"] \\
\end{tikzcd}\]
We can expand
\[\cP_p(x) = \sum_{i = 0}^k (P^i_k(x) \times v^{k-i}) +  (R_k^i(x) \times uv^{k-i-1}) \in H^{np}\]

For \(n = 2k\) even
\[P_k^i\colon H^{2k}(X,\IF_p)\to H^{2k+2i(p-1)(X, \IF_p)}\]
\[R_k^i\colon H^{2k}(X, \IF_p) \to H^{2k+2i(p-1)+1}(X;\IF_p)\]
and we check that the degrees match up properly.

For \(p=2\) we showed that \(P_2(\iota) = \iota\times u\) for \(\iota\in H^1(S^1, \IF_2)\).

\textbf{Fact.} \(\cP_p(\iota\wedge \iota) = (\iota\wedge \iota)\times v \in H^{2p}(S^2\times L(p); \IF_p)\) with \(\iota\wedge \iota \in H^2(S^2, \IF_p)\). So
\[P_1^0(\iota\wedge \iota) = \iota\wedge \iota\]
and
\[P_1^i(\iota\wedge \iota) = 0\]
for \(i\geq 0\).

\begin{thm}{Total powers for odd primes}{}
    Let \(p\) be an odd prime. The generators \(P_k^i\) have the following properties:
    \begin{enumerate}
        \item We have \(P_k^k(x) = x^p\), \(P_k^i(x) = 0\) for \(i > k\).
        \item Carten formula For \(x\in H^{2k}(X,\IF_p), y \in H^{2l}(Y, \IF_p)\) we have
        \[P_{k+l}^i(x\times y) = \sum_{a+b = 1} P_k^a(x)\times P_l^b(y)\]
        \item The operations \(P_k^i\) commute with 2 fold suspension in the following sense: For \(x\ni H^{2k}(X,\IF_p)\)
        \[P_{k+1}^i(x\wedge \iota\wedge \iota) P_k^i(x)\wedge \iota\wedge \iota\]
    \end{enumerate}
\end{thm}

\begin{proof}
    \begin{enumerate}
        \item Earlier: \(j^*\colon H^*(X\times L(p), \IF_p)\to H^*(X,\IF_p)\) satisfies \(j^*(P_p(x)) = x^p\). Also for \(z\in H^*(X,\IF_p), \epsilon\in \set{0,1}\), \(i\geq 0\)
        \[j^*(z\times u^\epsilon v^i) = \begin{cases}
            z& \epsilon = 0, i = 0 \\
            0 & \text{else}
        \end{cases}\]
        And we get
        \[x^p = j^*(P_p(x)) = j^*(\sum_{i = 0}^k P_k^i(x)\times v^{k-i}+ R_k^i(x)\times u v^{k-i-1}) = P_k^k(x)\times 1\]
        \item For \(\Delta\colon X\times Y\times L(p)\to X\times L(p)\times Y\times L(p)\)
        \[\begin{split}
            P_p(x\times y) &=\Delta^*( P_p(x)\times P_p(y)) \\
            &= \Delta^*(\sum_{a,b\geq 0} (P_k^a(x)\times v^{k-a} + R_k^a(x)\times uv^{k-a-1})\times (R_l^b(y)\times v^{k-b} + R_l^b(y)\times uv^{k-b-1})) \\
            &= \sum_{a,b\geq 0} P_k^a(x)\times P_l^b(y)\times v^{k+l-(a+b)} + (P_k^a(x)\times R_l^b(y)+ R_k^a(x)\times P_l^b(y)\times uv^{k+l-(a+b)-1})\\
            &= \sum_{i\geq 0}\sum_{a+b = i} (P_k^a(x)\times P_l^b(y)\times v^{k+l-i}) + \dots 
        \end{split}\]
        The chance I copied this entirely correctly is low, but never zero.

        Compare coefficient of \(v^{k+l-i}\) on both sides gives \(P_{k+l}^i(x\times y) = \sum_{a+b = i} P_k^a(x) \times P_l^b(y)\).
        \item
        \[\begin{split}
            P_{k+1}^i(x\wedge \iota\wedge \iota) &= \sum_{j = 0}^iP_k^{i-j}(x)\wedge P_1^j(\iota\wedge \iota) \\
            &= P_k^i(x)\wedge \iota\wedge \iota 
        \end{split}\]\qedhere
    \end{enumerate}
\end{proof}

\newLecture{10.12.2025}

\begin{defi}{Total power operations as stable operations}{}
    Let \(p\) be an odd prime, \(n \geq 0\). We define the stable mod-\(p\) cohomology operation
    \[P^i\colon H^n(X, \IF_p) \to H^{n+2i(p-1)}(X, \IF_p)\]
    by
    \[P^i(x) = \begin{cases}
        P^i_k(x) & n = 2k \text{ even} \\
        \Sigma^{-1}(P^i_k(\Sigma X)) & n = 2k-1 \text{ odd} \\
    \end{cases}\]
    For \(\Sigma\) the suspension isomorphism.
\end{defi}

\begin{thm}{}{}
    The Steenrod operations \(P^i\) have the following properties:
    \begin{enumerate}
        \item \(P^0\) is the identity.
        \item (Unstability) \(P^i(x) = x^p\) if \(\abs x = 2i\), \(P^i(x)= 0\) for \(2i > \abs x\) where \(\abs x\) denotes the cohomology degree of \(x\).
        \item Catan formulas:
        \[P^i(x\cup y) = \sum_{a+b = i} P^a(x)\cup P^b(y)\]
        for \(x,y \in H^*(X, \IF_p)\) and
        \[P^i(x\times y) = \sum_{a+b = i} P^a(x)\times P^b(y)\]
        for \(x \in H^*(X, \IF_p), y \in H^*(Y, \IF_p)\).
    \end{enumerate}
\end{thm}

\textbf{Fact.} For \(p = 2\), the operations \(\Sq^i\) generate \(\cA_2\) as a graded \(\IF_2\)-algebra. Note \(\Sq^1 = \beta\), \(\Sq^{n+1} = \Sq^1 \circ \Sq^n = \beta \circ \Sq^n\) for \(n\) even.

For \(p\) odd, the \(P^i\) and the Bockstein \(\beta\) generate \(\cA_p\) as a graded \(\IF_p\)-algebra. And also \((R^i = \beta \circ P^i)\). We have a correspondence
\[\Sq^i \cong \begin{cases}
    P^{1/2} & i \text{ even} \\
    \beta\circ P^{(i-1)/2} & i \text{ odd} \\
\end{cases}\]


\section{Adem relations}

The Adem relations are specific relations for \(\Sq^i \circ \Sq^j = \dots\).

\begin{thm}{}{}
    Let \(p\) be any prime. Let \(n \geq 0\), suppose that \(n\) is even, if \(p\) is odd. Then the image of the iterated total power operation
    \[P_p\circ P_p \colon H^2(X, \IF_p) \to H^{np^2}(X\times L(p)\times L(p); \IF_p)\]
    is invariant under the involution induced by switching the two factors of \(L(p)\).
\end{thm}

We will not proof this, because this requires \(X^{p^2}\times_{\Sigma_{p^2}} E\Sigma_{p^2}\). And we have never done this. He talks words I don't understand without writing on the board. Something \([K(G,1), K(H,1)]\cong \Hom(G,H)/\text{conjugation}\).

Now take \(p=2\), and we derive the Adem relations from the symmetry property of \(P_2 \circ P_2\).

\begin{construction}
    \(P(t) = \sum_{i = 0}^\infty \Sq^i \cdot t^i \in \cA_2[[t]]\)
\end{construction}
This construction is mainly a trick to shorten our notation vastly.

\begin{proposition}
    The power series \(P(t)\) satisfies the identity
    \[P(1+t) \cdot P(t^2) = P(t + t^2) \cdot P(1)\]
    in \(\cA_2\llbracket t\rrbracket\)
\end{proposition}

\begin{proof}
    Künneth isomorphism \(\times \colon H^*(X, \IF_2)\times H^*(L(2), \IF_2) \xrightarrow{\cong} H^*(X\times L(2), \IF_2) = H^*(X, \IF_2)[v]\) For \(v \coloneq 1 \times u\).

    Similarly \(H^*(X\times L(2)\times L(2), \IF_2) = H^*(X, \IF_2)[s, t]\)\footnote{This is a \enquote{tri-graded} ring.} with \(s = 1 \times 1 \times u, t = 1 \times u \times 1\). For \(n = \abs x\) we have
    \[\cP_2(x) = \sum_{i = 0}^\infty \Sq^i(x) \cdot v^{n-i} \]
    \[\begin{split}
        \cP_2(x) &= \sum_{ i = 0}^\infty \Sq^i(1\times u) \times u^{1-i} \\
        &= 1\times u \times u + 1\times u^2 \times 1 \\
        &= t \cdot s + t^2
    \end{split}\]
    Now we watch in \(H^*(X\times L(2)\times L(2), \IF_2) = H^*(X, \IF_2)[s,t]\)
    \[\begin{split}
        \cP_2(\cP_2(x)) &= \cP_2(\sum_{j\geq 0}\Sq^j(x)\cdot v^{n-j}) \\
        \text{addidivity \& Catan formula } &= \sum_{j \geq0} \cP_2(\Sq^j(x)) \cdot (\cP_2(v))^{n-j} \\
        &= \sum_{j\geq 0}\sum_{i \geq 0} \Sq^i(\Sq^j(x)) \cdot (ts+ t^2)^{n-j} \\
        &= s^n(s+t)^n\cdot t^n \sum_{i,j\geq 0} \Sq^i(\Sq^j(x)) \cdot s^{-i}(t+t^2s^{-1})^{-j} \\
    \end{split}\]
    where we nov work in the localization \(H^*(X,\IF_2)[s^{\pm 1}, t^{\pm1}]\). By previous proposition, this expression is invariant under exchanging \(s\) and \(t\), so also the sum
    \[\sum_{i, j \geq 0} \Sq^i(\Sq^j(x)) \cdot s^{-i}(t+ t^2s^{-1})^{-j}\]
    is invariant under exchanging \(s,t\) and also equals
    \[= P(s^{-1}) \cdot P((t+t^2s^{-1})^{-1})\]
    in \(\cA_2^*((s,t))\).

    In particular we get\footnote{Schwede remarks, how we should have a look at the original proof, it this seems to complicated. Afterwards it will seem rather simple.}
    \[P(s^{-1})\cdot P((t+t^2s^{-1})) = P(t^{-1}) \cdot P((s+s^2t^{-1})^{-1})\]

    We substitute \(s = 1/(1+v)\) where this \(v\) is a new variable. \(t = 1/(v+v^2)\). Then
    \[t+s^{-1}t^2 = \frac{1}{v+v^2} + \frac{1+v}{(v+v^2)^2} = \frac{v+v^2+1+v}{(v+v^2)^2} = \frac{1+ v^2}{(v+v^2)^2} = \frac{1}{v^2}\]
    and
    \[s+t^{-1}s^2 = \frac{1}{1+v} + \frac{v+v^2}{(1+v)^2} = 1\]

    and our relation simplifies to
    \[P(1+v) \cdot P(v^2) = P(v+v^2)\cdot P(1)\]
    what we wanted to show. More details to what we are doing here in Schwedes script.
\end{proof}

We make the above theorem explicit modulo \(t^3\).
\[\begin{split}
    P(1+t) \cdot P(t^2) &= (\sum_{i \geq 0} \Sq^i\cdot (1+t)^i)\cdot (\sum_{j\geq 0} \Sq^j t^{2j}) \\
    \mathrm{mod} t^3 &\equiv (\sum_{i \geq 0} \Sq^{i} \cdot (1 + i\cdot t + \binom{i}{2}t^2))\cdot (1 + \Sq^1 t^2) \\
    &= (\sum_{i \geq 0} \Sq^i) \cdot 1 + (\sum_{i \geq 0} \Sq^i \cdot i) \cdot t + (\sum_{i \geq 0} \Sq^i \cdot \binom{i }{2} + \Sq^i \cdot \Sq^1) \cdot t^2
\end{split}\]
and expanding the other side
\[\begin{split}
    P(t+t^2) \cdot P(1) &= (\sum_{i \geq 0} \Sq^i(t+t^2)^i) \cdot (\sum_{j \geq 0} \Sq^j) \\
    \mathrm{mod} t^3 &\equiv (1 + \Sq^1 (t+t^2) + \Sq^2 \cdot t^2) \cdot (\sum_{j\geq 0} \Sq^j) \\
    &= (\sum_{j\geq 0} \Sq^j) \cdot 1 + (\sum_{j\geq 0} \Sq^1\cdot \Sq^j) \cdot t + (\sum_{j/geq 0} \Sq^1 \Sq^j + \Sq^2 \Sq^j) t^2
\end{split}\]
Now we compare coefficients in front of \(t\) and get
\[\sum_{i\geq 0}\Sq^i \cdot i = \sum_{j \geq 0} \Sq^1 \Sq^j \in \prod_{j\geq 0} \cA_2^j\]
In Steenrod degree \(j+1\) we get \(\Sq^{j+1}(j+1) = \Sq^1 \cdot \Sq^j\). And this is
\[\Sq^1 \circ \Sq^j = \begin{cases}
    \Sq^{j+1} & j \text{ even } \\
    0 & j \text{ odd } \\
\end{cases}\]

And doing this for the coefficients of \(t^2\):
\[\sum_{i\geq 0} \Sq^i \cdot \binom{i }{2} + \Sq^i \cdot \Sq^1 = \sum_{j \geq 0} \Sq^1 \cdot\Sq^j + \Sq^2\cdot \Sq^j\]
in Steenrod degree \(j+2\) we get
\[\binom{j+2}{2} \Sq^{j+2} + \Sq^{j+1}\Sq^1 = \Sq^1 \circ \Sq^{j+1}+ \Sq^2\cdot\Sq^{j}\]
and rewriting tihs we get
\[\Sq^2 \circ \Sq^j = \begin{cases}
    \Sq^{j+1} + \Sq^{j+2}\Sq^1 & j \equiv 0 \mod 4 \\
    \Sq^{j+2} + \Sq^{j+1}\Sq^1 + \Sq^{j+2} & j \equiv 1 \mod 4 \\
    \Sq^{j+1}\Sq^1 & j \equiv 2 \mod 4 \\
    \Sq^{j+1}\Sq^1+ \Sq^{j+2} & j \equiv 3 \mod 4 \\
\end{cases} = \begin{cases}
    \Sq^{j+2}+ \Sq^{j+1}\Sq^1 & j\equiv 0,3 \mod 4 \\
    \Sq^{j+1}\Sq^1 & j \equiv 1,2 \mod 4 \\
\end{cases}\]

And now we want to do full Adem relations by a clever bookkeping trick.

\newLecture{15.12.2025}

\subsection{Residue calculuts}

Let \(R\) be a ring, not necessarily commutative.
\[R((t)) = \text{ring of Laurent power seires over } R\]
where the elements are formal
\[f(t) = \sum_{i> -\infty}^\infty f_i \cdot t^i \quad f_i \in R\]
we can also describe this ring as \(R[[t]](t^{-1})\). Let \(\tau(t)\in \iz{t}\) be an integer polynomial s.t. \(\tau(t) \equiv t \mod (t^2)\). Then
\[\tau(t) = t\cdot g(t), \quad f = 1 + \text{ integers }\]
so \(\tau(t)\) is invertible in \(\iz((t))\) and also in \(R((t))\). \(\tau(t) = t+t^2\).

The residue of \(f \in R((t))\) is \(\mathrm{Res}(f) = f_{-1} = \text{coefficient of }t^{-1}\).

\begin{proposition}
    Let \(R\) be a ring, \(f((t))\in R((t))\). Then
    \[\Res(f) = \Res(f(t+t^2)\cdot(1+2t)) \in R\]
\end{proposition}

If you took complex analysis this is motivated by \(\Res(f) = \Res(f(\tau(t))\frac{dt}{dt})\).

\begin{proof}
    Both sides of the equation are \(R\)-linear. The equation is \(0 = 0\) if \(f\in R[[t]]\). So \(f\in R((t))\) is a finite \(R\)-linear combination of \(t^{-j}\), \(j\geq 1\) and a power series. So wlog \(R= \iz\), \(f = t^{-j}\) for \(j\geq 1\).

    \textbf{Claim.} \(\Res((t+t^2)^{-j} \cdot (1+2t)) = \begin{cases}
        1 & j = 1\\
        0 & j\geq 2 \\
    \end{cases}\)
    \begin{description}
        \item[For \(j = 1\)]  \((t+t^2)^{-1} = (t(1+t))^{-1} = t^w{-1}\cdot(1+t)^{-1} = t^{-1} \cdot (1-t+t^2-t^3+\dots) = t^{-1} -1 + t- t^2 + t^3 \dots = \) So \(\Res((t+t^2)^{-1}\cdot (1+2t)) = 1\)
        \item[For \(j \geq 2\)] The formal derivative of
        \(f(t)\in \IZ((t))\)
        is \[f'(t) = \frac{df}{dt} = \sum i\cdot a_i \cdot t^{i-1}\]
        so \(\Res(f') = 0\). Also for \(n\in \iz\).
        \[\frac{d}{dt} (f^m) = m \cdot \frac{d}{dt} f^{m-1}\]

        We observe
        \[\frac{d}{dt}((t+t^2)^{1-j}) = (1-j)(1+2t)\cdot (t+t^2)^{-j} \quad \in \iz((t))\]
        and then
        \[0 = \Res(\frac{d}{dt} (t+t^2)^{1-j}) = (1-j) \cdot \Res((1+2t)(t+t^2)^j) \quad \in \IZ\]
        and because \(1-j \neq 0 \in \iz\) we have \(\Res((1+2t)(t+t^2)^{-j})\).
    \end{description}
\end{proof}

\begin{thm}{Adem Relations}{}
    For all \(a,b \geq 0\)
    \[\Sq^a \circ \Sq^b = \sum_{i = 0}^{\lfloor a/2 \rfloor} \binom{b-i-1}{a-2i} \Sq^{a+b-i} \circ \Sq^i\]
    where we use standard convention of \(\binom{n}{m} = 0\) for \(m< 0\).
\end{thm}

\begin{proof}
    We fix \(a,b \geq 0\). We blow the composition \(\Sq^a\circ \Sq^b\) up. We use \(P(t+t^2) \circ P(1) = P(1+t)\circ P(t^2)\).
    \[\begin{split}
        \Sq^a\circ \Sq^b &= \mathrm{Coeff}_{t^a}(\sum_{j= 0}^{a+b}\Sq^j \circ \Sq^{a+b-j} \cdot t_j) \\
        &= \Res\klam*{\sum_{j= 0}^{a+b}\Sq^j\circ \Sq^{a+b-j} \cdot t^{j-a-1}} \\
        &= \Res\klam*{\sum_{j = 0}^{a+b} \Sq^j \circ \Sq^{a+b-j} \cdot (t+t^2)^{j-a-1}} \\
        &= \Res(\sum_{j = 0}^{a+b} \Sq^j\circ \Sq^{a+b-j} (t+t^2)^j \circ (t+t^2)^{-a-1}) \\
        &= \Res\klam*{\sum_{j= 0}^{a+b} \Sq^j \circ \Sq^{a+b-j}\cdot (1+t)^j \cdot (t^2)^{a+b-j} (t+t^2)^{-a-1}} \\
        (i = a+b-j)&= \Res\klam*{\sum_{i = 0}^{a+b}\Sq^{a+b-i}\Sq^i (1+t)^{a+b-i}t^{2i}(t+t^2)^{-a-1}} \\
        &= \mathrm{Coeff}_{t^a}\klam*(\sum_{i = 0}^{a+b} \Sq^{a+b-i}\circ \Sq^{i}(1+t)^{b-i-1}\cdot t^{2i}) \\
        &= \sum_{i = 0}^{a+b} \binom{b-i-1}{a-2i} \Sq^{a+b-i}\circ \Sq^i \\
    \end{split}\]
\end{proof}

We now learn, why we did Adem relations, except for the fact that we are masochistic.

We note some more relations.
\[\Sq^3 = \Sq^1 \circ \Sq^2\]
\[\Sq^5 = \Sq^1 \circ \Sq^4\]
\[\Sq^6 = \Sq^2 \circ \Sq^4 + \Sq^5 \circ \Sq^1 = \Sq^2 \circ \Sq^4 + \Sq^1 \circ \Sq^4 \circ \Sq^1\]
\[\Sq^7 = \Sq^1 \circ \Sq^6 = \Sq^1 \circ \Sq^2 \circ \Sq^4\]

Everything here is a sum of composite of \(Sq^{2^i}\).

\textbf{Fact.} The Steenrod Algebra \(\cA_2\) is generated as a graded \(\IF_2\)-algebra by \(\Sq^{2^i}\) for \(i\geq 0\).

\begin{corollary}
    Let \(n\in \in_{\geq 1}\) s.t. \(n\) is not a power of 2. then \(\Sq^n\) is decomposable in \(\cA_2\), i.e. it is in the square of the ideal of positive dimensional elements. More concretely, \(\Sq^n\) is a sum of products of \(\Sq^i\)-s for lower degrees of \(i\).
\end{corollary}

\begin{proof}
    We can write \(n = 2^i(2k+1)\) for \(k \geq 1\). We then have a Adem relation:
    \[\Sq^{2^i} \circ \Sq^{n-2^i} = \sum_{j = 0}^{2^{i-1}} \binom{n-2^i-j-1}{2^i- 2j}\Sq^{n-j}\Sq^j\]
    For \(j = 0\), we have \(\binom{n-2^i-1}{2^i} \Sq^n\).
    
    \textbf{Claim.} \(\binom{n-2^i-1}{2^i}\) is odd.

    Assuming this then
    \[\Sq^n = \Sq^{2^i}\circ \Sq^{n-2^i} + \sum_{j = 1}^{2^{i-1}} \binom{n-2^i-j-1}{2^i-2j} \Sq^{n-j}\circ \Sq^j\]
    That would complete the proof.

    Going back to the claim:
    \[\begin{split}
        n-2^i - 1 &= 2^i(2k+1)-2^i - 1 \\
        &= 2^i(2k)-1  = 2^{i+1} \cdot k -1 \\
    \end{split}\]
    \(\binom{2^{i+1}\cdot k - 1}{2^i}\) is the coefficient of \(tw{2^i}\) in \((1+t)^{2^{i+1} k-1}\).
    \[\begin{split}
        ((1+t)^{2^{i+1}})\cdot (1+t)^{-1} &= (1+t^{2^{i+1}})^k (1+t+t^2 +\dots) \\
        &\equiv 1 \mod t^{2^{i+1}k}
    \end{split}\]
    So the coefficient of \(t^{2^i}\) is 1 (\(\mod 2\)).
\end{proof}

\textbf{Fact.} \(\IF_2\inner{\mathcal{Sq}^i : i\geq 1}\) The free associative graded \(\IF_2\)-algebra modding out the adem Relations becomes isomorphic to \(\cA_2\).

I.e. We found all \(\mod 2\)-stable cohomology actions and the Adem relations are also all the relations we could have found. We don't show this.

\begin{construction}
    Consider continuous based maps \(\alpha\colon S^m\to S^k\), \(\beta\colon S^n\to S^m\), \(k\geq 1\). Suppose that \(\alpha\circ \beta\) is null-homotopic. Let \(H\colon S^n\times [0,1]\to S^k\) be pa nullhomotopy of \(\alpha\circ \beta\). Let
    \[\bar H\colon CS^n= \frac{S^n\times [0,1]}{S^n\times \set1} \to S^k\]
    be the factorisation of \(H\). \(\bar H\) and \(\alpha\) are compatible to glue to a continuous map
    \[\alpha\cup H\colon C(\beta) = S^m\cup \beta C(S^n)\to S^k\]
    Let \(C(\alpha,\beta, H)\) denote the mapping cone of \(\alpha \cup \bar H \colon C(\beta) \to S^k\).

    Some pictures on what this means. This is a \(4\)-cell CW-complex.
\end{construction}

\textbf{Claim.} \(\eta^2 \neq 0\) in \(\pi_2^{\mathrm{St}}\)

\begin{proof}
    We argue by contradiction. If \(\eta^2 = 0\) in \(\pi_2^{\mathrm{st}}\), then for some \(n \geq 2\), the following composite
    \[S^{n+2}\xrightarrow{\eta} S^{n+1}\xrightarrow{\eta} S^n\]
    where formally we have \(\Sigma^{n-1}\eta\) and \(\Sigma^{n-2}\eta\). So we could study \(C(\eta, \eta, H)\).

    By a picture we get \(\Sq^2\circ \Sq^2 \colon H^n (C(\eta, \eta, H) \IF_2) \to H^{n+4}(C(\eta, \eta, H), \IF_2)\) is an isomorphism between \(1\)-dimensional \(\IF_2\) vector spaces, so \(\Sq^2 \circ \Sq^2 \neq 0\).

    But Adem relations tell us \(\Sq^2 \circ \Sq^2 = \Sq^3 \circ \Sq^1\) and that map factors through a zero-group for \(C(\eta, \eta, H)\) and is hence not 0.
\end{proof}

\textbf{Fact.} We have \(\eta^3 \neq 0\), but \(\eta^4 = 0\) in \(\pi_4^{\mathrm{St}}\).

The analoguos argument shows that some other products of Hopf maps are stably essential:

\begin{example}
    \(\eta \in \pi_1^{\mathrm{st}}\), \(\sigma\in \pi_7^{\mathrm{st}}\).
    \[\eta \circ \sigma \neq 0 \in \pi_8^{\mathrm{st}}\]

    By contradiction, if a null homotopy exists, we could build \(C(\eta, \sigma, H)\), but this is not possible due to a picture I did not draw.
\end{example}

We also get that the following are non-zero

\begin{center}
    \begin{tabular}{c|c}
        Adem relation & Product of Hopf maps \\\hline 
        \(\Sq^1\Sq^4 = \Sq^4\Sq^1 + \Sq^2\Sq^3\) & \(2\nu\) \\
        \(\Sq^1 \Sq^8 = \Sq^8\Sq^1+ \Sq^2\Sq^7\) & \(2\sigma\) \\
        \(\Sq^2\Sq^2 = \Sq^3\Sq^1\) & \(\eta^2\) \\
        \(\Sq^2\Sq^8 = \Sq^9\Sq^1 + \Sq^8\Sq^2+\Sq^4\Sq^6\) & \(\eta\sigma\) \\
        \(\Sq^4\Sq^4 = \Sq^7\Sq^1 + \Sq^6\Sq^2\) & \(\nu\nu\) \\
        ?? & \(\sigma\sigma\) \\
    \end{tabular}
\end{center}
We have that \(2\eta, \eta\nu, \nu\sigma\) are zero due to spectral sequences.


\newLecture{7.1.2026}

\section{Simplicial sets vs. topological spaces}

\enquote{Simplical sets model the homotopy theory of topological spaces}

So far we had

\[\begin{tikzcd}
    \mathbf{TOP} \ar[r, "S", bend left] & \mathbf{ssets} \ar[l, "\abs \cdot", bend left]
\end{tikzcd}\]

These functors descend to inverse equations of homotopy categories:
\[\begin{tikzcd}
    Ho(\text{CW-complexes}) \ar[r, phantom, "\cong"] & \mathbf{TOP}[w.eq^{-1}] \ar[r, bend left, "S", "\cong"'] & \mathbf{ssets}[w.eq^{-1}] \ar[l, "\abs\cdot", "\cong"', bend left] \ar[r, phantom, "\cong"] & Ho(\text{sset}^\text{Kan}) \\
\end{tikzcd}\]
where we \enquote{localize} categories, as one can localize rings.

\begin{defi}{simplicial sets}{}
    \(\Delta = \) (category of finite totally ordered sets) \(=\) category with objects \([n] = \set{0,1, \dots, n}\), \(n\geq 0\), and morphisms all weakly monotone maps.

    \(\mathbf{sset} = \mathbf{Fun}(\Delta^{\text{op}}, \mathbf{sets}) =\) category of functors \(X\colon \Delta^{\text{op}}\to \mathbf{sets}\) and natural transformations as morphisms.

    Notation: \(X_n = X([n])\), for \(\alpha\colon [n] \to [m]\) a morphism in \(\Delta\), \(\alpha^* = X(\alpha)\colon X_n \to X_m\).

    An element of \(X_n\), \(X\) a simplicial set, is called an \(n\)-simplex of \(X\).

    We call
    \begin{itemize}
        \item \(s_i\colon [n] \to [n-1]\) the unique surjective map with \(s_i(i) = s_i(i+i) = i\)
        \item \(d_i\colon [n-1]\to [n]\) the unique injective map with \(i \not \in \Img(d_i)\).
    \end{itemize}

    An \(n\)-simplex \(x \in X_n\) of a simplicial set is degenerate if the following equivalent conditions hold:
    \begin{enumerate}
        \item \(\exists 0 \leq i \leq n-1, y \in X_{n-1}\) such that \(x = s_i^*(y)\).
        \item there is a surjective morphism \(\alpha\colon [n]\to [m]\), \(m < n\) and \(z \in X_m\) such that \(x = \alpha^*(z)\).
        \item there is a non-injective morphism \(\beta\colon [n] \to [k]\) and \(w\in X_k\) such that \(x = \beta^*(w)\).
    \end{enumerate}

    A simplex is non-degenerate, if it is not degenerate.

    Later we will see the preferred CW-structure on \(\abs{X}\) has \(n\)-cells indexed by the non-degenerate \(n\)-simplices of \(X\).

    The simpilcial \(n\)-simplex is the represented simplicial set \(\Delta[n] = \Delta^n = \Delta(\_, [n])\).

    The Yoneda lemma yields for every simplicial set \(X\) and every \(x \in X_n\), there is a unique morphism of simplicial sets \(X^\flat \colon \Delta^n\to X\) such that
    \[x^\flat_n\colon \Delta^n([n]) = \Delta([n], [n])\to X^n, \quad x_n^\flat(\Id_{[n]}) = x.\]
    We will call \(x^\flat\) the characteristic morphism associated with \(x\). for \(\alpha\colon [m]\to [n]\), \(x_n^\flat(\alpha) = \alpha^*(x)\).

    The boundary \(\partial\Delta^n\) of \(\Delta^n\) is the simplicial subset given by \((\partial\Delta^n)_m = \set{\alpha\colon [m]\to [n], \alpha \text{ is not surjective}}\).
\end{defi}

\begin{example}
    For \(\alpha\colon [n]\to [m] \in (\Delta^n)_m\) is non-degenerate iff \(\alpha\) is injective.
\end{example}

\subsection{Minimal representation in geometric realizations.}

\begin{defi}{topological \(n\)-simplex}{}
    We define
    \[\nabla^n = \set{(t_0,\dots, t_n)\in\ir^{n+1} | t_i \geq 0, t_0+\dots+t_n = 1}\]
    or equivalently
    \(\nabla^n\) is the conveq hull of the standard basis vectors \(e_i = (0, \dots, 0, 1, 0, \dots, 0)\in \ir^{n+1}\) where the \(1\) is at the \(i\)-th place, starting with \(0\).
\end{defi}

This extends to a covariant functor
\[\nabla^\bullet\colon \Delta\to \mathbf{Top}\]
given by \([n]\mapsto \nabla^n\), \((\alpha\colon [m]\to [n])\mapsto \nabla(\alpha) = \alpha_*\colon \nabla^m\to \nabla^n\) the unique affine linear map s.t. \(\alpha_*(e_i) = e_{\alpha_i}\) concretely given by \(\alpha_*(s_0, \dots, s_m) = (t_0, \dots, t_n)\), \(t_i = \sum_{\alpha(j) = i} s_j\).

\begin{defi}{geometric realization}{}
    The geometric realization of a simplicial set \(X\) is the space
    \[\abs X = (\coprod_{n\geq 0} X_n \times \nabla^n)/\sim\]
    where \(X_n\) has the discrete topology, so the \(X_n \times \nabla^n = \coprod_{x\in X_n}\set{x}\times \nabla^n\) is a topological disjoint union.

    \(\sim\) is the equivalence relation on \(\coprod_{n\geq 0} X_n \times \nabla^n\) generated by
    \[X_m \times \nabla^m \ni (\alpha^*(x), s) \sim (x, \alpha_*(s)) \in X_n\times \nabla^m\]
    for every \(\alpha\colon[m]\to [n]\), \(x\in X_n\), \(s= (s_0, \dots, s_m)\in \nabla^m\).
\end{defi}

The generating relations are neither symmetric nor transitive, which means its hard to understand when two parts in \(\coprod_{n\geq 0} X_n\times \nabla^n\) are equivalent.

Geometric realization becomes a covariant functor \(\abs\cdot\colon \mathbf{ssets}\to \mathbf{Top}\) by \(f\colon X\to Y\) morphism of simplicial sets, \(\abs f\colon \abs X\to \abs Y\) is defined by
\[\abs f[y,s]\colon [f_m(y), s]\]
for \((y,s)\in X_m\times \nabla^m\). We need to check that this definition is compatible with the equivalence relation.

\(\abs X\) is also the coend of the functor \(\Delta^{text{op}}\times \Delta \to \mathbf{Top}, ([m], [n])\mapsto X_m\times \nabla^n\)\footnote{no idea what a coend is.}.

\begin{proposition}
    Let \(X\) be a simplicial set. Then
    \begin{enumerate}
        \item Every equivalence class under the relation \(\sim\) has a unique representative of minimal dimension.
        \item An element \((y,s)\in X_n\times \nabla^n\) is the minimal representative of its equivalence class if and only if  a) \(y\) is non-degenerate and b) \(s\) is an interior point of \(\nabla^n\), i.e. all \(s_i > 0\).
        \item If \((x,t)\in X_l\times \nabla^l\) is the minimal representative of its equivalence class and \((y,s) \in X_n\times \nabla^n\) is equivalent to \((x,t)\) then there is a unique trible \((\delta, \sigma, u)\) consisting of
        \begin{itemize}
            \item an injective morphism \(\delta\colon [k]\to [n]\)
            \item a surjective morphism \(\sigma\colon [k]\to [l]\)
            \item an interior point \(u\in \nabla^k\) s.t.
        \end{itemize}
        \(\delta^*(y) = \sigma^*(x)\), \(s = \delta_*(u)\), \(t = \sigma_*(u)\).
    \end{enumerate}
\end{proposition}

\begin{proof}
    This is more combinatorial than topological. We introduce notation
    \[X^{\text{nd}}_l =\text{ set of non-degenerate } l\text{-simplices of } X\]
    \[\mathrm{int}(\nabla^l) = \set{(t_0, \dots, t_l)\in \nabla^l: t_0 > 0, \dots, t_l > 0}\]
    A key step: construction of a map
    \[\rho\colon \coprod_{n\geq 0} X_n \times \nabla^n \to \coprod_{l\geq 0} X^{\text{nd}}_l \times \mathrm{int}(\nabla^l)\]
    such that \(\rho(y,s) \sim (y,s)\) for all \((y,s)\in \coprod_{n\geq 0} X_n\times \nabla^n\).

    We consider any pair \((y,s) \in X_n \times \nabla^n\), \(s = s_0, \dots, s_n\). since \(s_i\geq 0, s_0+ \dots + s_n = 1\) there is a tleast one \(0 \leq i\leq n\) such that \(s_i > 0\). Suppose that \(k+1\) of the numbers \(s_0, s_1, \dots, s_n\) are positive. \(0 \leq k \leq n\). Let \(v = (v_0, \dots, v_n)\) be the tuple obtained from \((s_0, \dots, s_n) = s\) by deleting all 0's eand keeping the positive coordinates in order. There is a unique injective morphism \(\delta\colon [k]\to [n]\) whose image are those indices \(i\) s.t. \(s_i > 0\).
    Then \(\delta_*(u) = s\) in particular \(u \in \mathrm{int}(\nabla^k)\). \((y,s)\sim (d^*(y), u)\).

    In exercise 8.1 we proved there is a unique pair \((\sigma, x)\) with \(\sigma\colon [k]\to [l]\) surjective morphism \(x\in X_l^{\text{nd}}\) s.t. \(d^*(y) = \sigma^*(x)\).

    Then \((\delta^*(y), u) = (\sigma^*(x),u) \sim (x, \sigma_*(u))\), which is non-degenerate and internal. We define
    \[\rho(y,s)\coloneq (x, \sigma_*(u))\in X_l^{\text{nd}}\times \mathrm{int}(\nabla^l)\]

    \textbf{Claim.} If \((y,s)\in X_n\times \nabla^n\) and \((\bar y, \bar s) \in X_{\bar n} \times \nabla^{\bar n}\) are equivalent, then \(\rho(y,s) = \rho(\bar y, \bar s)\). It suffices to show this when \((y,s)\) and \((\bar y, \bar s)\) are elementarily equivalent, i.e. there is a morphism \(\alpha\colon [n]\to [\bar n]\) s.t. \(y = \alpha^*(\bar y), \bar s = \alpha_*(s)\).
    \[(y,s) = (\alpha^*(\bar y), s) \sim (\bar y, \alpha_*(s)) = (\bar y, \bar s)\]
    We let \((\delta\colon [k]\hookrightarrow[n], u, \sigma\colon [k]\twoheadrightarrow[l], x)\) be the data produced in the construction of \(\rho(y,s)\). We choose a factorisation \(\alpha\circ \delta = \bar \delta\circ \bar\sigma\) for some injective morphism \(\bar\delta \colon [\bar k] \to [n]\) and surjective morphism \(\bar \sigma\colon [k]\to [\bar k]\). This is unique. Then
    \[\bar s = \alpha_*(s) = \alpha_*(\delta_*(u)) = \bar \delta_*(\bar\sigma_*(u))\]
    Since \(u\) is an interior point of \(\nabla^k\), \(\bar\sigma_*(u)\) is an interior point of \(\nabla^{\bar k}\). So \((\bar\delta, \bar \sigma_*(u))\) is the data in step 1 of th econstruction of \(\rho(\bar y, \bar s)\). Now we write \(\bar \delta^*(\bar y) = \hat\sigma^*(\hat x)\) for some surjective morphism \(\hat \sigma\colon [k]\to [\bar l]\) and \(\hat x \in X_{\bar l}\) non degenerate. \(\sigma^*(x) = \delta^*(y)= \delta^*(\alpha^*(\bar y)) = \bar\sigma^*(\bar \delta^*(\bar y)) = \bar \sigma^*(\hat\sigma^*(\hat x)) = (\hat \sigma \bar \sigma)^*(\hat x)\). This witnesses \(\sigma^*(x) = (\hat \sigma\bar\sigma)^*(\hat x)\) in two ways as a degeneracy of non-degenerate simplices. Since such a representation is unique, we conclude that \(l = \bar l, x = \hat x, \sigma = \hat \sigma\bar \sigma\). From this we conclude \(\bar \delta^*(y) = \hat\sigma^*(x) = \) So
    \[(\bar \delta, \sigma_*(u), \hat \sigma, x)\]
    is the data for the construction of \(\rho(\bar y, \bar s)\). So
    \[\rho(\bar y,\bar s) = (x,\hat\sigma_*(\bar\sigma_*(u))) = (x, \sigma_*(y)) = \rho(y,s)\]

    \begin{enumerate}
        \item Suppose \((y,s)\) is of minimal dimension of all pairs in its equivalence class.
        \[[n] \xhookleftarrow{\delta} [k] \xrightarrow{\sigma} [l]\]
        so \(n = l, n = k= l\) and \(\delta = \id_{[n]} = \sigma\), \(u = s\), \(\rho(y,s) = y,s\). If \((y', s')\) is another representative of minimal dimension, then
        \[(y,s) = \rho(y,s) = \rho(y',s') = (y',s')\]
        \item The proof of 1. shows that \(\rho(y,s)\) is the minimal representative in its equivalence class and \(\rho(y,s) \in X_l^{\text{nd}}\times \mathrm{int}(\nabla^l)\). So if \((y,s)\) is of minimal dimension in its class, then \(y\) is nondegenerate and \(s\) is interior. Conversely, if \(y\) is non-degenerate and \(s\) interior, \((y,s) = \rho(y,s)\) which is the minimal representative in its class.
        \item Let \((\delta, u, \sigma, x)\) be the data from the calculation of \(\rho(y,s)\), the minimal representative in the class of \((y,s)\), Then those data have the properties in 3. by construction.
    \end{enumerate}
\end{proof}

\newLecture{12.1.2026}

\begin{corollary}
    Let \(f\colon X \to Y\) be a morphism of simplicial sets, such that \(f_n\colon X_n \to Y_n\) is injective for all \(n \geq 0\) (this is precisely a monomorphism in \(\mathbf{sset}\). You could show this as a exercise)
    \begin{enumerate}
        \item For every non-degenerate \(x\in X_n\), the simplex \(f_n(x) \in Y_n\) is also non-degenerate.
        \item The continuous map \(\abs f\colon \abs X\to \abs Y\) is injective.
    \end{enumerate}
\end{corollary}

\textbf{Note.} We will later see that this injective map is even the inclusion of a subcomplex.

\begin{proof}
    \begin{enumerate}
        \item Let \(x\in X_n\) be non-degenerate. Suppose by contradiction that \(f_n(x)\) is degenerate, i.e. \(f_n(x) = s_i^*(y)\) for some \(y \in Y_{n-1}\). Then
        \[f_n(s_i^*(d_i^*(x))) = s_i^*(d_i^*(f_n(x))) = s_i^*(d_i^*(s_i^*(y))) = s_i^*(y) =  f_n(x)\]
        Since \(f_n\) is injective, \(s_i^*(d_i^*(x)) = x\), so it is degenerate, contradicting the hypothesis.
        \item Let \((x,t \in X_n\times \nabla^n)\) and \((x',t') \in X_m \times \nabla^m\) the unique minimal representatives in two equivalence classes that have the same image under \(\abs f\colon \abs X\to \abs Y\).
        \[[f_n(x), t] = \abs f[x,t] = \abs f[x',t'] = [f_m(x'), t']\]
        So \((f_n(x),t)\sim (f_m(x'), t')\). By minimality \(x, x'\) are non-degenerate. By \((i)\), \(f_n(x)\) and \(f_m(x')\) are non degenerate. So \((f_n(x), t)\) and \((f_m(x'), t')\) are the minimal representatives in their class.

        By uniqueness of minimal representatives, \(m= n\), \(f_n(x) = f_m(x) = f_m(x'), t= t'\). Since \(f_m\) is injective, also \(x = x'\) so \(\abs f\) is injective.
    \end{enumerate}
\end{proof}

\begin{corollary}
    For every simplicial set \(X\) the followving composite is surjective:
    \[\coprod_{n\geq 0} X_n^{\text{non-deg}}\times \nabla^n \hookrightarrow \coprod_{n\geq 0} \nabla^n \twoheadrightarrow \abs X\]
    If \(X\) has only finitely many non-degenerate simplices, then \(\abs X\) is quasi-compact.
\end{corollary}

\textbf{Remark.} \(\coprod_{n\geq 0} X_n ^{\mathrm{non-deg}}\times \mathrm{int}(\nabla^n)\to \coprod_{n\geq 0} X_n\times \nabla^n \to \abs X\) is continuous and bijective, but not a homeomorphism except for constant simplicial sets.

\textbf{From an exercise.} \(\abs \Delta^m \cong \nabla^m\) by mutually inverse homeomorphisms: \([\alpha, t] \mapsto \alpha_*(t)\) for \(\alpha\in (\Delta^m)_n, t \in \Delta^m\) and \(t\mapsto [\id_{[n]}, t]\)

\(\partial\nabla^m = \text{ boundary of } \nabla^m = \set{(t_0,\dots, t_m)\in \nabla^m : \text{ some } t_i = 0}\)

\begin{proposition}
    The composite
    \[\abs{\partial\Delta^m} \xrightarrow{\abs \incl} \abs {\Delta^m} \xrightarrow{\cong} \nabla^m\]
    is a homeomorphism onto \(\partial(\nabla^m)\).
\end{proposition}

\begin{proof}
    By a previos corollary, this is a continuous injection. \(\Delta^m\) only has finitely many non-degenerate simplices, hence so does \(\partial\Delta^m\), so \(\abs{\partial\Delta^m}\) is quasicompact. So the inclusion is a closed embedding, and hence a homeomorphism onto its image. For
    \[d_i\colon [m-1] \to [m]\in (\partial\Delta^m)_{m-1}\]
    its image is precisely \(\set{(t_0, \dots, t_m)\in \nabla^m: t_i = 0}\).

    So the image of \(\abs{\partial\Delta^m}\) is precisely \(\bigcup_{i = 0, \dots, n} \set{(t_0, \dots, t_m) : t_i = 0} = \partial(\nabla^m)\).
\end{proof}

\subsection{The preferred CW-structure on \(\abs{X}\)}

\begin{defi}{Simplicial skeleta}{}
    Let \(X\) be a simplicial set, \(m \geq 0\). The \(m\)-skeleton \(\sk^m(X)\) of \(X\) is the simplicial subset with
    \[(\sk^m X)_n = \set{x\in X_n : x = \alpha^*(y) \text{ for some } y \in X_m}\]
    with \(\alpha\colon [n] \to [m]\). This is the smallest simclicial subset of \(X\) that contains \(X_m\).
\end{defi}

\begin{example}
    \(X\) is \(0\)-dimensional, i.e. \(X = \sk^0 X\) iff \(X\) is constant, i.e. all \(\alpha^*\colon X_m \to X_n\) are bijective.

    \(\Delta^m\) is \(m\)-dimensional, i.e. \(\sk^m(\Delta^m) = \Delta^m\), because \(\alpha = \alpha^*(\id_{[n]})\) \(\id_{[n]}\in (\Delta^m)_m\).
    \(\partial\Delta^m\) is \((m-1)\)-dimensional, i.e. \(\sk^{m-1}(\partial\Delta^m) = \partial\Delta^m = \sk^{m-1}(\Delta^m)\).
\end{example}

\begin{proposition}
    Let \(X\) be a simplicial set \(m \geq 0\).
    \begin{enumerate}
        \item For \(n \leq m\) \((\sk^m X)_n = X_n\)
        \item For \(n > m\) every simlex in \((\sk^m X)_n\) is degenerate
        \item \(\sk^m X\subseteq \sk^{m+1} X\)
        \item \(X\) is a colimit of the sequence \(\sk^0 X\subseteq \sk^1 X \subseteq \sk^2 X\subseteq \dots\)
        \item For \(f\colon X \to Y\) a morphism of simplicial sets, \(f(\sk^m X) \subseteq \sk^m(Y)\)
    \end{enumerate}
\end{proposition}

\begin{proof}
    \begin{enumerate}
        \item For \(n \leq m\) we can choose morphism in \(\Delta\) \(\alpha\colon [n] \to [m], \sigma\colon [m] \to [n]\) \(\sigma\circ \alpha = \id_{[n]}\). Then for every \(x \in X_n\)
        \[x = (\sigma\circ \alpha)^*(x) = \alpha^*(\underbrace{\sigma^*(x)}_{\in X_m})\in (\sk^m X)_n\]
        \item Let \(n > m\), \(x = \alpha^*(y)\in X_n\), since \(\alpha\colon [n]\to [m], y \in X_m \in (\sk^m X)_n\) Because \(n > m\) \(\alpha\) cannot be injective, so there is some \(0 \leq i \leq n-1\) such that \(\alpha(i) = \alpha(i+1)\) so \(\alpha = \beta\circ s_i\) for some \(\beta\colon [n-1]\to [m]\) so \(x = \alpha^*(y) = s_i^*(\beta^*(y))\) so \(x\) is degenerate
        \item Let \(x \in (\sk^m X)_n\) so \(x = \alpha^*(y)\) for some \(y\in X_m\), some \(\alpha\colon [n]\to [m]\). Then
        \item \[x = \alpha^*(y) = (s_0 \circ d_q \circ \alpha)^*(y) = (d_0 \circ \alpha)^*(\underbrace{s_0^*(y)}_{\in X_{m+1}})\]
        so \(x \in (\sk^{m+1}X)_n\)
        \item Colimits and limits in functor categories are objectwise. So colimits of simplicias sets are objectwise. So we must show thot for all \(n \geq 0\)
        \[(\sk^0 X)_n \subseteq (\sk^1 X)_n \subseteq\dots\]
        is a colimit diagram of sets. But from \(n\) onwards this is just identities.
        \item Let \(x \in (\sk^m X)_n\), i.e. \(x = \alpha^*(y), y \in X_m, \alpha\colon [n] \to [m]\). Let \(f \colon X \to Y\) be a morphism of simplicia sets. Then \(f_n(x) = f_n(\alpha^*(y)) = \alpha^*(f_m(y))\in Y_m\) so \(f_n(x)\in (\sk^m Y)_n\).
    \end{enumerate}
\end{proof}

We note \(\sk^m\colon \mathbf{sset}\to \mathbf{sset}\) is a functor and the inclusions \(\sk^m X \to \sk^{m+1} X\) are natural transformations of functors.

\textbf{Note.} Let \(X\) be a simplicial set, \(Y\subseteq X\) simplicial subset, \(Y_{m-1} = X_{m-1}\) ore equally \(\sk^{m-1} Y = \sk^{m-1}X\). Then the following square commutes for all \(x \in X_m\)
\[\begin{tikzcd}
    \partial\Delta^m & \sk^{m-1}(\Delta^m)\ar[r, phantom, "\subseteq"] & \Delta^m \ar[d, "x^\flat"] \\
    \sk^{m-1} Y & \sk^{m-1} X \ar[r, hook] & X \\
\end{tikzcd}\]
He outspeeded me.

So the characteristic morphism \(x^\flat \colon \Delta^m \to X\) sends \(\partial \Delta^m\) into \(Y\).

\begin{proposition}
    Let \(X\) be a simplicial set, \(m \geq 0\). Let \(Y \subseteq X\) be a simplicial subset such that \(Y_{m-1} = X_{m-1}\). Suppose moreover that for all \(n > m\), all simplices in \(X_n \setminus Y_n\) are degenerate.
    \begin{enumerate}
        \item The commutative square
        \[\begin{tikzcd}
            \coprod_{x\in X_m \setminus Y_m} \partial \Delta^m \ar[r, hook]\ar[d] & \coprod_{x \in X_m\setminus Y_m} \Delta^m \ar[d, "x^\flat"] \\
            Y\ar[r, hook] & X \\
        \end{tikzcd}\]
        is a pushout in the category of simplicial sets.
        \item The square above after applying \(\abs\cdot\) is a pushout of spaces.
        \item The realization \(\abs X\) can be obtained from \(\abs Y\) by attaching \(m\)-cells indexed by \(X_m\setminus Y_m\) along the boundaries
    \end{enumerate}
\end{proposition}

\begin{proof}
    All the work goes into proofing 1. That is combinatorics.
    \begin{enumerate}
        \item Because simclicial sets are a functor category, all colimits --- such as pushouts and \(\coprod\) --- are objectwise. So we must show that for all \(k \geq 0\), the following is a pushout of sets.
        \[\begin{tikzcd}
            \coprod_{x \in X_m \setminus Y_m} (\partial\Delta^m)_k \ar[r, hook] \ar[d] & \coprod_{x \in X_m \setminus Y_m} (\Delta^m)_k \ar[d, "(x^\flat)_k"] \\
            Y_k \ar[r, hook] & X_k \\
        \end{tikzcd}\]
        So it suffices to show that the right vertical map restricts to a bijection between the complements of the horizontal maps:
        \[(X_m\setminus Y_m) \times \set{(\Delta^m)_k\setminus (\partial\Delta^m)_k} \to X_k \setminus Y_k, \quad (x, \alpha)\mapsto \alpha^*(x)\]
        we rewrite \((X_m \setminus Y_m)\times \set{\alpha\colon [k] \twoheadrightarrow [m] \text{ surjective}} X_k\setminus Y_k\). We look at
        \begin{description}
            \item[\(k < m\)]  No surjections and \( Y_k = X_k\) so both sides are empty.
            \item[\(k = m\)] \((X_m\setminus Y_m)\times \set{\id_{[m]}} \xrightarrow{\cong} X_m \setminus Y_m\), \((x, \id_{[m]})\mapsto x\).
            \item[\(k > m\)] We consider \(x \in X_k \setminus Y_k\) and write \(x\) uniquely as \(\alpha^*(\bar x)\) for some surjective morphism \(\alpha\colon [k]\to [n]\), some non-degenerate simplex \(\bar x \in X_n\). Because \(x \not \in Y_k\), also \(\bar x \not \in Y_n\) So \(n \geq m\). Also \(n \leq m\), because \(X_n = Y_n\) and all simplices in \(X_n \setminus Y_n\) are degenerate, but \(\bar x\) is not. So \(x \in (\sk^m X)_k\)
        \end{description}
        \item \(\abs \cdot\colon \mathbf{sset} \to \mathbf{Top}\) is left adjoint to \(\cS\colon \mathbf{Top} \to \mathbf{sset}\). So \(\abs\cdot\) preserves all colimits.
    \end{enumerate}
\end{proof}

\begin{thm}{preferred CW-structure}{}
    Let \(X\) be a simplicial set.
    \begin{enumerate}
        \item The subspaces \(\abs{\sk^m X}\) for \(m \geq 0\) form a CW-structure on \(\abs X\)
        \item The \(m\)-cells of this CW-structure biject with the non-degenerate \(m\)-simplices of \(X\)
        \item Suppose that for \(n > m\), all \(n\)-simlices of \(X\) are degenerate. Then \(\abs X\) is \(m\)-dimensional.
        \item For any morhpism of simplicial sets \(f\colon X \to Y\), \(\abs f\colon \abs X\to \abs Y\) is cellular.
        \item If \(Y\) is a simplicial subset of \(X\), then \(\abs{\incl}\colon \abs Y\to \abs X\) identifies \(\abs Y\) with a subcomplex of \(\abs X\).
        \item Suppose \(X\) has only finitely many non-degenerate simplices. Then \(\abs X\) is a finite CW-complex.
    \end{enumerate}
\end{thm}

\begin{proof}
    \begin{itemize}
        \item[1. + 2.] \((\sk^{m-1} X)_{m-1} = X_{m-1} = (\sk^m X)_{m-1}\) and for \(n > m\), every \(n\)-simplex of \(\sk^m X\) is degenerate. So we can apply the proposition to \((X,Y) = (\sk^m X, \sk^{m-1} Y)\). So we get
        \[\begin{tikzcd}
            X_m^{\text{n.d.}} \times \abs{\partial\Delta^m} \ar[r]\ar[d] & X_m^{\text{n.d.}} \times \abs{\Delta^m} \ar[d] \\
            \abs{\sk^{m-1}X}\ar[r, hook] & \abs{\sk^m X} \\
        \end{tikzcd}\]
        So \(\abs{\sk^m X}\) contains \(\abs{\sk^{m-1} X}\) as a closed subspace and can be obtained by attaching \(m\)-cells indexed by \(X_m^{\text{n.d.}}\).
        \(X = \colim_\IN \sk^m X\) implies \(\abs X = \colim_\IN \abs{\sk^m X}\), i.e. \(\abs X\) has the weak topology.
    \end{itemize}
    The rest he deems to clear to elaborate on.
\end{proof}




\end{document}