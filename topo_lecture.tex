\documentclass[language=english]{TemplateLecture}

\renewcommand{\ProfName}{Stefan Schwede}
\renewcommand{\LectureName}{Algebraic Topology I}
\renewcommand{\Semester}{WiSe 2025/26}
\renewcommand{\mName}{Jan Malmström}

\begin{document}

\newLecture{13.10.2025}

\subsection*{Organizatorial}

For this term we will be doing unstable homotopy theory. Next term we will be doing stable homotopy theory. Note that there were 2 previous courses. Note that all important information is shared on the website \url{https://www.math.uni-bonn.de/people/schwede/at1-ws2526}. You can sign up for the previous topology courses und see the lecture videos for these courses there.

There are no lecture notes for this lecture specifically, but some similar materials are linked on the webpage.

Exercise sheets will be uploaded fridays and handed in 11 days later via eCampus. Registration for eCampus opens at 4 today.

For exam admission you will have to score 50\% of the points on the exercise sheets and have presented 2 exercises in tutorial.

The first exam will be written in the last week of semester.

I fear I will not be able to copy pictures here.

\newpage

\setcounter{chapter}{1}

\section{Blakiers-Massy theorem/Homotopy excision}

We start with a reminder on relative homotopy groups.

\begin{defi}{Relative Homotopy Groups}{relative Homotopy groups}
    Let \((X,A)\) be a space pair i.e. \(A\) is a subspace of a topological space \(X\). We write
    \[I = [0,1] \quad I^{n} = [0,1]^n \text{ the } n\text{-cube}\]
    \[\partial(I^n) = \text{ boundary of } I^n\]
    \[I^{n-1} \subseteq I^n\]
    via Inclusion on the first \(n-1\) coordinates.
    \[J^{n-1} = I^{n-1}\times \set{1} \cup (\partial I^{n-1})\times [0,1]\]
    He draws a picture for \(n = 2\).

    For \(n \geq 1\) the \(n\)-th relative homotopy groups \(\pi_n(X,A,x)\) is the set of triple homotopy classes of trible maps \(x \in A\subseteq X\)
    \[(I^n, \partial I^n,J^{n-1}) \to (X,A, \set{x})\]
    where a triple map takes each subset on the left into the subset on the right. A triple-homotopy must also conserve these conditions.

    For \(n \geq 2\) or \(n = 1\) and \(A = \set{x}\) the set \(\pi_n(X,A,\set{x})\) has a group structure by concatenation in the first coordinate. He again draws a picture.

    The group structure is commutative if \(n \geq 3\) or \(n = 2\) and \(A = \set{x}\).
\end{defi}

\begin{defi}{n-Connectedness}{n-Connectedness}
    Let \(n \geq 0\). A space pair \((X,A)\) is \emph{\(n\)-connected}, if the following equivalent conditions hold:
    \begin{enumerate}
        \item For all \(0 \leq q\leq n\) every pair map \((I^q, \partial I^q) \to (X,A)\) is homotpic relative \(\partial(I^q)\) to a map with image in \(A\)
        \item For all \(a \in A\), \(\incl_*\colon \pi_q(A,a) \to \pi_q(X,a)\) is bijective for \(q \leq n\) and surjective for \(q = n\).
        \item \(\pi_0(A) \to \pi_0(X)\) is bijective and for all \(1 \leq q \leq n\) the relative homotopy group
        \[\pi_q(X,A,x) \cong 0\]
    \end{enumerate}
\end{defi}

\begin{proof}
    You proof the equivalence using the LES\footnote{Long exact sequence} of homotopy groups.
\end{proof}


Let \(Y\) be a space, \(Y_1, Y_2\) open subsets of \(Y = Y_1 \cup Y_2\), \(Y_0 \coloneq Y_1\cap Y_2\).

Excision in homology shows that for all abelian groups \(B\), \(i \geq 0\)
\[H_i(Y_2; Y_0, B) \to H_i(Y,Y_1; B)\]
is an isomorphism.

Excision does not generally hold for homotopy groups, i.e. for \(x \in Y_0\)
\[\incl_*\colon \pi_i(Y_2, Y_0; x) \to \pi_i(Y, Y_1; x)\]
is \textbf{not} generally an isomorphism.

\enquote {Blakiers Massey theorem implies that excision holds for homotopy groups in a range.}

\begin{thm}{Blakiers Massey}{Blakiers Massey}
    Let \(Y\) be a space, \(Y_1, Y_2\) open subsets with \(Y= Y_1 \cup Y_2, Y_0 \coloneq Y_1 \cap Y_2\). Let \(p, q \geq 0\), such that for all \(y \in Y_0\)
    \[\pi_i(Y_1, Y_0, y) = 0 \text{ for all } 1 \leq i \leq p\]
    and
    \[\pi_i(Y_2, Y_0, y) = 0 \text{ for all } 1 \leq i \leq q\]
    Then for all \(y \in Y_0\), the map
    \[\incl_*\colon \pi_i(Y_2, Y_0, y) \to \pi_i(Y, Y_1, y)\]
    is an isomorphism for \(1 \leq i < p+q\) and surjective for \(i = p+q\). He notes how the referenced literature uses different indices. They have proofs in more detail and pictures, however Lücks script contains typos
\end{thm}

\begin{proof}
    Schwede explains he doesn't like the proof, it is to technical and not very enlightening.

    We define what cubes are

    Cubes in \(\ir^n\), \(n \geq 1\).
    \(a = (a_1, \dots, a_n) \in \ir^n\) the \enquote{lower left corner of the cube}

    \(\partial \in \IR_{\geq 0}\) \enquote{side length of the cube}

    \(L \subset \set{1, \dots n}\) \enquote{relevant dimensions}
    \[W = W(a, \delta, L) = \set{x = (x_1, \dots, x_n)\in \ir^n : a_i\leq x_1\leq a_i + \delta \text{ for all } i \in L, \; x_i = a_i \text{ for all } i \in \set{1, \dots n} \setminus L}\]\footnote{W weil Würfel}
    A face \(W'\) of \(W\) is a subset of the form
    \[W' = \set{x \in W : x_i = a_i \text{ for all } i \in L_0, x_i = a_i + \delta \text{ for all } i \in L_1}\]
    for some subsets \(L_0, L_1 \subseteq L\)

    Let \(1 \leq p \leq n\) we define two subsets of a cube \(W = ^(a, \delta, L)\).
    \[K_p(W) = \set{x \in W : x_i < a_i + \delta/2 \text{ for at least} p \text{values of} i \text{ in } L}\]
    We call these \enquote{\(p\) small coordinates}
    \[G_p(W) = \set{x \in W : x_i > a_i + \delta/2 \text{ for at least } p \text{ coordinates } i \text{ in } L}\]
    these are \enquote{\(p\) big coordinates}.

    For \(p > \dim(W), K_p(W) = G_p(W)= \emptyset \) If \(p+ q \geq \dim(W)\), then \(K_p(W) \cap G_q(W) = \emptyset\).

    He draws pictures.

    \begin{lem}{1.14}{1.14}
        It is Lemma 1.14 in Lücks Script.

        Let \((Y,A)\) be a space pair, \(W \subseteq \ir^n\) a cube, \(f \colon W \to Y\) continuous. Suppose that for some \(p\leq \dim(W)\), \(f^{-1}(A) \cap W' \subseteq K_p(W')\) for all proper\footnote{subcube of the boundary} faces \(W'\) of \(W\).
        
        Then there is a continuous map \(g \colon W \to Y\) homotopic to \(f\) relative \(\partial W\) such that all \(g^{-1}(A)\subseteq K_p(W)\)
    \end{lem}
    \begin{proof}
        Wlog: \(W = I^n = W(0, 1, \set{1,\dots ,n})\)

        Let \(I_2^n\) be the subcube \([0, 1/2]^n\). He draws a picture. \(x_4 = (1/4, \dots 1/4) \in I_2^n\). We define a continuous map \(h \colon I^n \to I^n\) by radical projection away from \(x_4\). Picture. Let \(r(y)\) be the ray from \(x\) to \(y\). We map all of \(r(y) \cap I^n\setminus I^n_2\) to the intersection point of \(r(y)\) and \(\partial I^n\) and the rest linearly extends as far as required.\footnote{I hope this description is clear, hard without the picture.}

        Obviously\footnote{Meaning he's too lazy to come up with formulas for the map.} \(h\) is homotopic relative boundary \(\partial I^n\) to the identity.

        We set \(g \colon f \circ h\colon I^n \to Y\), which is then homotopic relative \(\partial(I^n)\footnote{I am very inconsistent in remembering these parantheses with the boundary operator. Just imagine it always being as here.}\) to \(f\).

        It remains to show that \(g^{-1}(A) \subseteq K_p(W)\). Consider \(z \in I^n\) with \(g(z) \in A\).
        \begin{description}
            \item[Case 1] for all \(i = 1, \dots, n, z_1 < 1/2\), i.e. \(z \in I_2^n\), then \(z \in K_n(I^n) \subseteq K_p(I^n)\).
            \item[Case 2] There is an \(i \in \set{1, \dots,n}\), s.t. \(z \geq 1/2\). Then \(h(z) \in \partial(I^n)\). Let \(W'\) be some proper face of \(W\), with \(h(z) \in W'\). Since \(f(h(z)) = g(z)\in A\), by hypothesis, \(h(z) \in K_p(W')\), so \(h(z) < 1/2\) for at least \(p\) coordinates. By expansion\footnote{no idea if this is the word he wrote} property of \(h\), also \(p\) coordinates of \(z\) are small coordinates. 
        \end{description}
    \end{proof}

    \begin{proposition}\footnote{11.5 in Lücks notes}
        Let \(Y_1, Y_2\) be open subsets of \(Y, Y_0\coloneq Y_1 \cap Y_2\). Suppose that \((Y_1, Y_0)\) is \(p\)-connected, \((Y_2, Y_0)\) is \(q\)-connected. Let \(f \colon I^n \to Y\) be continuous. Let \(\cW = \set{W}\) be a subdivision of \(I^n\) into subcubes of the same side length s.t. for all \(W \in \cW\) \(f(W) \subseteq Y_1\) or \(f(W) \subseteq Y_2\). Then there is a homotopy \(h \colon I^n \times I \to Y\) with \(h_0 = f\) such that for all \(W \in \cW\):
        \begin{enumerate}
            \item If \(f(W) \subseteq Y_j, j \in {0,1,2}\), then \(h_t(W) \subseteq Y_j\) for all \(t \in [0,1]\)
            \item If \(f(W) \subseteq Y_0\), then \(h_t\rvert_{W} = f\rvert_W\), i.e. \(h\) is constant on \(W\).
            \item If \(f(W) \subseteq Y_1\), then \(h_1^{-1}(Y_1 \setminus Y_0)\subseteq K_{p+1}(W)\).
            \item If \(f(W) \subseteq Y_2\), then \(h_1^{-1}(Y_2\setminus Y_0) \subseteq G_{q+1}(W)\).
        \end{enumerate}
    \end{proposition}
    \begin{proof}
        We let \(C^k \subseteq I^n\) be the union of all cubes in \(\cW\) of dimension at most \(k\). We construct homotopies \(h[k]\colon C_k\times I \to Y\), such that for all \(W \in \cW, W \subseteq C_k\) conditions 1. to 4. hold, and \(h[k]\) is constant on \(C_{k-1}\times I\). Then the final \(h[n]\) does the job.

        \textbf{Note.} If \(W \in \cW\) and \(f(W)\subseteq Y_0\) and 2. holds, then also 3. and 4. hold.

        \[h^{-1}(Y_1\setminus Y_0) = h_1^{-1}(Y_2\setminus Y_0) = \emptyset\]
        If \(W \in \cW\), is such that \(f(W) \subseteq Y_1\) and \(f(W) \subseteq Y_2\), then \(f(W)\subseteq Y_1\cap Y_2 = Y_0\). So each \(W \in \cW\) is in excactly one of the following cases
        \begin{itemize}
            \item \(f(W) \subseteq Y_0\)
            \item \(f(W) \subseteq Y_1\) and \(f(W) \not\subseteq Y_1\)
            \item \(f(W) \subseteq Y_2\) and \(f(W)\not \subseteq Y_2\)
        \end{itemize}
        Inductive construction \(k = 0\), i.e. vertexes of the cubes \(w \in \cW\). If \(w\in Y_2\), take \(h[0]_t = \const_{w_0}\).

        Suppose \(f(W_0)\in Y_1\), but \(f(w_0) \not \in Y_2\). Since \(Y_1,Y_0\) is \(0\)-connected, there is a path \(\pi\colon I \to Y_1\) from \(w_0\) to a point in \(Y_0\). We take \(h[0]\) as the path on \(w_0\). Analoguos if \(f(W_0)\in Y_2\setminus Y_1\).

        \textbf{Inductive Step} Let \(W \in \cW\) be a cube of exact dimension \(k\). Then \(\partial W = W \cap C_{k-1}\). Since \((W, \partial W)\) has the HEP, we can extend the previous homotopy \(h[k-1]\rvert_{\partial W}\) to some homotopy on \(W\) relative to \(f\rvert_W\). Let this be \(h'[k]\colon C_k\times I \to Y\): this satisfies conditions 1. and 2. but not yet 3. and 4.

        We produce another homotopy \(h[k]''\) and set \(h[k] = h[k]' * h[k]''\).

        Consider a cube \(W \in \cW\) of dimension \(k\).

        If \(f(W) \subseteq Y_0\) set \(h[k]''\) as the constant homotopy on \(W\).

        If \(h[k]_1'(W) \subseteq Y_1\), but \(h[k]_1'(W) \not \subseteq Y_2\) there is a homotopy relative \(\partial W\) from \(h[k]_1'\) to a map \(f_1(W)\subseteq Y_0\).

        If \(k = \dim(W) > p\) the we use the lemma \ref{lem:1.14} for \(f = h[k]_1'\rvert_W\) and the resulting homotopy is \(h[k]''\rvert_W\).

        If \(h[k]_1'(W)\subseteq Y_2\) but \(h[k]_1'(W) \not \subseteq Y_1\), use the complement case of the lemma\footnote{rest of the proof next lecture. I am very sure some words won't make sense, as they were unreadable on the board.}
    \end{proof}

\newLecture{15.10.2025}

Now for the actual proof of Blakiers Massey
Let \(F(Y_1,Y, Y_2) = Y_1 \times_Y Y^{[0,1]} \times_Y Y_2\).

Let \(F(Y_1, Y_1, Y_0)\) be the subspace of those \(w \in F(Y_1, Y, Y_2)\) such that \(w([0,1])\subseteq Y_1\).

\begin{proposition}
    Assume that \((Y_1, Y_0)\) is \(p\)-connected, \((Y_2, Y_0)\) is \(q\)-connected. Then the pair \((F(Y_1, Y,Y_2), F(Y_1, Y_1, Y_0))\) is \((p+q-1)\)-connected.
\end{proposition}

\begin{proof}
    We consider a map of pairs
    \[\phi\colon (I^n \partial I^n) \to (F(Y_1, Y, Y_2), F(Y_1, Y_1, Y_0))\]
    for \(1 \leq n \leq p+q+1\). We want to homotop \(\phi\) relative \(\partial I^n\) to a map with image in \(F(Y_1, Y_1, Y_0)\). We use the adjoint
    \[\map(X\times [0,1], Z) \cong \map(X, Z^{[0,1]})\]
    We let \(\Phi\colon I^n \times I \to Y\) be the adjoint of \(\phi\), this is \emph{admissable}, in the sense that
    \begin{enumerate}
        \item \(\Phi(x, 0) \in Y_1\) for all \(x \in I^n\)
        \item \(\Phi(x,1) \in Y_2\) for all \(x \in I^n\)
        \item \(\Phi(x,t) \in Y_1\) for all \(x \in \partial I^n\), \(t \in [0,1]\)
    \end{enumerate}
    We want a homotopy of \(\Phi\) through admissable maps to a map \(\Phi' \colon I^n \times I \to Y\) such that \(\img(\Phi')\subseteq Y_1\).

    Apply Proposition 11.5 to \(\Phi\) (with \(n+1\) instead of \(n\)). This gives a homotopy through admissable maps to \(\Psi = g\).
    Let \(h\colon I^n \times I \times I \to Y\) be a homotopy witnessing this. \(h_0 = \Phi, h_1 F \Psi\).

    Consider the projection \(\pr\colon I^n \times I\to I^n\) away from the last coordinate.

    \textbf{Claim.} The image under \(\pr\) of \(\Psi^{-1}(Y\setminus Y_1)\) and \(\Psi^{-1}(Y\setminus Y_2)\) are disjoint.

    Suppose there is \( y \in I^n\) in the intersection of the preimages, so \(z_1 \in \Psi^{-1}(Y\setminus Y_1), z_2 \in \Psi^{-1}(Y\setminus Y_2)\) s.t \(\pr(x_1) = y = \pr(z_2)\). Let \(W = I^{n+1}\) be a subcube of the subdivision of Proposition 11.5 such that \(z_1 \in W\). Since \(z_1 \in \Psi^{-1}(Y\setminus Y_1)\), \(z_1 \in K_{p+1}(W)\), so \(y = \pr(z_1) \in K_p(\img(W))\). Analogous \(y = \pr(z_2) \in G_q(\img(W))\). Since \(p+q > n\), this is a contradiction and the claim is proven.

    \textbf{Claim.} The intersection of \(\pr(\Psi^{-1}(Y\setminus Y_1))\) with \(\partial I^n\) is empty since \(\Psi\) is admissable, and thus \(\Psi(\partial I^n)\subseteq Y_1\). So \(\pr(\Psi^{-1}(Y\setminus Y_1))\) and \(\pr(\Psi^{-1}(Y\setminus Y_2))\cup (\partial I^n)\) are two disjoint closed subsets of the compact space \(I^n\).

    So there is a continuous separating function \(\tau\colon I^n \to [0,1]\), s,t. \(\tau\equiv 0\) is \(\pr(\Psi^{-1}(Y\setminus Y_1))\) and \(\tau\equiv 1\) is \(\partial I^n \cup \pr(Y\setminus Y_2)\)
    
    We define another homotopy starting with \(\Psi\) by
    \[h\colon (I^n\times I)\times I \to Y\]
    by
    \[((x,t),s) \mapsto \Psi(x,(1-s)t + s\cdot t \tau(x))\]
    This is
    \begin{itemize}
        \item Homotopy through admissable maps
        \item \(h(x,t,0) = \Psi(x,t)\)
    \end{itemize}
    \textbf{Claim.} \(h(\_,\_, 1)\) has image in \(Y_1\).

    \[h(x,t,1) = \Psi(x, t\cdot\tau(x))\]
    \begin{itemize}
        \item if \(x \in \Psi^{-1}(Y\setminus Y_1)\), then \(\tau(x) = 0\), \(h(x,t,1) = \Psi(x,0) \in Y_1\).
        \item if \(x \in \Psi^{-1}(Y_1)\), then by admissability \(\Psi(x,t\cdot\tau(x))\in Y\).
    \end{itemize}
\end{proof}

For the actual proof \((Y_1, Y_0)\) \(p\)-connected \((Y_2, Y_0)\) \(q\)-connected

some diagram about something being Serre fibration

We compare wo Serre fibrations
\[\begin{tikzcd}
    F(\set{y_0}, Y_1, Y_0) & F(\set{y_0}, Y, Y_2) \\
    F(Y_1, Y_1, Y_0) \ar[r] & F(Y_1, Y, Y_2) \\
    Y_1  & Y_1 \\
\end{tikzcd}\]
partial 5-lemma shows that also the pair
\[(F(\set{y_0}, Y, Y_2), F(\set{y_0}, Y_1, Y_0))\]
is \((p+q-1)\)-connected.

The following square commutes for all \(n \geq 1\)
\[\begin{tikzcd}
    \pi_{n-1}(F(\set{y_0}, Y_1, Y_0), \const_{y_0})\ar[d, "\cong"] \ar[r] & \pi_{n-1}(F(\set{y_0}, Y, Y_1), x) \ar[d, "\cong"]\\
    \pi_n(Y_1, Y_0, y_0) \ar[r, "\id_*"] & \pi_n(Y, Y_1, y_0) \\
\end{tikzcd}\]
\end{proof}

\section{Feudenthal suspension theorem}
Section 6.4 in tom Deicks book

\begin{defi}{}{}
    Let \(X\) be a based space. The \emph{unreduced suspension} is \(X^\diamond = X\times [-1,1]/\sim\) where \(\sim\) identifies all points with second variable \(-1\) to \(S\) and all points with second variable \(1\) to \(N\). We use \(S\) as the base point of \(X^\diamond\).\footnote{south and north pole}
\end{defi}

\textbf{Note.} If \(X\) is well based, i.e. \(\set{x_0} \hookrightarrow X\) has the HEP, then the quotient map \(X^\diamond \sigma X = \frac{X\times [0,1]}{X\times \set{0,1}\cup \set{x_0}\times [0,1]}\) is a homotopy equivalence.

The suspension homomorphism \(S\colon \pi_k(X,x_0) \to \pi_{k+1}(X^\diamond, S)\) for \(k \geq 1\) is \(S[f\colon S^k \to X] \coloneq [f^\diamond \colon S^{k+1}= (S^k)^\diamond \to X^\diamond]\)

\begin{thm}{Freudenthal suspension theorem}{}
    Let \(X\) be an \((n-1)\)-connected space, \(n \geq 1\). Then the suspension homomorphism is \begin{itemize}
        \item bijective foor \(i\leq 2n-2\)
        \item surjective for \(i = 2n-1\)
    \end{itemize}
\end{thm}

\begin{proof}
    We cover \(X^\diamond\) by \(Y_1 = X^\diamond \setminus \set{N}\), \(Y_2 F X^\diamond \setminus \set S\). We use repeatedly that \(Y_1\) and \(Y_2\)0are contractible. Then \(X\to Y_1\cap Y_2 = Y_0\) is a homotopy equivalence. We claim without proof that the following diagram commutes:

    I couldn't keep up.

    So we may show that \(\pi_{i+1} (Y_2, Y_0, x_0)\to \pi_{i+1}(Y, Y_1, y_0)\) is bijective for \(i \leq 2n-2\) and surjective for \(i \leq 2n-1\).

    Because \(X\) is \((n-1)\)-connected
    \[\pi_{k+1}(Y_1, Y_0, x_0) \xrightarrow[\partial]{\cong} \pi_k(Y_0, x_0)\]
    So \(\pi_i(Y_1, Y_0, x_0) = 0\) for \(i \leq n\), i.e. \((Y_1, Y_0)\) is \(n\)-connected. Also \((Y_2, Y_0)\) is \(n\)-connected.

    BM gives \(\pi_k(Y_2, Y_0, y) \to \pi_k(X^\diamond, Y_1, y)\) is bjective for \(k < 2n-1\) and surjective for \(k = 2n\).

    Setting \(i = k-1\) ends the proof.
\end{proof}

Take \(X = S^n\)
\begin{corollary}
    The suspension homomorphism
    \[S\colon \pi_i(S^n, *) \to \pi_{i+1}(S^{n+1}, *)\]
    is bijective for \(i \leq 2n-2\) and surjective for \(i = 2n-1\).
\end{corollary}

\begin{corollary}
    For all \(n \geq 1\), \(\pi_n(S^n, *)\cong \IZ\), generated by the class of \(\id_{S^n}\). Moreover \(\deg\colon \pi_n(S^n, *)\to \IZ\) is an isomorphism.
\end{corollary}

\begin{proof}
    Induction on \(n\). For \(n= 1\) we have this by covering theory.

    \(n\geq 1\) By Freudenthal, the suspension homomorphism
    \[S\colon \pi_n(S^n, *), \to \pi_{n+1}(S^{n+1}, *)\]
    is surjective. The composite
    \[\pi_n(S^n, *) \xrightarrow{S} \pi_{n+1}(S^{n+1}, *) \xrightarrow{\deg} \IZ\]
    is bijective by induction. So \(S\) is also injective and \(\deg\) in one dimension higher is also an isomorphism.
\end{proof}

\textbf{Recall.} \(\pi_3(S^2, *) \cong \IZ\set{\eta}\), where \(\eta \colon S^3 \to S^2\) is the Hopf map.

\begin{proposition}
    For \(n \geq 3\), \(\pi_{n+1}(S^n, *) = \IZ/2\set{S^{n-2}(\eta)}\) cycles of order two.
\end{proposition}

\begin{proof}
    Only partial. For \(n \geq 2\), \(S\colon \pi_{n+1}(S^n, q) \mapsto \pi_{n+2}(S^{n+1}, S^{n+1}, *)\) is surjective.
    \[\iz\set{\eta} = \pi_3(S^2, *) \twoheadrightarrow \pi_4(S^3, *) \xrightarrow[\cong]{S} \pi_5(S^4, *)\dots\]
    \textbf{Claim.} \(S(2\eta) = 0\) in \(\pi_4(S^3, *)\). Which gives \(\pi_{n+1}(S^n, *)\) is either trivial or order 2.

    Consider the commutative square

    I did not manage to copy. Something complex conjugation

    \(\implies [\eta] = [\eta \circ \text{complex conjugation}] = [d \circ \eta]\) in \(\pi_3(S^2, * \cong \iz)\).

    If \(f\colon S^n \to S^n\) has degree \(k\), then precomposition of \(\pi_n(X,x)\to \pi_n(X,x)\) is multiplication by \(k\).

    Let \(c\colon S^1\to S^1\) be any map of degree \(-1\). Then \(c\wedge S^2\), \(S^1 \wedge d\): bothe have degree \(-1\). So \(c \wedge S^2 \sim S^1 \wedge d\)

    two diagrams I did not copy.

    We get \(S(\eta) = S(d \circ \eta)=S(\eta \circ (c\wedge S^1)) = S(-\eta) = - S(\eta)\) and hence \(2 \cdot S(\eta) = 0\).
\end{proof}

\newLecture{20.10.2025}

We make a bit of a preview\footnote{He just spammed random stuff, I don't think I copied enough for it to make sense.} for stable homotopy theory. Following Lücks notes today rather closely

\begin{defi}{}{Stable Homotopy Groups}
    Let \(X\) be a based space. The \(n\)-th stable homotopy group of \(X\) is the colimit \(\pi_n^S(X,*)\)
\[\pi_n(X,*)\xrightarrow{S} \pi_{n+1}(S^1 \wedge X, *) \xrightarrow{S} \pi_{n+2}(S^2 \wedge X, *)\dots\]
\end{defi}

Since \(S^k \wedge X\) is \((k-1)\)-connected, the suspension stabilises (i.e. \(S\) is isomorphism for \(\pi_{n+k}(S^k \wedge X, *)\) onwards).

\(\pi_n^S()\) is a functor from based spaces to abelian groups. and it is homotopy invariant. It comes with a natural transformation \(\pi_n(X,*) \to \pi_n^S(X,*)\).

\textbf{Preview:} \(\set{\pi_n^S}_{n \in \iz}\) form a generalized homology theory on based spaces.

If \((X,A)\) is a pair of based space, \(x \in A \subseteq X\), we define
\[\pi_n^S(X,A,*) \coloneq \pi_n^S(X\cup_A CA, *)\]
The map collapsing \(X\) is
\[X\cup_A CA \to \frac{X\cup_A CA}{X} \cong \frac{CA}{A} \cong A^\diamond\]
induces an connecting homomorphism
\[\partial\colon \pi_n^S(X,A,x) = \pi_n^S(X \cup_A CA, *) \xrightarrow{p_*} \pi_n^S(A^\diamond) \to \pi_n^S(A \wedge S^1) \xrightarrow{\cong} \pi_{n-1}^S(A)\]

The following sequence will then be exact:
\[\dots \pi_n^S(A,x)\xrightarrow{\incl_*} \pi_n^S(X,x,) \xrightarrow{\incl_*}\pi_n^S(X,A,*) \xrightarrow{\partial} \pi_{n-1}^S(A,*)\to \dots\]

Stable stems are the special case
\[\pi_n^S(S^0) = \colim(\pi_n(S^0, *) \to \pi_{n+1}(S^1, *)\to \dots)\]

\begin{center}
    \begin{tabular}{l|c|c|c|c|c|c|c|c|c}
        n & 0&1&2&3&4&5&6&7&8 \\\hline
        \(\pi_n^S = \pi_n^S(S^0)\) & \(\iz\) & \(\iz/2\) & \(\iz/2\) & \(\iz/24\) & \(0\)& \(0\) & \(\iz/2\) & \(\iz/240\) & \((\iz/2)^2\) \\\hline
        Generator & \(\id\) & \(\eta\) & \(\eta^2\) & \(\nu\), \(\eta^3 = 12 \nu\) & - & - & \(\nu^2\) & \(\sigma\) & \(\eta\sigma, \epsilon\)
    \end{tabular}
\end{center}

There is a graded-commutative ring structure on \(\pi_*^S = \set{\pi_n^S}_{n \in \iz}\). \(\pi_n^S \times \pi_m^S \to \pi_{n+m}^S\)

\[[f\colon S^{n+k}\to S^k\times S^k] \times [g \colon S^{m+l}\to S^l] = [SW{n+m+k+l} \to S^{n+k} \wedge S^{m+l} \xrightarrow{f\wedge g} S^{k+l}]\]

I missed a bit more

Nishidas theorem says: every pohitive dimensional element of \(\pi_*^3\) is nilpotent.

From Serre spectral sequence we will see: For \(m > n\geq 1\) \(\pi_m(S^n, *)\) is finite except \(\pi_{4k - 1}(S^{2k}, *) \cong \iz \oplus \) some finite group.

This was exercise 5.2 of last summer term: \emph{Hopf invariant} \(h\colon \pi_{2k-1}(S^k, *) \to \iz\) for \([f\colon S^{2k-1}\to S^k]\)
\[H^*(S^k\cup_f D^{2k}, \iz) = \begin{cases}
    \iz & i = 0,k,2k \\
    0 & \text{else}
\end{cases}\]
we have 
\(H^k(Cf, \iz) = \iz\set a\) \(H^{2k}(Cf, \iz) \cong \iz\set b\) and \(a \cup a = h(f) \cdot b\).

In general we can look at

\(t_k\colon S^{2k-1}\to S^k\vee S^k\) the attaching map of the \(2k\)-cell in the product minimal CW-structure of \(S^k \times S^k\). And we get \(h(t_k) = 2\).

The Hopf invariant 1 theorem tells us when we can realize Hopf invariant 1 and here we see that we always find Hopf invariant 2.

\section{Hurewicz-theorem}

This is a \enquote{trivial}\footnote{meaning 1.5 lectures of intermediate steps} Corollary of the Blakiers-Massey theorem.

\begin{defi}{}{Hurewicz-Homomorphism}
    Let \(X\) be a based space. Choose an orientation of \(S^n\), \(n \geq 1\), i.e. \([S^n] \in H_n(S^n, \iz)\). The Hurewicz homomorphism
    \[h\colon \pi_n(X,x) \to H_n(X;\iz)\]
    is defined by
    \[h[f\colon S^n\to X] = H_n(f, \iz)[S^n].\]
    This is a group homomorphism.
\end{defi}

\textbf{Group Homomorphism:} Let \(\Delta\colon S^n \to S^n \vee S^n\) be a pinch map.
Group structure on \(\pi_n(X,x)\) is given by \([f]+[g] = [S^n\xrightarrow{\Delta}S^n\vee S^n \xrightarrow{f+g} X]\). Then
\[
\begin{tikzcd}
    \pi_n(X, x) \arrow[d, "\cong"'] \arrow[rd, "h"] & \\
    \pi_n(X, y) \arrow[r, "h"] & H_n(X, \mathbb{Z})
\end{tikzcd}
\]
\[([f]+ [g])_* = [f]_* + [g]_* \colon H_n(S^n) \to H_n(X)\]
So Hurewicz is a homomorphism.

\begin{lem}
    Let \(w\colon [0,1]\to X\)0be a path from \(x = w(0)\) to \(y = w(1)\). Then the following commutes:
    \[\begin{tikzcd}
        \pi_n(X,x) \ar[d, "\cong"] \ar[rd, "h"] & \\
        \pi_n(X,y) \ar[r, "h"] & H_n(X, \iz) \\ 
    \end{tikzcd}\]
\end{lem}

\begin{proof}
    \(f\) and \(f *w\) are freely homotopic \(h_n(\_, \iz)\) is homotopy invariant
\end{proof}

For \(n = 1\) we have Poincaré: \(X\) based path conneted, \(h\colon \pi_1(X,x)\to H_1(X,\IZ)\) is surjective with kernel the commutator subgroup or equivalently
\[\pi_1(X,x)_{ab} \xrightarrow{cong} H_1(X,\iz)\]

\begin{thm}{}{Hurewicz}
    Let \(X\) be an \((n-1)\)-connected space, \(n\geq 2\). Then the Hurewicz homomorphism
    \[h\colon\pi_n(X,x) \to H_n(X,\iz)\]
    is an isomorphism.
\end{thm}

\begin{proposition}[11.9 in Lücks notes]
    Let \(m,n \geq 0\), let
    \[\begin{tikzcd}
        A \ar[r, "f"] \ar[d, "i"] & B \ar[d, "\bar i"]\\
        X \ar[r, "\bar f"] & Y \\
    \end{tikzcd}\]
    be a pushout square of spaces. Suppose \(i\colon A \to X\) is a \emph{cofibration}, i.e. a closed embedding with the HEP.
    \begin{enumerate}
        \item If \(f\) is \(n\)-connected, then so is \(\bar f\)
        \item If \(f\) is \(n\)-connected, and \(i\) is \(m\)-connected, then for all \(a \in A\), \(\pi_i(f, \bar f) \colon \pi_i(X,A, a) \to \pi_i(Y, B, f(a))\) is bijective for \(1 \leq i < m+n\) and surjective for \(i = m+n\).
    \end{enumerate}
\end{proposition}

\begin{proof}
    \enquote{Basically reduction to Blakiers-Massey}

    We can replace \(X, B\) and \(Y\) up to homotopy equivalence by appropriate mpping cylinders:
    \[\mathrm{cyl}(f) = A\times [0,1]\cup_{A\times 1 , f} B\]
    We get a homotopy commutative diagram
    \[\begin{tikzcd}
        A \ar[r]\ar[d] & \mathrm{Cyl}(f) \ar[d]  \ar[rd, "\sim"]& \\
        \mathrm{Cyl}(i) \ar[r]\ar[rd, "\sim"] & \mathrm{cyl}(i) \cup_A \mathrm{cyl}(f)\ar[rd, "\sim"] & B\ar[d, "\bar i"] \\
        & X \ar[r, "\bar f"] & Y \\
    \end{tikzcd}\]
    This diagram does not commute. The homotopy equivalences shown are not trivial.

    We apply Blakiers-Massey then to the following open subspace of \(W = \mathrm{cyl}(i)\cup_A \cyl(f)\).

    \(W_2 = [0,1/2) \times A\cup_{A\times 0} \cyl(f)\), \(W_1 = \cyl(i) \cup_{A\times 0} [0, 1/2)\). \(W_0 = W_1\cap W_2 \sim A\)

    Now having \(f\) is \(n\)-connected gives \((W_2, W_0)\) is \(n-connected\). \(i\) being \(m\)-connected gives \((W_1, W_0)\) is \(m\)-connected.

    Now BM gives \(\pi(W_1, W_0, a)\to \pi_i(W, W_2, a)\) is bijective for \(1 \leq i \leq m+n-1\) and surjective for \(i = m+n\).

    Using the homotopy equivalences, we translate this back.
\end{proof}

\begin{proposition}[11.1 in Lücks]
    Let \(m,n \geq 0\), let \(c\colon A \to X \) be a cofibration. Suppose that \(i\) is \(m\)-connected and \(A\) is \(n\)-connected. Then 
    \[\pi_k(\pr)\colon \pi_k(X,A,a) \to \pi_k(X/A, *)\]
    for all \(a \in A\) is bijective for \(1 \leq k \leq m+n\) and surjective for \(k = m+n+1\)
\end{proposition}

\begin{proof}
    We consider the pushout
    \[\begin{tikzcd}
        A \ar[d, "i"]\ar[r, "j", "(n+1)\text{-con}"'] & C(A) = A\times [0,1]/A\times 1 \ar[d] \\
        X \ar[r] & X \cup_A CA \ar[r, "\sim"] & X/A\\
    \end{tikzcd}\]

    \begin{remark}
        \(X\) \(k\)-connected \(\Lra\) \(\set{*}\to X\) \(k\)-connected \(\Lra\) \(X\to \set{*}\) is \((n+1)\)-connected.
    \end{remark}
\end{proof}

\begin{proposition}[11.12 for Lück]
    Also an exercise (1.2) Let \(X,Y\) be well-pointed spaces. Let \(m,n \geq 1\). Let \(X\) be \(m\)-connected, \(Y\) \(n\)-connected. Then
    \begin{enumerate}
        \item The inclusion \(X\vee Y\to X\times Y\) induces isomorphisms
        \[\pi_k(X\vee Y, *) \to \pi_k(X\times Y, *)\]
        for all \(0 \leq k \leq m+n\).
        \item \(\pi_k(X\times Y, X\vee D, *)\) and \(\pi_k(X\wedge @, *)\) are trivial for all \(0 \leq k\leq m+n+1\).
        \item The map \((pr_*^X, pr_*^Y)\colon \pi_k(X\vee Y, x) \to \pi_k(X, x_0)\times \pi_k(Y, y_0)\) is an isomorphism for all \(1 \leq k \leq m+n\).
    \end{enumerate}
\end{proposition}



\end{document}